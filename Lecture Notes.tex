\documentclass[12pt]{article}
\usepackage{amsthm}
\usepackage{fullpage}
\usepackage{amsfonts}
\usepackage{amssymb}
\usepackage{times}
\usepackage{graphicx}
\usepackage{extsizes}
\usepackage[margin=1in,footskip=0.25in]{geometry}
\vfuzz=2pt

% some "funny lines" referred to later:
\newtheorem{thm}{Theorem}[section]
\newtheorem{cor}[thm]{Corollary}
\newtheorem{lem}[thm]{Lemma}
\newtheorem{prop}[thm]{Proposition}
{ \theoremstyle{remark}\newtheorem*{remark}{Remark} }
\theoremstyle{definition}
\theoremstyle{definition}
\newtheorem{defn}[thm]{Definition}


\def\BB{\mathcal{B}}
\def\OO{\mathcal{O}}
\def\ZZ{\mathbb{Z}}
\def\NN{\mathbb{N}}
\def\CC{\mathbb{C}}
\def\QQ{\mathbb{Q}}
\def\RR{\mathbb{R}}
\def\FF{\mathbb{F}}
\def\Tr{\mbox{Tr}}
\def\sgn{\mbox{sgn}}
\def\nm{\mbox{N}}
\def\tk{\theta_K}
\def\Hom{\mbox{Hom}}
\def\Aut{\mbox{Aut}}
\def\Inn{\mbox{Inn}}



\begin{document}

\title{MATH 8510 Galois Theory}
% or if you want, simply \title{Title of the article}
\author{LU Junyu}
%\email{luj9@myumanitoba.ca}

% \begin{abstract}
% Here's a \LaTeX\ template.
% \end{abstract}

\maketitle

\section*{Week 1}
\label{Week 1}
\subsection{Warm up}

Let's agree on some facts and conventions from elementary abstract algebra, in particular those with polynomial rings before we dig into Galois theory.

\smallskip

A ring is always commutative with multiplicative identity 1 unless otherwise stated. $R^*$ is the multiplicative group of units in $R$ and $R^\times = R\setminus \{0\}$. We can use these two notations interchangeably when $R$ is a field.

\smallskip

Let $F$ be a field. A polynomial ring $F[X]$ with an indeterminate $X$ is an $F$-vector space with basis $1,X,X^2,...,X^n,...$ with the multiplication $$(\sum_i a_iX^i)(\sum_j b_j X^j) =  \sum_k (\sum_{i+j=k}a_ib_j)X^k,$$ where $X^0$ is defined to be 1. Alternatively, we can identify $R[X]$ with $$R^{(\NN)}=\{(a_i): a_i\in R, a_i=0 \mbox{ for all but finitely many }i\in\NN\}$$ in an obvious way. But usually, we want to say $R$ embeds into $R[X]$ although the most formal way is to identify $R$ with a subring of $R[X]$. The degree function has the following properties:
\begin{enumerate}
    \item $\deg(f+g) \leq \max(\deg f,\deg g),$
    \item $\deg(fg)=\deg f+\deg g.$
\end{enumerate}

There are plenty results by arguing over the degree of a polynomial. We have $(R[X])^* =  R^*$ if $R$ is an integral domain. We have the division algorithm on $R[X]$.

\begin{thm}
Let $F$ be a commutative ring. Then $F[X]$ is a PID if and only if $F$ is a field.
\end{thm}

Hence or otherwise, $\ZZ[X]$ is not a PID. Indeed, $\langle 2, X\rangle$ is an example of an ideal that cannot be generated by a single polynomial. $K[X,Y]$ is not a PID as $\langle X, Y \rangle$ is not principal.

\begin{thm}
    An ideal in a PID is prime if and only if it is maximal.
\end{thm}

\smallskip

\begin{defn}
    If $f(X)\in F[X]$ where $F$ is a field, then a \textbf{root} of $f$ in $F$ is an element $\alpha\in F$ such that $f(\alpha)=0$.
\end{defn}

Given a polynomial $f[X]\in F[X]$ and any $u\in F$, the division algorithm give us: $$f(X)= q(X)(X-u)+f(u).$$

And lying in the center of proving that every finite subgroup of $F^\times$ is cyclic is counting the roots of polynomial $X^n-1$.

\begin{thm}
    Let $F$ be a field and $f[X]\in F[X]$ a polynomial of degree $n$. Then $f$ has at most $n$ roots.
\end{thm}

\begin{defn}
    Let $F$ be a field. A nonzero polynomial $p(X) \in F[X]$ is said to be \textbf{irreducible} over $F$ (or \textbf{irreducible} in $F[X]$) if $\deg p \geq 1$ and there is no factorization $p=fg$ in $F[X]$ with $\deg f < \deg p$ and $\deg g < \deg p$.
\end{defn}

A quadratic or cubic polynomial is irreducible in $F[X]$ if and only if it has no root in $F$.

\begin{thm}[Gauss's Lemma]
    A polynomial $f(X)\in \ZZ[X]$ is irreducible if and only if it is irreducible over $\QQ[X]$.
\end{thm}

\begin{thm}[Eisenstein's Criterion]
    Let $f(X) = a_0+a_1X+...+a_nX^n\in \ZZ[X]$ be a polynomial over integers with $a_n\not= 0$. Suppose that there exists a prime $p$ such that \begin{enumerate}
        \item  $p\nmid a_n$, 
        \item  $p\mid a_i$  for  $i=0,1,...,n-1$,
        \item  $p^2\nmid a_0$.
    \end{enumerate}
    Then $f(X)$ is irreducible over $\ZZ[X]$.
\end{thm}

A typical application of Eisenstein's Criterion is to prove the irreducibility of the $p$-th cyclotomic polynomial $\Phi_p(X) = \frac{X^p-1}{X-1}$, where $p$ is a prime. The idea is to apply the criterion to $\Phi(X+1)$.

\begin{thm}
    Let $F$ be a field and $f(x)$ a polynomial in $F[X]$. Then $(f(X))$ is a prime ideal in $F[X]$ if and only if $f(X)$ is irreducible. Equivalently, $f$ is irreducible if and only if $K[X]/(f)$ is a field.
\end{thm}

Making use of above results, we finally reach the very last theorem which functions as a cornerstone in many arguments.

\begin{thm}
    Let $k$ be a field and $f[X]$ a monic irreducible polynomial in $k[X]$ of degree $d$. Let $K=k[X]/I$, where $I=(f)$, and $\beta = X+I\in K$. Then:
    \begin{enumerate}
        \item $K$ is a field and $k'=\{a+I: a\in k\}$ is a subfield of $K$ isomorphic to $k$,
        \item $\beta$ is a root of $g$ in $K$,
        \item if $g(X)\in k[X]$ and $\beta$ is a root of $g$ in $K$, then $f\mid g$ in $k[X]$,
        \item $f$ is the unique monic irreducible polynomial in $k[X]$ having $\beta$ as a root,
        \item $1,\beta,\beta^2,...,\beta^{d-1}$ forms a basis of $K$ as a vector space over $k$ and so $\dim_k(K)=d$.
    \end{enumerate}
\end{thm}

\subsection{Extensions of fields}

Most of this course will involve studying fields relative to certain subfield which we feel we understand better. For example, if $\alpha\in\CC$ is the root of some polynomial with coefficients in $\QQ$, we might wish to study $\QQ(\alpha)$, the smallest subfield of $\CC$ containing $\alpha$ and all of $\QQ$. Certainly, if we want to understand how "complicated" the number $\alpha$ is, it makes sense to consider how "complicated" the field $\QQ(\alpha)$ is as an extension of $\QQ$. If $F \subset E$ are fields, we will denote denote the extension by $E/F$ (this just means that $F$ is a subfield of $E$, and that we're considering $E$ relative to $F$, in particular, $E/F$ is not a quotient or anything too formal). Note that often we will consider $E$ to be an extension of $F$ even if $F\nsubseteq E$, as long as there is an obvious embedding of $F$ into $E$ (an embedding is a homomorphism with is injective). 

We will make a lot of use of the observation that if $E/F$ is an extension
of fields, then we may view $E$ as a vector space over $F$.


% \newpage
% \begin{thebibliography}{99} % don't worry about the 99

% % \bibitem{ExamplePaper}
% % E.~Edelman, ``The probability that a random real Gaussian matrix has $k$ real eigenvalues, related distributions, and the circular law'', {\em J.~Multivariate~Anal.} {\bf 60} (1997), no.~2, 202--232.

% % \bibitem{ExampleBook}
% % H.~L.~Montgomery and R.~C.~Vaughan, {\em Multiplicative Number Theory I: Classical Theory}, Cambridge University Press (2007).

% \end{thebibliography}

\end{document}
