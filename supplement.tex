\documentclass[12pt,reqno]{article}
\usepackage{fullpage}
\usepackage{amsfonts}
\usepackage{amsthm}
\usepackage{amssymb}
\usepackage{times}
\usepackage{graphicx}
\usepackage{mathtools}
\usepackage{pgfplots}
\usepackage[all,cmtip]{xy}
\usepackage{hyperref}
\vfuzz=2pt

\pgfplotsset{compat=1.17}


\DeclarePairedDelimiter\ceil{\lceil}{\rceil}
\DeclarePairedDelimiter\floor{\lfloor}{\rfloor}

\newtheorem{theorem}{Theorem}
\newtheorem{corollary}[theorem]{Corollary}
\newtheorem{lemma}[theorem]{Lemma}
\newtheorem{proposition}[theorem]{Proposition}
{\theoremstyle{remark}\newtheorem*{remark}{Remark}}
\theoremstyle{definition}
\newtheorem{definition}[theorem]{Definition}
\newtheorem{example}[theorem]{Example}
\newtheorem{innercustomgeneric}{\customgenericname}
\providecommand{\customgenericname}{}
\newcommand{\newcustomtheorem}[2]{
  \newenvironment{#1}[1]
  {
   \renewcommand\customgenericname{#2}
   \renewcommand\theinnercustomgeneric{##1}
   \innercustomgeneric
  }
  {\endinnercustomgeneric}
}
\newcustomtheorem{solution}{Solution}

\newcommand{\ccc}{\mathcal{C}}
\newcommand{\cc}{\mathbb{C}}
\newcommand{\zz}{\mathbb{Z}}
\newcommand{\ta}[1]{\langle #1 \rangle}
\newcommand{\ff}{\mathbb{F}}
\newcommand{\qq}{\mathbb{Q}}
\newcommand{\gal}{\text{Gal}}
\newcommand{\Mod}[1]{\ (\mathrm{mod}\ #1)}
\newcommand{\mymod}[3]{#1 \equiv #2 \Mod{#3}}
\newcommand{\nm}[1]{\text{N}(#1)}
\newcommand{\nmm}[2]{\text{N}_{#1}(#2)}

\begin{document}

\title{Inverse Galois Problems for $S_p$ and Abelian Groups}

\author{LU Junyu}

\maketitle

\noindent In this short article, we construct a field extension $E$ over the rationals $\qq$ with Galois group $\gal(E/\qq)\cong S_p$, $p$ prime, or $\gal(E/\qq)$ any finite abelian group. If it is the latter case, the extension $E$ is so constructed that it is a subfield of some cyclotomic extension.

Through all this article, $\zeta_n$ is a primitive $n$-th root and $\Phi_n(x)$ is the $n$-th cyclotomic polynomial, where $n$ is a positive integer.

% Cyclotomic extensions are of great interests in algebraic number theory. Among the most outstanding theorems is the Kronecker-Weber theorem, which says every abelian Galois extension of the rationals embeds into some cyclotomic extension.

\begin{lemma}
	Let $p$ be a prime. If a subgroup $G$ of the symmetric group $S_p$ contains a transposition and a $p$-cycle, then $G$ is the whole group $S_p$.
\end{lemma}

\begin{proof}
	After renaming elements, we can assume the transposition $\sigma=(1~ 2)$. We can write a $p$-cycle $\tau$ as $\tau=(1~ i_2 ~\cdots ~i_p)$ after rotations on $\tau$, if necessary. Now $i_j=2$ for some $2\leq j\leq p$, and then $\tau^{j-1}=(1~2~\cdots)$ is also a $p$-cycle. After renaming elements, we get $\sigma=(1~2), \tau=(1~2\cdots~p)$ and then $\sigma,\tau$ generate $S_p$.
\end{proof}

\begin{theorem}\label{symgal}
	Let $f\in\qq[x]$ be a monic irreducible polynomial of degree $p$, $p$ prime. If $f$ has precisely two complex roots and $p-2$ real roots, then the Galois group of $f$ is isomorphic to the symmetric group $S_p$.
\end{theorem}

\begin{proof}
	Fix an algebraic closure $\overline{\qq}\subset\cc$. Let $E$ be the splitting field of $f$ over $\qq$ and $\alpha$ one of the roots. Note that $E/\qq$ is a Galois extension and $\gal(E/\qq)$ is (isomorphic to) a subgroup of $S_p$. Since $f$ is irreducible, $[\qq(\alpha):\qq ]=p$ and so $p\mid \left[ E:\qq\right]=|\gal(E/\qq)|$. By Cauchy's theorem (or Sylow's theorem), $\gal(E/\qq)$ contains an element of order $p$. But the only elements in $S_p$ of order $p$ are $p$-cycles. Hence $\gal(E/\qq)$ contains a $p$-cycle. Note the complex conjugation exchanges the two complex roots of $f$ and fixes reals, so it is also an element in $\gal(E/\qq)$ and is a transposition indeed. Since $\gal(E/\qq)$ contains a transposition and a $p$-cycle, $\gal(E/qq)$ is the whole group $S_p$ by the lemma above.
\end{proof}

\begin{example}
	Probably the simplest example of a polynomial over $\qq$ with Galois group $S_n~ (n > 1)$ is $x^n-x-1$. This is proved in a paper by H.~Osada in J.~Number Theory, 25(1987), 230–238.
\end{example}

\begin{example}
	Let $p\geq 5$ be a prime. Define $f(x),g(x)\in \qq[x]$ as
	\[g(x)=(x^4+4)(x-2)(x-4)\cdots(x-2(p-2)),~f(x)=g(x)-2.\]

	If we draw $f,g$ on the plane, we see that $g(x)$ interests $x$-axis at $2,4,\dots,2(p-2)$ and that $g(x)>2$ for $x=3,5,7,\dots,2p-1$. The graph of $f$ is obtained by shifting down 2 units of that of $g$. Therefore, $f$ has precisely $p-2$ real roots. Write $f(x)$ as
	\[f(x)=x^p+d_{p-1}x^{p-1}+\dots+d_{0}.\]
	Then $d_{0}= 4k-2$ for some nonzero integer $k$ and hence $2^2\nmid d_{0}$ while it is easily seen that $2\mid d_{j}$ for $j=0,\dots,d-1$. By Eisenstein's criterion, $f$ is irreducible. And Theorem \ref{symgal} says the Galois group of $f$ over $\qq$ is $S_p$.
\end{example}


Now we move the the case where we want the Galois group be finite abelian. Recall from the classification on finite abelian groups, we can write a finite abelian group $G$ as \[G\cong \zz/p_1^{e_1}\times\dots\times \zz/p_r^{e_r},\] where $p_i$ are primes not necessarily distinct and $e_r$ are positive integers. And for two rings $R_1$ and $R_2$, we have \[(R_1\times R_2)^*=R_1^*\times R_2^*.\]

The following theorem is a special case of Dirichlet's theorem about primes in arithmetic progression. To be self-contained, we prove it using cyclotomic polynomials

\begin{theorem}
	Let $n>1$ be a positive integer. Then there are infinitely many primes $p$ such that $p\equiv 1\Mod{n}$.
\end{theorem}

\begin{proof}
	Let $\Phi_n(x)$ be the $n$-th cyclotomic polynomial.
	We first note that $\Phi_1(0)=-1$ and $\Phi_n(0)=1$ for $n\geq 2$. This can be easily done by induction on $n\geq 2$. Hence the constant term for $\Phi_n(x)$ is 1 when $n>1$.

	Claim: Let $p$ be a prime. If $p\mid \Phi_n(x_0)$ for some integer $x_0$, then $p\mid n$ or $p\equiv 1\Mod{n}$.


	Proof of Claim: Note that $p\mid \Phi_n(x_0)\mid x_0^n-1$. We must have $p\nmid x_0$. Let $k$ be the order of $x_0$ in $(\zz/p)^*$. Since $|(\zz/p)^*|=p-1$, we have $k\mid (p-1)$ and so $p\equiv 1\Mod{k}$. Since $x_0^n\equiv 1\Mod{p}$, we have $k\mid n$. If $k=n$, then $p\equiv 1\Mod{n}$ and we are done. If $k<n$, then $p\mid x_0^k-1$ implies $p\mid \Phi_d(x_0)$ for some $d\leq k<n$. Since $p$ also divides $\Phi_n(x_0)$, $x_0$ is a double root of $x^n-1$ when we regard it as a polynomial in $\ff_p[x]$. This can only happen if $p$ divides $n$.


	Assume that there are only finitely many primes $p\equiv 1\Mod{n}$. We define \[N=n\prod_{p \text{ prime, } p\equiv 1\Mod{n}} p.\] Then $N>n>1$ is well-defined. Consider the monic polynomial $\Phi_n(x)$. We have $\Phi_n(N^k)>1$ for some large enough integer $k$. Let $p$ be a prime divisor of $\Phi_n(N^k)$. Note the constant term of $\Phi_n(x)$ is 1 and then $\Phi_n(N^k)-1$ is a multiply of $N$. But $p\mid \Phi_n(N^k)$ implies $p\nmid \Phi_n(N^k)-1 $ and $p\nmid N$ and $p\nmid n$. It follows from the claim that $p\equiv 1\Mod{n}$. On the other hand $p\nmid N$ means $p$ is not any of the primes in the definition of $N$. Contradiction.
\end{proof}

We need one lemma more before going to construct abelian extensions.

\begin{lemma}
	Let $G$ be a finite abelian group. Then there is a surjective homomorphism $$\phi:(\zz/n)^*\to G$$ for some positive integer $n$.
\end{lemma}

\begin{proof}
	By the classification of finite abelian groups, we can write \[G\cong \zz/n_1\times\dots\times\zz/n_r,\] where $\zz/n_i$ is a cyclic group of order $n_i$.

	Since there are infinitely many primes $p\equiv 1\Mod{n_i}$, we can choose distinct primes $p_i$ such that $p_i=n_im_i+1$ for some positive integer $m_i$ for $i=1,\dots,r$. Now $(\zz/p_i)^*$ is a cyclic group of order $n_im_i$ and hence there is a surjection $\phi_i: (\zz/p_i)^* \to \zz/n_i$. Collecting all the surjections, we can define a surjective homomorphism
	\[\phi:(\zz/p_1)^*\times\dots\times (\zz/p_r)^*\to \zz/n_1\times\dots\times\zz/n_r,~ (a_1,\dots,a_r)\mapsto (\phi_1(a_1),\dots,\phi_r(a_r)).\]
	Note that $(\zz/p_1)^*\times\dots\times (\zz/p_r)^* = (\zz/p_1\times\dots\times \zz/p_r)^*$ and that by Chinese Remainder Theorem $\zz/p_1\times\dots\times \zz/p_r \cong \zz/(p_1\dots p_r)$. We get a surjection $\phi': (\zz/(p_1\dots p_r))^*\to G.$
\end{proof}

\begin{theorem}
	Let $G$ be a finite abelian group. Then there is a subfield $E$ of $\qq(\zeta_n)$, where $\zeta_n$ is a primitive $n$-th root for some positive integer $n$, such that $E$ is Galois over $\qq$ and $\gal(E/\qq)\cong G$.
\end{theorem}

\begin{proof}
	By the lemma above, we can find a positive integer $n$ such that there is a surjection $$\phi:(\zz/n)^*\to G.$$ Then the kernel $H=\ker(\phi)$ is a normal subgroup.

	Now let $E=\qq(\zeta_n)^H$ be the fixed subfield of $H$. Since $H$ is normal, by the fundamental theorem about Galois theory, $E/\qq$ is Galois and \[\gal(E/\qq) \cong \gal(\qq(\zeta_n)/\qq)/\gal(\qq(\zeta_n)/E) \cong (\zz/n)^*/H \cong G.\]
\end{proof}

Note the difference between this theorem and Kronecker-Weber theorem. Kronecker-Weber theorem says \emph{every} abelian extension over the rationals can be embedded into a cyclotomic extension, while we we constructed \emph{some} extension with Galois group a finite abelian group $G$ that happens to embed into a cyclotomic extension.

\begin{theorem}[Kronecker-Weber]
	Let $E/\qq$ be a finite Galois extension such that $\gal(E/\qq)$ is abelian. Then there is a root of unity $\zeta$ such that $E \subset \qq(\zeta)$.
\end{theorem}

And we will prove a special case of Kronecker-Weber theorem.

\begin{theorem}\label{webspe}
	Let $p>2$ be a prime. Then the only quadratic subfield over $\qq$ of $\qq(\zeta_p)$, where $\zeta_p$ is a primitive $p$-th root, is $M=\qq(\sqrt{p})$ if $p\equiv 1\Mod{4}$ and $M=\qq(\sqrt{-p})$ if $p\equiv 3\Mod{4}$.
\end{theorem}

\noindent Note this theorem leads immediately a corollary about quadratic extensions.

\begin{corollary}
	Let $E$ be a quadratic Galois extension over $\qq$. Then $E$ embeds to some cyclotomic extension.
\end{corollary}

\begin{proof}
	Note that $\sqrt{2}\in \qq(\zeta_8)$ and $i\in \qq(\zeta_4)$. We fix an algebraic closure $\qq\subset E\subset\overline{\qq}\subset\cc$. A quadratic extension $E$ over $\qq$ looks like $E=\qq(\sqrt{d})$ from some square-free integer $d$. We can do inductions on the prime factors of $$d=\pm \prod_{p_i\mid n} p_i,$$ with the observation that $\qq(\zeta_m)\subset\qq(\zeta_n)$ if $m\mid n$.
\end{proof}

We need a technical lemma before proving Theorem \ref{webspe}.

\begin{lemma}
	Let $p>2$ be a prime and $g$ a generator of $\ff_p^*$. Then the number of solutions of the equation $x^2+gy^2=r$ over $\ff_p$ for some $r\in \ff_p$ is given as
	\[ |\{(x,y)\in\ff_p^2|x^2+gy^2=r\}| =
		\begin{cases}
			1    & \quad \text{if } p\equiv 1\Mod{4} \text{ and } r=0,     \\
			p+1  & \quad \text{if } p\equiv 1\Mod{4} \text{ and } r\not=0, \\
			2p-1 & \quad \text{if } p\equiv 3\Mod{4} \text{ and } r=0,     \\
			p-1  & \quad \text{if } p\equiv 3\Mod{4} \text{ and } r\not=0.
		\end{cases}
	\]
\end{lemma}
\begin{proof}
	Consider the case $\mymod{p}{1}{4}$. The equation $x^2+gy^2=0$ has the trivial solutions only. If not, say $y\not=0$, then $g=-(x^2/y^2)=-(x/y)^2$ but this says $g^{(p-1)/2}=(x/y)^{p-1}=1$ contradicting to the assumption that $g$ is a generator of $\ff_p^*$. Therefore, the quadratic polynomial $T^2+g\in \ff_p[T]$ has no solution in $\ff_p$. Let $\alpha=\sqrt{-g}$ be one of its roots. Then $\ff_p[\alpha]/\ff_p$ is a Galois extension of degree 2. The norm of an element $a+b\alpha\in \ff_p[\alpha]$ is given as $\nm{a+b\alpha}=a^2+b^2g$. The norm mapping $\text{N}:\ff_p[\alpha]^\times\to \ff_p\times$ is a group homomorphism. The map is surjective since $\nm{\alpha}=g$ and the kernel is of size $|\ff_p[\alpha]^\times|/ |\ff_p^\times|=(p^2-1)/(p-1)=p+1$. Namely, for each $r\in \ff_p^*$ we will get $p+1$ solutions $x+y\alpha$ with $\nm{x+y\alpha}=x^2+y^2\alpha=r$.


	Now consider the case $\mymod{p}{3}{4}$. In this case, $-1$ is not a quadratic residue  and so $-g$ is, say $-g=\beta^2$ for some $\beta\in \ff_p^*$. We have $x^2+gy^2=(x-\beta y)(x+\beta y)=0$ if and only if $x=\pm\beta y$ and so we get $2p-1$ solutions. Now consider the polynomial $T^2-g\in \ff_p[x]$ and let $\alpha=\sqrt{g}$ be one of its roots and $\gamma\in \ff_p$ with $\gamma^2=-1$. Then the norm map \[\text{N}: \ff_p[\alpha]\to \ff_p, x+y\gamma \alpha \mapsto x^2-y^2\gamma^2 g =x^2+y^2 g\] is surjective since $\nm{\gamma\alpha}=g$. The restriction of $N$ on the non-zero-norm element is again a group homomorphism. The size of kernel is then $(p^2-(2p-1))/(p-1)=p-1$. Namely, there are precisely $p-1$ solutions for $\nm{x+y\gamma \alpha}=x^2+gy^2=r$ for each $r\not=0$. 
\end{proof}

\begin{proof}[Proof of Theorem \ref{webspe}]
	Note that $(\zz/q)^*=\ff_p^*$ is of order $p-1=2m$ for some positive integer $m$. Let $g$ be a generator of $\ff_p^*$ and $\zeta$ a primitive $p$-th root of unity. Then $\gal(\qq(\zeta)/\qq)\cong (\zz/q)^*$ and is generated by $\sigma$ who is defined by $\sigma(\zeta)=\zeta^g$. Hence $\ta{\sigma^2}$ is a subgroup of index 2 in $\gal(\qq(\zeta)/\qq)$. The fixed subfield $E=\qq(\zeta)^{\ta{\sigma^2}}$ is then a quadratic extension over $\qq$. Note that $\gal(\qq(\zeta)/\qq)$ is abelian and every subgroup is normal. Therefore, $E/\qq$ is Galois.

	Note that elements
	\begin{align*}
		\alpha & =\sigma^2(\zeta)+\dots+\sigma^{p-1}(\zeta)=\zeta^{g^2}+\zeta^{g^4}+\dots+\zeta^{g^{p-1}}=\sum_{i=1}^{\frac{p-1}{2}}\zeta^{g^{2i}} = \sum_{i=0}^{\frac{p-3}{2}}\zeta^{g^{2i}} \\
		\beta  & =\sigma(\alpha)= \sum_{i=0}^{\frac{p-3}{2}}\zeta^{g^{2i+1}}
	\end{align*} are invariant under $\sigma^2$. Because the terms $\zeta^{j}$ in $\alpha,\beta$ run out all possibilities of the form $\zeta^j$ for some $0<j<p$ as $g$ is a generator, we see either $\alpha\notin \qq$  or $\beta\not\in \qq$, otherwise, $\zeta$ would satisfy a polynomial of degree $< p-1$ in $\qq[x]$. Without loss of generality, we assume $E=\qq(\alpha)$.


	We want to construct a quadratic with $\alpha,\beta$ its roots: \[x^2-(\alpha+\beta)x+\alpha\beta.\] Note that \[\alpha+\beta =\sum_{i=0}^{\frac{p-3}{2}}\zeta^{g^{2i}}+\sum_{i=0}^{\frac{p-3}{2}}\zeta^{g^{2i+1}} =\sum_{i=1}^{p-1} \zeta^i =(\sum_{i=0}^{p-1} \zeta^i) -1=\Phi_p(\zeta)-1=-1.\] Hence indeed $\qq(\alpha)=\qq(\beta)$. And then
	\[\alpha\beta =  (\sum_{i=0}^{\frac{p-3}{2}}\zeta^{g^{2i}})(\sum_{j=0}^{\frac{p-3}{2}}\zeta^{g^{2j+1}})= \sum_{i=0}^{\frac{p-3}{2}}\sum_{j=0}^{\frac{p-3}{2}} \zeta^{g^{2i}+g^{2j+1}}= \sum \zeta^{x^2+gy^2},\] where $x=g^i$ and $y=g^j$ for $i,j=0,\dots,(p-3)/2$ and the number of such term $\zeta^{x^2+gy^2}$ is $(p-1)^2/4$. Note that \[(g^i)^2=g^{2i}=g^{2i+(p-1)}=(g^{i+\frac{p-1}{2}})^2.\]
	Extending the range of $x=g^i$ and $y=g^j$ to $i,j=0,\dots, p-2$, we get  \[4\alpha\beta=\sum_{x,y\in \ff_p^*} \zeta^{x^2+gy^2}.\]

	First consider the case $p\equiv 1\Mod{4}$. By the lemma above, we can count the number of solutions of $x^2+gy^2=r$. But we need to exclude the case where $x$ or $y$ is zero. If $r=0$, then the quadratic form $x^2+gy^2$ is non-isotopic, which means the only solution is $x=y=0$.Thus if $x,y\in \ff_p^*$, then $x^2+gy^2$ is never 0. Then in how many ways we can get $r\not=0$, the lemma above says there are $p+1$. But we have to exclude the case $x=0$ or $y=0$. If $r$ is a quadratic residue, then we get two solution for $x$ if $y=0$ and no solution for $y$ if $x=0$. If $r$ is not a quadratic residue, then we get no solution for $x$ if $y=0$ and two solution for $y$ if $x=0$. Either case, we get $p-1$ solutions for $x,y\in \ff_p^*$. And hence \[4\alpha\beta= (p-1)\sum_{i=1}^{p-1} \zeta^i=-(p-1).\]
	And the minimal polynomial of $\alpha,\beta$ is $x^2+x-(p-1)/4$ and then \[\pm(\alpha-\beta)=\sqrt{\Delta}=\sqrt{1^2-4 \cdot(-\frac{p-1}{4})}= \sqrt{p}\in E=\qq(\alpha).\] Therefore, $E=\qq(\sqrt{p})$.


	Now consider the case $p\equiv 3\Mod{4}$. In this case, the quadratic residue is isotopic, which means we get nontrivial solution for $x^2+gy^2=0$. The number of solutions is then $2p-1$ by the lemma above. But we have to exclude the trivial solution $x=y=0$. So indeed, there are $2p-2$ solution when $x,y\in \ff_p^*$. Exactly the same argument as last case, we get $p-3$ solutions for $x^2+gy^2=r$ for each $r\not=0$ when $x,y\in \ff_p^*$. Therefore, we have
	\[4\alpha\beta= 2p-2+(p-3)\sum_{i=1}^{p-1} \zeta^i = 2p-2-(p-3)=p+1. \]
	And the minimal polynomial of $\alpha,\beta$ is $x^2+x+(p+1)/4$ and then \[\pm(\alpha-\beta)=\sqrt{\Delta}=\sqrt{1^2-4 \cdot(\frac{p+1}{4})} =\sqrt{-p}\in E=\qq(\alpha).\] Therefore, $E=\qq(\sqrt{-p})$.
\end{proof}

\end{document}
