\documentclass[12pt]{report}
\usepackage{amsthm}
\usepackage{fullpage}
\usepackage{amsfonts}
\usepackage{amsmath}
\usepackage{amssymb}
\usepackage{times}
\usepackage{graphicx}
\usepackage[margin=1in,footskip=0.25in]{geometry}
\vfuzz=2pt

% some "funny lines" referred to later:
\newtheorem{thm}{Theorem}[section]
\newtheorem{cor}[thm]{Corollary}
\newtheorem{lemma}[thm]{Lemma}
\newtheorem{prop}[thm]{Proposition}
{\theoremstyle{remark}\newtheorem*{remark}{Remark}}
\theoremstyle{definition}
\newtheorem{defn}[thm]{Definition}
\newtheorem{example}[thm]{Example}

\def\AA{\mathfrak{A}}
\def\BB{\mathcal{B}}
\def\OO{\mathcal{O}}
\def\ZZ{\mathbb{Z}}
\def\NN{\mathbb{N}}
\def\CC{\mathbb{C}}
\def\QQ{\mathbb{Q}}
\def\RR{\mathbb{R}}
\def\FF{\mathbb{F}}
\def\im{\text{im}}
\def\Tr{\text{Tr}}
\def\sgn{\text{sgn}}
\def\nm{\text{N}}
\def\tk{\theta_K}
\def\Hom{\text{Hom}}
\def\Aut{\text{Aut}}
\def\Inn{\text{Inn}}
\def\aa{\alpha}
\def\bb{\beta}
\def\aaa{\mathfrak{a}}
\def\bbb{\mathfrak{b}}
\def\mm{\mathfrak{m}}
\def\Gal{\text{Gal}}
\def\char{\text{char}}
\def\irr{\text{irr}}
\def\of{\overline{F}}
\def\ok{\overline{K}}


\begin{document}

\title{MATH 8510 Galois Theory}
\author{LU Junyu}
\date{Winter 2021}
%\email{luj9@myumanitoba.ca}

% \begin{abstract}
% Here's a \LaTeX\ template.
% \end{abstract}

\maketitle

\tableofcontents

\chapter*{Week 1}
\setcounter{chapter}{1}
\section{Review on polynomial rings}

Let's agree on some facts and conventions from elementary abstract algebra, in particular those with polynomial rings before we dig into Galois theory.

\smallskip

A ring is always commutative with multiplicative identity 1 unless otherwise stated. $R^*$ is the multiplicative group of units in $R$ and $R^\times = R\setminus \{0\}$. We can use these two notations interchangeably when $R$ is a field.

\smallskip

Let $F$ be a field. A polynomial ring $F[X]$ with an indeterminate $X$ is an $F$-vector space with basis $1,X,X^2,\cdots,X^n,\cdots,$ with the multiplication $$(\sum_i a_iX^i)(\sum_j b_j X^j) =  \sum_k (\sum_{i+j=k}a_ib_j)X^k,$$ where $X^0$ is defined to be 1. Alternatively, we can identify $R[X]$ with $$R^{(\NN)}=\{(a_i)_{i\in\NN}: a_i\in R, a_i=0 \mbox{ for all but finitely many }i\in\NN\}$$ in an obvious way. But usually, we want to say $R$ embeds into $R[X]$ although the most formal way is to identify $R$ with a subring of $R[X]$.  We will also use notations like $F[x], k[x]$ and $k[X]$ for polynomial rings as long as there is no confusion.

The degree function has the following properties:
\begin{enumerate}
    \item $\deg(f+g) \leq \max(\deg f,\deg g),$
    \item $\deg(fg)=\deg f+\deg g.$
\end{enumerate}

There are plenty results by arguing over the degree of a polynomial. We have $(R[X])^* =  R^*$ if $R$ is an integral domain. We have the division algorithm on $R[X]$.

\begin{thm}
Let $F$ be a commutative ring. Then $F[X]$ is a PID if and only if $F$ is a field.
\end{thm}

Hence or otherwise, $\ZZ[X]$ is not a PID. Indeed, $\langle 2, X\rangle$ is an example of an ideal that cannot be generated by a single polynomial. $K[X,Y]$ is not a PID as $\langle X, Y \rangle$ is not principal.

\begin{thm}
    An ideal in a PID is prime if and only if it is maximal.
\end{thm}

\smallskip

\begin{defn}
    If $f(X)\in F[X]$ where $F$ is a field, then a \emph{root} of $f$ in $F$ is an element $\alpha\in F$ such that $f(\alpha)=0$.
\end{defn}

Given a polynomial $f[X]\in F[X]$ and any $u\in F$, the division algorithm give us: $$f(X)= q(X)(X-u)+f(u).$$

And lying in the center of proving that every finite subgroup of $F^\times$ is cyclic is counting the roots of polynomial $X^n-1$.

\begin{thm}
    Let $F$ be a field and $f[X]\in F[X]$ a polynomial of degree $n$. Then $f$ has at most $n$ roots.
\end{thm}

\begin{defn}
    Let $F$ be a field. A nonzero polynomial $p(X) \in F[X]$ is said to be \emph{irreducible} over $F$ (or \emph{irreducible} in $F[X]$) if $\deg p \geq 1$ and there is no factorization $p=fg$ in $F[X]$ with $\deg f < \deg p$ and $\deg g < \deg p$.
\end{defn}

A quadratic or cubic polynomial is irreducible in $F[X]$ if and only if it has no root in $F$.

\begin{thm}[Gauss's Lemma]
    A polynomial $f(X)\in \ZZ[X]$ is irreducible if and only if it is irreducible in $\QQ[X]$.
\end{thm}

\begin{thm}[Eisenstein's Criterion]
    Let $f(X) = a_0+a_1X+\cdots+a_nX^n\in \ZZ[X]$ be a polynomial over integers with $a_n\not= 0$. Suppose that there exists a prime $p$ such that \begin{enumerate}
        \item  $p\nmid a_n$, 
        \item  $p\mid a_i$  for  $i=0,1,\cdots,n-1$,
        \item  $p^2\nmid a_0$.
    \end{enumerate}
    Then $f(X)$ is irreducible in $\ZZ[X]$.
\end{thm}

A typical application of Eisenstein's Criterion is to prove the irreducibility of the $p$-th cyclotomic polynomial $\Phi_p(X) = \frac{X^p-1}{X-1}$, where $p$ is a prime. The idea is to apply the criterion to $\Phi(X+1)$.

\begin{thm}
    Let $F$ be a field and $f(X)$ a polynomial in $F[X]$. Then $(f(X))$ is a prime ideal in $F[X]$ if and only if $f(X)$ is irreducible. Equivalently, $f$ is irreducible if and only if $F[X]/(f)$ is a field.
\end{thm}

\begin{thm}
    Any ring homomorphism between two fields is injective.
\end{thm}

\noindent One sentence proof: there is no non-trivial proper ideal in a field.

\section{Extensions of fields}

Most of this course will involve studying fields relative to certain subfield which we feel we understand better. For example, if $\alpha\in\CC$ is the root of some polynomial with coefficients in $\QQ$, we might wish to study $\QQ(\alpha)$, the smallest subfield of $\CC$ containing $\alpha$ and all of $\QQ$. Certainly, if we want to understand how "complicated" the number $\alpha$ is, it makes sense to consider how "complicated" the field $\QQ(\alpha)$ is as an extension of $\QQ$. If $F \subset E$ are fields, we will denote the extension by $E/F$ (this just means that $F$ is a subfield of $E$, and that we're considering $E$ relative to $F$, in particular, $E/F$ is not a quotient or anything too formal). Note that often we will consider $E$ to be an extension of $F$ even if $F\nsubseteq E$, as long as there is an obvious embedding of $F$ into $E$ (an embedding is a homomorphism with is injective). 

We will make a lot of use of the observation that if $E/F$ is an extension
of fields, then we may view $E$ as a vector space over $F$.

\begin{defn}
    Let $E/F$ be an extension of fields. We say that $E$ is a \emph{finite extension} of $F$ if $E$ is finite-dimensional as a vector space over $F$. In this case we denote the dimension by $[E:F]$. We say that $E$ is an \emph{infinite extension} of $F$ if $E$ is infinite-dimensional as a vector space over $F$, and we write $[E:F] = \infty$.
\end{defn}

\begin{example}
    $\{1,i\}$ is a basis for $\CC$ as a vector space over $\RR$. So $\CC$ is a finite extension of $\RR$ and $[\CC:\RR]=2$. 
\end{example}

\begin{example}
    It is widely known that $\sqrt{2}\notin\QQ$. Thus $1,\sqrt{2}$ are linearly independent over $\QQ$. On the other hand $(\sqrt{2})^2\in \QQ$ and then any polynomial in $\sqrt{2}$ with rational coefficients is just a $\QQ$-linear combinations of $1$ and $\sqrt{2}$. Since $$\frac{1}{a+b\sqrt{2}} = \frac{a}{a^2-2b^2} +\frac{-b}{a^2-2b^2}\sqrt{2},$$ every rational function of $\sqrt{2}$ can be written as a $\QQ$-linear combinations of $1$ and $\sqrt{2}$. It follows immediately that $\QQ(\sqrt{2})=\QQ[\sqrt{2}]$ and $[\QQ[\sqrt{2}]:\QQ]=2$.  
\end{example}

\begin{example}
    We can show $[\CC(x):\CC]=\infty$ by arguing $\{1,x,x^2,\cdots\}$ is a linear independent set.
\end{example}

\begin{example}
    To show $[\RR:\QQ]=\infty$, we make use of the unique factorization theorem of integers and argue that $\{\ln(p): p \mbox{ is a prime}\}$ is a linearly independent set.
\end{example}

\begin{thm}
    Let $K\subseteq F\subseteq E$ be fields. Then $E/K$ is a finite extensions if and only if both $F/K$ and $E/F$  are, and when this is the case, we have $$[E:K]=[E:F][F:K].$$
\end{thm}

\begin{proof}[Sketch of proof]
    If $\{a_i\}$ and $\{b_j\}$ are bases for $E/F$ and $F/K$ respectively, then $\{a_ib_j\}$ is a basis for $E/K$.
\end{proof}

\begin{example}
    Consider field extensions $\QQ\subset E= \QQ[\sqrt{2}]\subset F=\QQ[\sqrt{2},\sqrt{3}]$. We already know $[E:\QQ]=2$ and since $\sqrt{3}\notin E$ and it is a root of $x^2-3\in E[x]$ (so $1,\sqrt{3}$ form a $E$-basis of $F$), we also have $[F:E]=[E[\sqrt{3}]:E]=2$. And then $[\QQ[\sqrt{2},\sqrt{3}]:\QQ]=4$.
\end{example}

\begin{defn}
    Let $E/F$ be a field extension. An element $\aa\in E$ is \emph{algebraic} over $F$ if there is a non-zero polynomial $f(x) \in F[x]$ such that $f(\aa) = 0$. Otherwise we say that $\aa$ is \emph{transcendental} over $F$. The extension $E/F$ is \emph{algebraic} if every element of $E$ is algebraic over $F$, and is \emph{transcendental} otherwise.
\end{defn}

\begin{example}
    Both $\sqrt{2}$ and $i$ are algebraic over $\QQ$ as they are roots of $x^2-2$ and $x^2+1$.  But $\pi$ and $e$ are transcendental. As you can see, it's much easier to show that something is algebraic over a subfield than to show that it isn't (since to show that it is, one simply needs to exhibit a non-trivial
    polynomial relation). This shows that $\RR/\QQ$ is a transcendental extension, but some more work is required to show that $\QQ(\sqrt{2})$ is algebraic, namely, we need to make sure that the smallest
    field containing $\QQ$ and $\sqrt{2}$ doesn't somehow contain transcendental elements over $\QQ$.
\end{example}

\begin{thm}\label{minimal}
    Let $E/F$ be a finite extension of fields and $\aa$ a non-zero element in $E$. Then we have the following:
    \begin{enumerate}
        \item $\aa$ is algebraic;
        \item there is a unique non-zero monic irreducible polynomial $f(x) \in F[x]$ such that $f(\aa) = 0$, moreover, $\deg(f)\leq [E:F]$;
        \item if $\aa$ is a root of a polynomial $g(x)\in F[x]$, then $f\mid g$;
        \item if $I=(f)$, then $F[x]/I \cong F(\aa)$; indeed, there exists an isomorphism $\phi: F[x]/I \to f(\aa)$ with $\phi(x+I)=\alpha$ and $\phi(a+I)=a$ for all $a\in F$;
        \item if $\aa'$ is another root of $f$ (in $E$), then $F(\aa)\cong F(\aa')$
    \end{enumerate}
\end{thm}

\begin{proof}
    Suppose that $E/F$ is a finite extension and $\aa\in E$. Consider the elements $$1,\aa,\aa^2,\cdots,\aa^{[E:F]}\in E.$$
    Since there are $[E:F]+1$ elements, they must be linearly dependent over $F$. Hence we can find $c_i\in F$ such that $$c_o\cdot 1+c_1\aa+\cdots+c_{[E:F]}\aa^{[E:F]} =0.$$  In other words, $\aa$ is a root of the (non-zero) polynomial $$g(x)=\sum_{i=0}^{[E:F]}c_i x^i \in F[x].$$ And the degree of $g$ is at most $[E:F]$.

    Now consider the evaluation map $$\varphi: F[x]\to E, f(x)\mapsto f(\aa),$$ where one may consider it as the restriction of $e_\aa: E[x]\to E$. Then $\ker(\varphi)$ is non-empty since $g$ lies in it and then $\ker(\varphi)=(f(x))$ for some monic $f(x)\in F[x]$ since $F[x]$ is a PID. Any polynomial $g(x)\in F[x]$ with a root $\aa$ belongs to the kernel and hense is divisible by $f(x)$. Clearly, $\deg f$ is no bigger than $\deg g$ and then no bigger than $[E:F]$. Since $E$ is a field as well, $\im(\varphi)$ is a domain. So the kernel is a prime (hence maximal) ideal and therefore $f$ is irreducible and $\im(\varphi)$ is a field containing $\QQ$ and $\aa$ indeed. $\phi$ is the canonical isomorphism  induced by $\varphi$. 

    Finally , notice both $F(\aa), F(\aa')$ are isomorphic to $F[x]/I$.
\end{proof}

\begin{defn}
    The polynomial $f$ constructed in Theorem \ref{minimal} is called the \emph{minimal polynomial} of $\aa$ over $F$.
\end{defn}

\begin{remark}
    In some textbook, the minimal polynomial is denoted by $\mbox{irr}(\aa,F)$.
\end{remark}

In other words, in a finite extension, every element is the root of some polynomial over the smaller field. The next theorem is a partial converse to this, and we will use it often.

\begin{thm}\label{stem}
    Let $k$ be a field and $f[x]$ a monic irreducible polynomial in $k[x]$ of degree $d$. Let $K=k[x]/I$, where $I=(f)$, and $\beta = x+I\in K$. Then:
    \begin{enumerate}
        \item $K$ is a field and $k'=\{a+I: a\in k\}$ is a subfield of $K$ isomorphic to $k$,
        \item $\beta$ is a root of $f$ in $K$,
        \item if $g(x)\in k[x]$ and $\beta$ is a root of $g$ in $K$, then $f\mid g$ in $k[x]$,
        \item $f$ is the unique monic irreducible polynomial in $k[x]$ having $\beta$ as a root,
        \item $1,\beta,\beta^2,\cdots,\beta^{d-1}$ form a basis of $K$ as a vector space over $k$ and so $\dim_k(K)=d$.
    \end{enumerate}
\end{thm}

\begin{proof}
    With the knowledge form the warm-up part, we can prove this theorem easily.
    \begin{enumerate}
        \item  $I$ is a prime ideal hence maximal since $F[x]$ is a PID. So the quotient ring $K=k[x]/I$ is a field. Every field homomorphism is injective and so $k$ embeds into $K$ with its image $k'$.
        \item Let $f(x)= a_0+a_1x+\cdots+a_{d-1}x^{d-1}+x^d,$ where $a_i\in k$ for all $i$. In $K=k[x]/I$, we have \begin{align*}
            f\bb) &= (a_0+I)+(a_1+I)\bb+\cdots+(1+I)\bb^d \\
            &= (a_0+I)+(a_1+I)(x+I)+\cdots+(1+I)(x+I)^d \\
            &= (a_0+I)+(a_1x+I)+\cdots+(x^d +I)\\
            &= a_0+a_1x+\cdots+a_{d-1}x^{d-1}+x^d +I \\
            &= f(x)+I = 0+I.
        \end{align*} So $\beta$ is a root of $p$.
        \item If $f\nmid g$ in $k[x]$, then their $\gcd$ is 1 since $f$ is irreducible. Therefore, we can find polynomials $s,t$ in $k[x]$ such that $1=sf+gt$. Treating them as polynomials in $K[x]$ and evaluating at $\beta$, we get $1=0$, a contradiction.
        \item Let $g$ be a monic irreducible polynomial in $k[x]$having $\bb$ as a root. Then by part (3) we have $f\mid g$. Since $g$ is irreducible, we have $g=ch$ for some constant $c$. But both $f,g$ are monic, we have $c=1$ and $f=g$.
        \item Every element of $K$ has the form $g+I$, where $g(x)\in k[x]$. By the division algorithm, we have $g=qf+r$ with either $r=0$ or $\deg(r)<\deg(f)$. Then $g+I=r+I$ since $g-r=qf\in I$. By the calculation similar in part (2), it follows that $r+I = b_0+b_1\bb+\cdots+ b_{d-1} \bb^{d-1}$ if we express $r(x)= b_0+b_1 x+\cdots+b_{d-1}x^{d-1}.$   
        
        If $\{1,\beta,\beta^2,\cdots,\beta^{d-1}\}$ is not linearly independent, then we can find coefficients $c_i\in k$ not all zero such that $$c_0+c_1\bb+\cdots+c_{d-1}\bb^{d-1}=0.$$ Define $g(x)\in k[x]$ by $f(x)=\sum_{i=0}^{d-1}c_ix^i$. Then $g(\bb)=0$ and $\deg(g)\leq d-1 <\deg(f)=d$. By part (3) says $\deg(f)\leq \deg(g)$ since $f\mid g$. We reach a contradiction.
    \end{enumerate}
\end{proof}

\begin{remark}
    The pair $(K,\bb)$ is called the \emph{stem field} (of $f$) in Milner.
\end{remark}

\begin{example}
    The polynomial $x^2+1\in \RR[x]$ is irreducible so $K=\RR[x]/(x^2+1)$ is a finite extension of $\RR$ with degree 2. If $\bb$ is a root of $x^2+1$ in $K$, then $\bb^2=-1$. Moreover, every element of $K$ has a unique expression $a+b\bb,$ where $a,b\in \RR$.
\end{example}

\begin{example}
    Let $f(x)= x^4-10x^2+1\in \QQ[X]$. This is an irreducible polynomial: it has no rational roots (if $r/s$ in lowest form was one, then $r\mid 1$ and $r\mid 1$; the only possible rational root was $r/s =\pm 1/1=\pm 1$) and a direct factorization $f(x)=(x^2+ax+b)(x^2-ax+c)$ is also impossible. (One can show, however, $f$ is reducible in $\FF_p[x]$ for any prime $p$.) The roots of $f$ are $$\sqrt{2}+\sqrt{3}, -\sqrt{2}-\sqrt{3}, \sqrt{2}-\sqrt{3}, -\sqrt{2}+\sqrt{3}.$$
    Let $\bb$ be one of the roots. Consider the field extensions $\QQ\subset\QQ[\bb]\subset \QQ[\sqrt{2},\sqrt{3}]$. We already know from pervious example $$[\QQ[\sqrt{2},\sqrt{3}]:\QQ]=4=[\QQ[\sqrt{2},\sqrt{3}]:\QQ[\bb]][\QQ[\bb]:\QQ].$$ But $\bb$ is a root of irreducible polynomial of degree 4 and therefore $$[\QQ[\bb]:\QQ]=4.$$
    We see that $[\QQ[\sqrt{2},\sqrt{3}]:\QQ[\bb]]=1$ and then $$\QQ[\sqrt{2},\sqrt{3}]=\QQ[\bb].$$ 
    And hence all roots of $f$ lies in $\QQ[\bb]$.
\end{example}

\section{Automorphisms}
When one is first introduced to the complex numbers, it is usually as a superset of the reals. We're introduced to $\CC$ as a vector space over $\RR$ with basis $\{1,i\}$ which happens to also admit the structure of a field. One function which helps with the very basic study of $\CC$ from this perspective is the complex conjugation: $$\overline{x+yi}=x-yi$$ for $x,y\in \RR$. The important properties of this function are that it is an automorphism of $\CC$ and that it fixes real numbers (and only real numbers. We would like to identify functions of this form for arbitrary field extensions.

\begin{defn}
    Let $F$ be a field, and let $X \subset F$ be a subset. Then $\varphi: F \to F$ is an automorphism if it is a bijection and a homomorphism, namely, $\varphi(x+y)=\varphi(x)+\varphi(y)$ and $\varphi(xy)=\varphi(x)\varphi(y)$. We denote the group of automorphisms of $F$ by $\Aut(F)$. We say that $\varphi\in \Aut(F)$ fixes $X$ if $\varphi(x) = x$ for all $x \in X$, and we denote the set of automorphisms of $F$ fixing $X$ by $\Aut(F/X)$.
\end{defn}

It's worth noting that this definition of fixing a set is what might more
rightly be referred to as fixing $X$ pointwise. It is sometimes useful to consider functions which fix X setwise, meaning that $\varphi(x) \in X$ for all $x \in X$. Unless otherwise stated, "fix" means "fix pointwise".
Note that, in the lemma below, we make no special assumptions about
the nature of $X \subset F$.

\begin{prop}
    For any field $F$, and any set $X \subset F$, the set $\Aut(F/X)$ is a group under composition.
\end{prop}

\begin{proof}
    Just straightforward verifications.
\end{proof}

\begin{example}
    Consider $\Aut(\CC/\RR)$. Every element of $\CC$ can be written as $x+yi$ with $x,y\in\RR$. For any $\sigma\in \Aut(\CC/\RR)$, we must have $\sigma(x+yi)=x+y\sigma(i)$. Furthermore, we also have $$-1=\sigma(-1)=\sigma(i^2)=\sigma(i)^2,$$ and hence $\sigma(i)=\pm i$. So $\Aut(\CC/\RR)$ contains exactly two elements: the trivial one and the complex conjugation. It is clear that $\Aut(\CC/\RR)$ is group --- we need to check the complex conjugation is an automorphism of $\CC$ and twice the complex conjugation is just the identity map.
\end{example}

This example gives us a feeling about how $\Aut(E/F)$ will be for a field extension $E/F$. In general, if $E/F$ is a finite extension with $[E:F] = n$, then we can choose a basis $\aa_1,\cdots,\aa_n\in E$ for $E/F$. Any element of $E$ can be written uniquely in the form $$c_1\aa_1+\cdots+c_n\aa_n,$$ with $c_i\in F$. If $\sigma\in\Aut(E/F)$, then we have $$\sigma(c_1\aa_1+\cdots+c_n\aa_n)=c_1\sigma(\aa_1)+\cdots+c_n\sigma(\aa_n).$$  In other words, the automorphism $\sigma$ is entirely defined by the $n$ values $\sigma(\aa_1),\cdots,\sigma(\aa_n)$. Moreover, if $f_i(x) \in F[x]$ is the minimal polynomial for $\aa_i$, then $$f_i(\sigma(\aa_i)) = \sigma(f_i(\aa_i)) = \sigma(0) = 0.$$
So $\sigma(\aa_i)$ is one of the (finitely many) roots of $f_i$ in $E$. So there are only finitely many possible values for $\sigma(\aa_i)$, for each $i$. We won't count how many automorphisms the can be (this will become easier later), but we've just made the following useful observation:

\begin{thm}
    Let $E/F$ be a finite extension of fields. Then $\Aut(E/F)$ is a finite group. Moreover, if we have $E = F(\aa)$ for some $\aa \in E$, then $\Aut(E/F)$ naturally embeds into the group of permutations of the roots of the minimal polynomial of $\aa$ over $F$.
\end{thm}

Note that $E/F$ does not need to be a finite extension for us to define $\Aut(E/F)$ (indeed, $F$ need not even be a field).

Unfortunately, there are interesting extensions $E/F$ for which the group $\Aut(E/F)$ is trivial and hence not interesting. Here is an example:

\begin{example}
    Let $\aa$ be the real cube root of $2$, and let $E = \QQ(\aa)$. Then $[E:\QQ] = 3$ (since the minimal polynomial of $\aa$, which is $f(x) = x^3 - 2$, is irreducible over $\QQ$). Now suppose that $\sigma \in \Aut(E/\QQ)$. We've seen that $\sigma$ is entirely determined by $\sigma(\aa)$. But $E \subset \RR$, and $\sigma(\aa)$ has to satisfy $$\sigma(\aa)^3 = \sigma(\aa^3) = 2.$$
    In particular, $\sigma(\aa)$ is a real cube root of $2$, and so the only possibility is $\sigma(\aa) = \aa$. In other words, the only element of $\Aut(E/\QQ)$ is the trivial element $\sigma(x) = x$ for all $x \in E$.
\end{example}

\begin{example}
    We can show $\Aut(\RR/\QQ)$ is also trivial. Let $\sigma\in \Aut(\RR/\QQ)$. From the observation that $$\sigma(a^2)=\sigma(a)^2 > 0,$$ we see $\sigma$ must take positive to positive and hence order-preserving. And then it must be continuous (by more detailed arguments) but any continuous map on $\RR$ which is the identity on $\QQ$ is the identity map (again you may fill the details if you want).
\end{example}

\noindent Our next example says something about finite fields. We do a quick catch-up here.

Recall there is a natural ring homomorphism $\pi:\ZZ\to F$ determined by $\pi(1)=1_F$ for any field $F$. The image is a integral domain, hence $\ker(\pi)$ is trivial or is $(p)$, where $p$ is prime. We say the field is of characteristic 0 in former case and $p$ latter. And the image is again a field --- we call it the \emph{prime field} of $F$.

We denote the finite field of order exactly $p$, where $p$ is a prime, by $\FF_p=\{0,1,\cdots,p-1\}$. If $F$ be a finite field with $q$ elements and suppose that $F \subset K$ where $K$ is also a finite field. Then $K$ (isomorphic to $F^n$ as a vector space) has $q^n$ elements where $n = [K:F]$ from the knowledge on finite field extensions. Hence a finite field is isomorphic to $\FF_{p^n}$ where $p$ is its characteristic and $n\in \NN$.

Since $\FF_{p^n}^\times$ is cyclic of order $p^n-1$, we have $a^{p^n} = a$ for all $a\in \FF_{p^n}$. The polynomial $x^{p^n}-x$ has at most $\deg =  p^n$ roots and we conclude 
\begin{equation}\label{ff2}
    x^{p^n}-x = \prod_{a\in \FF_{p^n}} (x-a) \in \FF_{p^n}[x],
\end{equation}
in other words, $\FF_{p^n}$ is the splitting field of $x^{p^n}-x$. However, we do not need this terminology yet at this stage.


On the other hand, by Theorem \ref{stem}, we have the following proposition:

\begin{prop}\label{ff3}
    For a prime $p$ and a monic irreducible $f(x)$ in $\FF_p[x]$ of degree $n$, the ring $\FF_p[x]/(f(x))$ is a field of order $p^n$.
\end{prop}

\begin{example}
    Two fields of order 8 are $\FF_2[x]/(x^3+x+1)$ and $\FF_2[x]/(x^3+x^2+1)$. Two fields of order 9 are $\FF_3[x]/(x^2+1)$ and $\FF_3[x]/(x^2+x+2)$. 
\end{example}

\begin{thm}\label{ff1}
    Let $f(x)\in \FF_p[X]$ be irreducible and of degree $n$. Then $f$ has no repeated roots and $f$ divides $X^{p^n}-X$. (Hence or otherwise, $X^{p^n} -X$ has a factorization $X^{p^n}-X =  \prod_{d\mid n}\prod_{f_d} f_d$, where $f_d$ runs over all irreducible polynomials of degree $d$.)
\end{thm}

\begin{proof}[Sketch of proof]
    Combining Equation \ref{ff2} and Proposition \ref{ff3}, we know $f$ and $X^{p^n}-X$ share a root. So $f$ divides $X^{p^n}-X$ by Theorem \ref{minimal}. Moreover, $X^{p^n}-X$ has no repeated roots and so is $f$. 
\end{proof}

\begin{thm}
    Every finite field $F$ is isomorphic to $\FF_p[x]/(f)$ for some prime $p$ and some irreducible polynomial $f(x)\in \FF_p[x]$.
\end{thm}

\begin{proof}
    Combining what we discussed just now, we have $|F|=p^n$ and an embedding $\FF_p \hookrightarrow F$.
    $F^\times$ is cyclic, say, it is generated by $\aa$. Then $f$ is the minimal polynomial of $\aa$ over $\FF_p$.
\end{proof}

\begin{thm}
    Two finite fields $E,F$ of the same size are isomorphic.
\end{thm}

\begin{proof}
    Since $|E|=p^m$ and $|F|=q^n$, where $p,q$ are primes and $m,n$ positive integers, we must have $p=q$ and $m=n$. They are both isomorphic to $\FF_{p^n}$.
\end{proof}

\begin{example}
    Let $p$ be a prime, and consider the extension $\FF_{p^n}/\FF_p$. We define a function $\sigma: \FF_{p^n} \to \FF_{p^n}$ by $\sigma(x) = x^p$. By the binomial theorem, and the fact that p divides the binomial coefficient $\binom{p}{j}$for any $1 \leq j \leq p-1$, we have $$\sigma(x + y) = (x + y)^p = x^p + y^p + p\cdot \mbox{(something)} = \sigma(x) + \sigma(y).$$
    And of course $\sigma(xy) = \sigma(x)\sigma(y)$. So $\sigma$ is a homomorphism. We wish to show that $\sigma$ is an automorphism of $\FF_{p^n}$. Since $\FF_{p^n}$ is finite, we simply need to show that $\sigma$ is either surjective or injective. We'll show that it's injective. To see this, suppose to the contrary that there's some non-zero $x \in \FF_{p^n}$ with $\sigma(x) = 0$. Since the group of non-zero elements $\FF_{p^n}$ is cyclic, say, generated by $\gamma$. If $x =\gamma^j$, then $$x^{p^n} = (\gamma^j)^{p^n} = (\gamma^{p^n})^j = \gamma^j = x.$$
    On the other hand,  $$x^{p^n}= \sigma^{(n)}(x) = \sigma^{n-1}(\sigma(x)) = \sigma^{(n-1)}(0) = 0,$$ where $\sigma^{(n)}$ means compose $\sigma$ with itself $n$ times. We reach a contradiction.
    Also, note that $\sigma$ fixes $\FF_p$, so really $\sigma \in \Aut(\FF_{p^n}/\FF_p)$. It's possible to show that $\sigma$ generates this group (laster).

    $\sigma$ is usually referred to as the \emph{Frobenius homomorphism}.
\end{example}

\chapter*{Week 2}
\setcounter{chapter}{2}

\section{Algebraic closure}

Although we can get by with constructing fields like $\QQ(i)$ as quotients:
$$\QQ(i) \cong \QQ[x]/(x^2 + 1),$$
it is often useful to think of all of the possible algebraic extensions as being subfields of one large (infinite) extension.

\begin{defn}
    A field $F$ is said to be \emph{algebraically closed} if every non-constant polynomial $f(x) \in F[x]$ has a root in $F$ or, equivalently, if every polynomial in $F[x]$ factors as a product of linear terms. An \emph{algebraic closure} of a field $k$ is an algebraic extension $\overline{k}$ of $k$ that is algebraically closed.
\end{defn}

The Fundamental Theorem of Algebra says that $\CC$ is algebraically closed; moreover, $\CC$ is an algebraic closure of $\RR$. One may have seen a proof of Fundamental Theorem using fundamental group, but the simplest proof of the Fundamental Theorem is probably that using Liouville's Theorem in complex variables: every bounded entire function is constant.

\begin{thm}\label{ac}
    Let $F$ be a field. Then \begin{enumerate}
        \item there exists a field $\overline{F}\supset F$ which is algebraically closed and algebraic over $F$,
        \item this field $\overline{F}$ is unique up to isomorphism.
    \end{enumerate}
\end{thm}

As you might guess, Zorn's lemma is inevitable in such kind proof. The problem is how to apply it.

One can imagine looking at the "set" of all algebraic extensions of $F$. This collection of fields is partially ordered under inclusion, and linearly ordered chains in the collection have least upper bounds (their union). If this were a set of fields, then it would follow form Zorn's Lemma that there is a maximal element, call it $\overline{F}$. This would be an algebraic extension of $F$: it would have no proper algebraic extension, and hence it would have to be algebraically closed. This is the algebraic closure of $F$. The problem with this argument is that the collection of algebraic extensions of $F$ is too large to be a set, and hence one cannot apply Zorn's Lemma.

\begin{lemma}\label{ac4}
    Let $E/F$ be a field extension. If $\aa_1,\cdots,\aa_n \in E$ are algebraic over $F$, then $F(\aa_1,\cdots,\aa_n)/F$  is a finite extension.
\end{lemma}

\begin{proof}
    We can do induction on $n$. It is straightforward.
\end{proof}

\begin{lemma}\label{ac3}
    If $E$ is an algebraic extension of $F$ and $F$ is an algebraic extension of $K$ then $E$ is an algebraic extension of $K$.
\end{lemma}

\begin{proof}
    Take any non-zero element $\aa\in E$. Since $E/F$ is algebraic, there is a polynomial $$f(x)=c_0+c_1x+\cdots+c_n x^n\in F[x]$$ that has $\aa$ as a root. By viewing $f(x)$ as an element in $K(c_0,\cdots,c_n)[x]$, we see that $$K(\aa,c_0,\cdots,c_n)/K(c_0,\cdots,c_n)$$ is a finite extension. But all the $c_i, i=1,\cdots, n$ are in $F$ hence algebraic over $K$,$$K(c_0,\cdots,c_n)/K$$ is also a finite extension by Lemma \ref{ac4}. And so $K(\aa,c_0,\cdots,c_n)/K$ is also a finite extension hence $\aa$ is algebraic over $K$. 
\end{proof}

\begin{lemma}\label{ac2}
    Let $k$ be a field, and let $k[T]$ be the polynomial ring in a set $T$ of indeterminates (indeed, $|T|$ can be infinite). If $t_1, \cdots, t_n \in T$ are distinct, where $n \geq 2$, and $f_i(t_i) \in k[t_i] \subset k[T]$ are non-constant polynomials, then the ideal $I = (f_1(t_1),\cdots, f_n(t_n))$ in $k[T]$ is proper.
\end{lemma}

\begin{proof}
    If $I$ is not a proper ideal in $k[T]$, then there exist $h_i(T) \in k[T]$ with $$ 1 = h_1(T)f_1(t_1) + \cdots + h_n(T)f_n(t_n).$$

    Consider the extension field $k(\aa_1,\cdots,\aa_n)$, where $\aa_i$ is a root of $f_i(t_i)$ for $i = 1,\cdots , n$ (the $f_i$ are not constant. Denote the variables involved in the $h_i(T)$ other than $t_1,\cdots , t_n$, if any, by $t_{n+1}, \cdots , t_m$. Evaluating when $t_i = \aa_i$ if $i \leq n$ and $t_i = 0$ if $i \geq n + 1$. The right side is 0, and we have the contradiction $1 = 0$.
\end{proof}

\begin{proof}[Proof of Theorem \ref{ac} part (1)]
    Let $T$ be a set in one-to-one correspondence with the family of non-constant monic polynomials in $k[x]$. (So $T$ is a very very large set in general.) Let $R = k[T]$ be the big polynomial ring, and let $I$ be the ideal in $R$ generated by all elements of the form $f(t_f)$, where $t_f\in T$; that is, if $$f(x) = x_n + a_{n-1}x^{n-1} + \cdots + a_0,$$ where $a_i \in k$, then $$f(t_f) = (t_f)^n + a_{n-1}(t_f)^{n-1} + \cdots + a_0.$$

    We claim that the ideal $I$ is proper; if not, $1 \in I$, and there are distinct and finite! $t_1, \cdots , t_n \in T$ and polynomials $h_1(T), \cdots , h_n(T) \in k[T]$ with $$ 1 = h_1(T)f_1(t_1) + \cdots + h_n(T)f_n(t_n),$$ contradicting Lemma \ref{ac2}. Therefore, there is a maximal ideal $\mm$ in $R$ containing $I$. Define $K = R/\mm$. The proof is now completed in a series of steps.
    \begin{enumerate}
        \item $K/k$ is an extension field.
        
        We know that $K = R/\mm$ is a field because $\mm$ is a maximal ideal. Let $i : k \to  k[T]$ be the ring map taking $a \in k$ to the constant polynomial a, and let $\theta$ be the composite $k\to 
        k[T] = R\to R/\mm  = K$. Now $\theta$ is injective, (recall ring homomorphisms between fields are injective). We identify $k$ with $ \im (\theta)  \subset K$.

        \item  Every non-constant $f(x) \in k[x]$ has a root in $K[x]$, hence factors into linear terms in $K[x]$.
        
        By definition, for each $t_f \in T$, we have $f(t_f ) \in I \subset \mm$, and so the coset $t_f +\mm \in R/\mm  = K$ is a root of $f(x)$. By induction on the degree of $f$

        \item The extension $K/k$ is algebraic.
        
        It suffices to show that each $t_f+\mm$ is algebraic over $k$ (for $K = k(\mbox{all } t_f + \mm))$; but this is obvious, for $t_f$ is a root of $f(x) \in k[x]$.

        \item $K$ is algebraically close.
        
        Let $\pi(x)\in K[x]$ be a monic irreducible polynomial and $\aa$ one of the roots in some extension $K'$. Since both $K(\aa)/K$ and $K/k$ are algebraic, $K(\aa)/k$ is also algebraic by Lemma \ref{ac3}. Hence $\aa$ is a root of some polynomial in $k[x]$. But polynomials in $k[x]$ factors into linear terms in $K[X]$, we must have $\aa$ lies in $K$ as well. 
    \end{enumerate}
\end{proof}

\begin{cor}
    If $k$ is a countable field, then it has a countable algebraic closure. In particular, the algebraic closure of the prime fields $\QQ$ and $\FF_p$ are countable 
\end{cor}

\begin{proof}
    $k$ is countable $\Longrightarrow$ $T$ is countable $\Longrightarrow$ $k[T]$ is countable $\Longrightarrow$ $K=k[T]/\mm$ is countable.
\end{proof}

To prove the (non)uniqueness of a algebraic closure needs more work.

\begin{defn}
    If $F/k$ and $K/k$ are extension fields, then a $k$-map is a ring homomorphism $\varphi : F \to K$ that fixes $k$ pointwise.
\end{defn}

\begin{lemma}\label{ac5}
    If $K/k$ is an algebraic extension, then every $k$-map $\varphi: K \to K$ is an automorphism of $K$.
\end{lemma}

\begin{proof}
   Since $\varphi$ is a ring homomorphism between fields, $\varphi$ is injective. To see that $\varphi$ is surjective, let $a \in K$. Since $K/k$ is algebraic, there is an irreducible polynomial $p(x) \in k[x]$ having $a$ as a root. The $k$-map $\varphi$ permutes the set $A$ of all those roots of $p(x)$ that lie in $K$. Therefore, $a \in \varphi(A) \subset  \im(\varphi)$, since $A$ is finite.
\end{proof}

\begin{lemma}\label{extensionlemma}
    Let $k$ be a field and let $K/k$ be an algebraic closure. If $F/k$ is an algebraic extension, then there is an injective $k$-map $\psi: F \to K$. In particular, every $k$-map $F\to K$ extends to a $k$-map $K\to K$, which is an isomorphism indeed.
\end{lemma}

If $F$ is countably generated over $k$, namely, $F=k[\aa_1,...,\aa_n]$, then we can extend the inclusion $\iota: k\to K$ to $k[\aa_1]$, then to $k[\aa_1,\aa_2]$, and so on. 

\begin{proof}
    If $E$ is an intermediate field, $k \subset  E \subset  F$, let us call an ordered pair $(E, f)$ an approximation if $f : E \to K$ is a $k$-map. Define $X = \{\mbox{approximations } (E, f) : k \subset  E \subset  F\}$. Note that $X \not= \emptyset$ because $(k, i) \in X$. Partially order $X$ by $$(E, f) \preceq (E', f') \mbox{ if } E \subset E' \mbox{ and } f'|E = f.$$
    That the restriction $f'|E$ is $f$ means that $f'$ extends $f$; that is, the two functions agree whenever possible: $f'(u) = f(u)$ for all $u \in E$.

    It is easy to see that an upper bound of a chain $$S = \{(E_j, f_j) : j\ \in J\}$$ is given by $(\cup E_j,\cup f_j)$. That $E_j$ is an intermediate field is, by now, a routine argument. We can take the union of the graphs of the $f_j$, but here is a more down-to-earth description of $\varphi =\cup f_j$: if $u \in \cup E_j$, then $u \in E_{j_0}$ for some $j_0$, and $\varphi(u)= fj_0 (u)$. Note that $\varphi$ is well-defined: if $u \in E_{j_1}$, we may assume, for notation, that $E_{j_0} \subset  E_{j_1}$, and then $f_{j_1} (u) = f_{j_0} (u)$ because $f_{j_1}$ extends $f_{j_0}$. Observe that $\varphi$ is a $k$-map because all the $f_j$ are.
    
    By Zorn's Lemma, there exists a maximal element $(E_0, f_0)$ in $X$. We claim that $E_0 = F$, and this will complete the proof (take $\psi  = f_0)$. If $E_0 \subsetneq F$, then there is $a \in F$ with $a \notin E_0$. Since $F/k$ is algebraic, we have $F/E_0$ algebraic, and there is an irreducible $p(x) \in E_0[x]$ having a as a root. Since $K/k$ is algebraic and $K$ is algebraically closed, we have a factorization in $K[x]$: $$f_0^*(p(x)) = \prod_{i=1}^n(x-b_i),$$ where $f_0^*: E_0[x] \to K[x]$ is the map $$f^*_0 : e_0 + \cdots + e^nx^n \mapsto f_0(e_0) +\cdots + f_0(e^n)x^n.$$ If all the $b_i$ lie in $f_0(E_0) \subset K$, then $f^{-1}_0 (b_i) \in E_0 \subset  F$ for some $i$, and there is a factorization of $p(x)$ in $F[x]$, namely, $p(x) =\prod_{i=1}^n[x - f^{-1}_0 (b_i)]$. But $a \notin E_0$ implies $a \not= f^{-1}_0 (b_i)$ for any $i$. Thus, $x - a$ is another factor of $p(x)$ in $F[x]$, contrary to unique factorization. We conclude that there is some $b_i \notin f_0(E_0)$. By Theorem \ref{minimal}, we may define $f_1 : E_0(a) \to K$ by $$c_0 + c_1a + c_2a^2 +\cdots \mapsto f_0(c_0) + f_0(c_1)b_i + f_0(c_2)b^2_i+\cdots .$$ A straightforward check shows that $f_1$ is a (well-defined) $k$-map extending $f_0$. Hence, $(E_0, f_0) \prec (E_0(a), f_1)$, contradicting the maximality of $(E_0, f_0)$. 

    For the "in particular" part, note $K/F$ is also algebraic and $K$ is a algebraic closure of $F$ as well. By previous lemma, such a $k$-map is an automorphism.
\end{proof}

\begin{proof}[Proof of Theorem \ref{ac} part (2)]
    Let $K$ and $L$ be two algebraic closures of a field $k$. By Lemma \ref{extensionlemma}, there are injective $k$-maps $\psi : K \to L$ and $\theta : L \to K$. By Lemma \ref{ac5}, both composites $\theta \psi : K \to K$ and $\psi \theta : L \to L$ are automorphisms. It follows that $\psi$ (and $\theta$) is a $k$-isomorphism.
\end{proof}

It is now permissible to speak of \emph{the} algebraic closure of a field.

\section{Splitting fields}

Most of the cases, the algebraic closure $\overline{k}$ of $k$ is too big. We may want to a smaller but still big enough field extension $E/k$.

\begin{defn}
    A \emph{splitting field} of a non-constant polynomial $f(x)\in F[x]$ is a field extension $E/F$ such that $f(x)$ factors into linear terms in $E[x]$, namely, $$f(x)=c(x-\aa_1)\cdots(x-\aa_n),$$
    and such that $E=F(\aa_1,\cdots,\aa_n)$. 
\end{defn}

This is, the splitting field is the smallest field (unique up to isomorphism) extension containing all the roots of $f(x)$. If $f$ is irreducible and separable, the all linear terms above are distinct. This may not be true if $f$ is inseparable.

\begin{example}
    Let $f(x)=x^3-2\in\QQ[x]$. $\QQ(\sqrt[3]{2})$ is not the splitting field of $f$, since it does not contain all the root. The splitting field is $E=\QQ(\sqrt[3]{2},\omega\sqrt[3]{2},\omega^2\sqrt[3]{2},)$ where $\omega$ is a third of unity, in other words, $1,\omega,\omega^2$ are roots of $x^3-1$. Note that we also have $E=\QQ(\omega,\sqrt[3]{2})$. The symbol $\sqrt[3]{2}$ is ambiguous, but we can either take the real cube root, or simply note that any choice defines the same field (only now that we've thrown in $\omega$).
\end{example}

\begin{thm}[Kronecker]
    If $k$ is a field and $f(x)\in k[x]$ a non-constant polynomial, there exists an extension field $K/k$ with $f$ a product of linear polynomials in $K[x]$.
\end{thm}

\begin{proof}
    The proof is induction on $\deg(f)$. If $\deg(f)=1$, then $f$ is linear and we can choose $K=k$. If $\deg(f) > 1$, write $f=pg$, where $p,q\in k[x]$ and $p$ irreducible. Now Theorem \ref{stem} provides a field $F$ containing $k$ and a root $z$ of $p$. Hence, in $F[x]$, there is $h(x)$ with $p(x)=(x-z)h(x)$ and so $f(x)=(x-z)h(x)g(x)$. By induction, there is a field $K$ containing $F$ so that $hg$, and hence $f$, is a product of linear factors in $K[x]$. 
\end{proof}

\begin{cor}
    If $k$ is a field and $f(x)\in k[x]$ a non-constant polynomial, then a splitting field of $f$ over $k$ exists.
\end{cor}

Let's see an example but with some definitions first.

\begin{defn}
    If $n \geq  1$ is a positive integer, then an $n$-th \emph{root of unity} in a field $k$ is an element $\zeta \in  k$ with $\zeta^n = 1$.
\end{defn}


The (complex) numbers $e^{\frac{2\pi i k}{n}} = \cos(2\pi k/n) + i \sin(2\pi k/n)$ for some $k$ with $0 \leq  k \leq  n- 1$ are \emph{all} the complex $n$-th roots of unity (in $\CC$). Just as there are two square roots of a number $a$, namely, $\sqrt{a}$ and $-\sqrt{a}$, there are $n$ different $n$-th roots of $a$, namely, $e^{\frac{2\pi i k}{n}}\sqrt[n]{a}$  for $k = 0, 1,\cdots , n - 1$. (We can assume $\sqrt[n]{a}$ to be the real root if one wants.) Every $n$-th root of unity is, of course, a root of the polynomial $x^n- 1$. Therefore, \begin{equation}\label{cycoeq1}
    x^n - 1 = \prod_{\zeta^n=1}(x -\zeta).
\end{equation}


\begin{defn}
    If $\zeta$ is an $n$-th root of unity and $n$ is the smallest positive integer for which $\zeta^n = 1$, we say that $\zeta$ is a \emph{primitive} $n$-th root of unity.
\end{defn}

\begin{example}
    $i$ is an 8-th root of unity (for $i^8 = 1$), but not a primitive 8-th root of unity; $i$ is a primitive 4-th root of unity.
\end{example}

\begin{example}
    Let $f(x)=x^n-1 \in k[x]$ for some field $k$ and $E/k$ a splitting field. The set of all $n$-th roots of unity is a cyclic group and a primitive $n$-th root $\omega$ generates it. It follows that $E=k(\omega)$ is a splitting field of $f$. 
\end{example}

We say "a" splitting field instead of "the" splitting field because it is not obvious whether any two splitting fields of $f$ over $k$ are isomorphic (they are). Analysis of this technical point will not only prove uniqueness of splitting fields, it will enable us to prove that any two finite fields with the same number of elements are isomorphic.

\begin{thm}\label{kummer1}
    Let $p$ be a prime and $k$ a field. If $f(x)=x^p-c\in k[x]$ and $\aa$ is a $p$-th root of $c$ (in some splitting field), then either $f$ is irreducible in $k[x]$ or $c$ has a $p$-th root in $k$. In either case, if $k$ contains the $p$-th root of unity, then $k(\aa)$ is a splitting field of $f$.
\end{thm}

\begin{proof}
    By Kronecker's Theorem, there exists an extension field $K/k$ that contains all the roots of $f$, namely, $K$ contains all the $p$-th root of $c$. If $\aa^p=c$, then every such root has the form $\zeta\aa$, where $\zeta$ is a $p$-th root of unity.

    If $f$ is not irreducible in $k[x]$, then there is a factorization $f = gh$ in $k[x]$, where $g(x), h(x)$ are non-constant polynomials with $d = \deg(g) < \deg(f) = p$. Now the constant term $b$ of $g$ is, up to sign, the product of some of the roots of $f$: $$\pm b = \aa^d\zeta',$$ where $\zeta'$, which is a product of $p$-th roots of unity, is itself a $p$-th root of unity. It follows that $$(\pm b)^p = (\aa^d\zeta')^p = \aa^{dp} = c^d.$$
    But $p$ being prime and $d < p$ force $\gcd(d, p) = 1$; hence, there are integers $s$ and $t$ with $1 = sd + tp$. Therefore, $$c = c^{sd+tp} = c^{sd}c^{tp} = (\pm b)^{ps}c^{tp} = [(\pm b)^sc^t]^p,$$ and $c$ has a $p$-th root in $k$.

    We now assume that $k$ contains the set $\Omega$ of all the $p$-th roots of unity. If $\aa  \in  K$ is a $p$-th root of $c$, then $f(x) =\prod_{\omega\in \Omega}(x- \omega\aa )$ shows that $f$ splits over $K$ and that $k(\aa)$ is a splitting field of $f$ over $k$. 
\end{proof}

Observe that if $\varphi : F \to F'$ is a homomorphism, then
$$\varphi_*:F[x]\to F'[x], a_0+\cdots+a_nx^n\mapsto \varphi(a_0)+\cdots+\varphi(a_n)x^n $$  
is a ring homomorphism. If $\varphi : F \to F'$ is an isomorphism, then $\varphi_*$ is also an isomorphism, and for any $f \in F[x]$, it sends the ideal $\langle f\rangle$ to the ideal $\langle \varphi_*(f)\rangle$.

\begin{lemma}\label{splitlemma1}
    Let $\varphi : F \to F'$ be an isomorphism between two fields, and $f \in F[x]$ be irreducible in $F[x]$. If $\aa$ is a root of $f$ in some extension of $F$, and $\bb$ is a root of $\varphi_*(f)$ in some extension of $F'$, then there is an isomorphism $\phi : F(\aa) \to F'(\bb)$ such that $\phi(\aa) = \bb$, and $\phi|_F= \varphi$.
\end{lemma}

\begin{proof}
    Because $f$ is irreducible in $F[x]$, $\varphi(f)$ is irreducible in $F'[x]$. Because $\varphi_*$ the ideal $\langle f\rangle$ to the ideal $\langle \varphi_*(f)\rangle$, the map $F[x] \to F'[x]/\varphi_*(f)\rangle$ will have kernel equal to $\langle f\rangle$, so we have an isomorphism $$F(\aa)\cong F[x]/\langle f\rangle \cong F'[x]/\langle \varphi_*(f)\rangle \cong F'(\bb).$$
\end{proof}

\begin{thm}
    Let $\varphi: F \to F'$ be an isomorphism, and $f$ be a non-constant polynomial in $F[x]$. If $E$ is a splitting field for $f$ in $F$, and $E'$ is a splitting field for $\varphi_*(f)$ in $F'$, then there is an isomorphism $\phi : E \to E'$ such that $\phi|_F= \varphi$.
\end{thm}

\begin{proof}
    Let $p$ be an irreducible factor of $f$ of degree $\geq$ 2. Let $\aa_1 \in E$ be a root of $p$ and $\bb_1 \in E'$ be root of $\varphi_*(p)$. By the previous lemma, there is an isomorphism $F(\aa_1) \to F'(\bb_1)$ that restricts to $\varphi$ on $F$. If we repeat this process (until $f$ no longer has any irreducible factors of degree $\geq 2$), then we have a isomorphism $F(\aa_1,\cdots, \aa_k) \cong F'(\bb_1,\cdots, \bb_k)$ that restricts to $\varphi$ on $F$. Because $f$ splits in $F(\aa_1,\cdots, \aa_k)$, and $\varphi(f)$ splits in $F'(\bb_1,\cdots, \bb_k)$, it follows that $E = F(\aa_1,\cdots, \aa_k)$ and $E' = F'(\bb_1,\cdots, \bb_k)$, which completes the proof.
\end{proof}

The following theorem, which is perhaps surprising, says that the splitting field of a polynomial contains a splitting field for the minimal polynomial of any element of that field.

\begin{thm}\label{normal}
    Let $E/F$ be a finite extension of fields. The following are equivalent:
    \begin{enumerate}
        \item $E$ is the splitting field of some polynomial over $F$;
        \item every irreducible polynomial $g(x)\in F[x]$ with a root in $E$ factors completely over $E$;
        \item any embedding $\sigma: E\to \overline{E}$ fixing $F$ is an automorphism of $E$, namely, $\im(\sigma)=E$.
    \end{enumerate}
\end{thm}

\begin{defn}
    A finite extension $E/F$ is \emph{normal} if it satisfies any of the equivalent conditions in Theorem \ref{normal}.
\end{defn}

\begin{remark}
    In Theorem \ref{normal}, you can replace a \emph{finite extension} by an \emph{algebraic extension}, and then what we need is a possibly infinite set of polynomials.
\end{remark}

% \begin{lemma}\label{nlam}
%     Let $E/F$ be a finite extension. Suppose that $F'\subset E$ is a field and that there is a embedding $\psi: F'\to\overline{E}$ fixing $F$, Then $\psi$ can be extended to an embedding $\phi: E\to\overline{E}$ in the sense that $\phi|_{F'}=\psi$.
% \end{lemma}

% \begin{proof}
%     We proceed by induction on $[E:F']$. If $[E:F']=1$, then $E=F'$ and we are done. 
    
%     Suppose that $[E:F]>1$ and that the statement is true for fields $F\subset F''\subset E$ with $[E:F'']<[E:F']$. Since $E\not=F'$, there is an element $\bb\in E\setminus F'$ which, necessarily, is algebraic over $F'$. Now we have been given an embedding $\psi: F'\to\overline{F}$ which fixes $F$. For a polynomial $g(x)\in F'[x]$, we let $g^\psi(x)\in \overline{F}$ be the polynomial obtained by applying $\psi$ to each coefficient. Then clearly $g(x) \to g^\psi(x)$ is an isomorphism from $F'[x]$ to $K[x]$, where $K \subset F$ is the image of $F'$ under $\psi$. Now, if $g^\psi(\bb') = 0$, it follows easily that
%     $$F'(\bb)\cong F'[x]/(g)\cong K[x]/(g^\psi) \cong K(\bb').$$
%     This gives an embedding of  $\psi': F'(\bb) \to F$ which extends $\psi$. Since $[E:F'(\bb)] < [E : F']$ (because $[F'(\bb) : F0] > 1$), we may apply the induction hypothesis to  $\psi': F'(\bb) \to E$, and conclude that it extends to an embedding $E \to \overline{F}$.
% \end{proof}

\begin{proof}[Proof of Theorem \ref{normal}] We show that $$(3)\Longrightarrow(2)\Longrightarrow (1)\Longrightarrow (3).$$
    
    (2) $\Longrightarrow$ (1): For a primitive extension $E=F(\alpha)$, $E$ is the splitting field of the minimal polynomial of $\alpha$ over $F$. If $E=F(\aa_1,\cdots,\aa_n)$, the $E$ is the splitting field of the products of the minimal polynomials of $\aa_i$ over $F$ for $i=1,\cdots,n$.

    (3) $\Longrightarrow$ (2): Let $g(x)\in F[x]$ be an irreducibility polynomial with a root $\bb\in E$. For each root $\beta'\in \overline{E}$, we obtain an embedding: $F(\beta) \to \overline{E}$ with $\phi(\bb)=\bb'$. By Lemma \ref{extensionlemma}, we can extend to an embedding $\sigma: E\to \overline{E}$ (fixing $F$). The image of $\sigma$ is $E$ by assumption. In other words, $\bb'=\phi(\bb)$ is in $E$. As $\bb'$ runs over all possible roots, $E$ contains all the roots of $g$ and so $g$ factors completely in $E[x]$.

    (1) $\Longrightarrow$ (3): Let $E$ be the splitting field of $f(x)\in F[x]$ and $\sigma: E\to\overline{E}$ an embedding fixing $F$. We know that $\sigma$ permutes the roots of $f(x)$ in $\overline{E}$, so the image of $\sigma$ contains all the roots of $f$. But $E$ is exactly generated by the roots of $f$, so the $\im(\sigma)$ is as well. It follows that $\sigma(E)=E$.
\end{proof}

\begin{defn}
    A \emph{normal closure} of an algebraic field extension $L/K$ is a field extension field $L'$ of $L$ such that $L'/L$ is algebraic and $L'/K$ is normal with the property that there is no proper subfield of $L'$ satisfying these conditions.
\end{defn}

\begin{thm}\label{normalclosure}
    Let $L/K$ be an algebraic field extension. Then we have:
    \begin{enumerate}
        \item $L/K$ admits a normal closure $L'/K$, where $L'$ is unique up to isomorphism over $L$.
        \item $L'/K$ is finite if $L/K$ is finite
        \item If $M/L$ is an algebraic field extension such that $M/K$ is normal, we can choose $L'$ as an intermediate field of $M/L$. In this case, $L'$ is unique. More precisely, if $(\sigma_i)_{i\in I}$ is the family of all $K$-homomorphisms from $L$ to $M$, then $L' = K(\sigma_i(L): i \in I)$. We call $L'$ the normal closure of $L$ in $M$.
    \end{enumerate}
\end{thm}

\begin{proof} 
    Note the only difference when we pass from a finite extension to an algebraic extension is that $L=K(\AA)$ and $\AA$ is a (possibly infinite) set of algebraic elements over $K$.
    
    Let $f_j$ be the minimal polynomial of $a_j$ over $K$. If $M$ is an algebraic extension field of $L$ such that $M/K$ is normal (we can set $M=\overline{L}$), then the polynomials $f_j$ decompose in $M[X]$ into a product of linear factors. Now let $L'$ be the subfield of $M$ that is generated over $K$ by the zeros of the $f_j$. Then $L'$ is defined as a splitting field of the $f_j$. Furthermore, we have $L \subset L' \subset M$, and it is clear that $L'/K$ is a normal closure of $L/K$. Also we see that $L'/K$ is finite if $L/K$ is finite. On the other hand, if $L'/K$ is a normal closure of $L/K$, then the field $L'$ contains necessarily a splitting field of the $f_j$ and thus, due to the minimality condition, is a splitting field of the $f_j$ over $K$.


    To establish the uniqueness assertion, consider two normal closures $L'_1/K$ and $L'_2/K$ of $L/K$. As we have just seen, $L'_1$ and $L'_2$ are splitting fields of the polynomials $f_j$ over $K$ and hence also splitting fields of the $f_j$ over $L$. Then the uniqueness of splitting fields says $L_1$ and $L_2$ are isomorphic.


    For the last piece of the statement, consider a $K$-homomorphism $\sigma : L \to M$. $\sigma$ maps the zeros of the $f_j$ to zeros of $f_j$. Since $L'$ is generated over $K$ by these zeros, we see that $K(\sigma_i(L): i \in I) \subset L'$. Conversely, for every zero $a \in L'$ of one of the polynomials $f_j$ we can define a $K$-homomorphism $K(a_j) \to L'$ such that $a_j \mapsto a$. This can be extended by Lemma \ref{extensionlemma} to a $K$-automorphism of an algebraic closure of $L'$ and subsequently be restricted to a $K$-homomorphism $\sigma : L \to L'$, using the normality of $L'/K$. Thus, $a$ lie in $K(\sigma_i(L): i \in I)$, and the equality $L' = K(\sigma_i(L): i \in I)$ is clear.
\end{proof}

\chapter*{Week 3}
\setcounter{chapter}{3}

\section{Separable extensions}
Let $f(x)\in F[x]$ be an irreducible polynomial and $(E=F[\aa],\aa)$ its stem field (or $E$ a possibly larger field containing all the roots). From what we have learnt from the first week, we know an element in $\Aut(E/F)$ shall permute the roots of $f$. It then follows not surprisingly that we want the distinctness of the roots; in other words, we want the roots to be separable.

\begin{defn}
    Let $k$ be a field. A nonzero polynomial $f(x) \in k[x]$ is called \emph{separable} if it has no repeated roots (in any extension field). 
\end{defn}

Recall that the derivative of a polynomial $f(x)=\sum a_i x^i$ is defined to be $f'(x) = \sum ia_ix^{i-1}$. When $f$ has coefficients in $\RR$, this agrees with the definition in calculus. The usual rules for differentiating sums and products still hold, but note that in characteristic $p$ the derivative of $x^p$ is zero.

\begin{thm}\label{sep}
    Let $K$ be a field and $f(X)\in K[X]$ a non-constant polynomial. Then $f$ is separable if and only if $\gcd(f,f')=1$ in $K[X]$. In particular, when $f$ is irreducible, then
    \begin{enumerate}
        \item  if $\char(K)=0$, then $f$ is separable;
        \item  if $\char(K)=p>0$, then $f$ has a multiple root if and only if $f'(X)=0$ if and only if $f(X)=g(X^{p^r})$ for some $g(X)\in K[X]$ where the integer $r$ can be chosen to be maximal such that every zero of $f$ has multiplicity $p$.
    \end{enumerate}
\end{thm}

\begin{proof}
    Let $f(X)$ be a non-constant polynomial in $K[X]$. Suppose $f(X)$ is separable, and let $\aa$ be a root of $f(X)$ (in some extension of $K$. Then $f(X) = (X -\aa)h(X)$ for some $h(x)\not=0$. Since $f'(\aa)=h(\aa)\not=0$, $f$ and $f'$ cannot have any common roots as $\aa$ runs all possible roots of $f$. Hence $\gcd(f,f')=1$. 

    Now suppose $f(X)$ is not separable and $\aa$ is a repeated root (in an extension field). Then we can write $f(X)=(X-\aa)^2g(X)$, where $g(x)$ is non-zero, and then  $f'(X)= (X-\aa)^2g'(X)+2(X-\aa)g(x)$. It follows that $f'(\aa)=0$. By Theorem \ref{minimal}, both $f,f'$ are divisible by the minimal polynomial of $\aa$ in $K[X]$ and then $\gcd(f,f')\not=1$.
    
    \smallskip

    Let's discuss the in "particular part". Assume $f$ is irreducible.
    \begin{enumerate}
        \item Note $\deg(f')=\deg(f)-1\not=0$. Then by the irreducibility of $f$, it follows immediately that $\gcd(f,f')=0$. Hence $f$ is irreducible.
        \item Suppose $f$ is not separable. Since $\gcd(f,f')\not=1$ and $f$ is irreducible, the only possibility is $f'=0$. Take $$f(X)=\sum_{i=0}^n a_i X^i \mbox{ and then } f'(X)=\sum_{i=1}^n ia_i X^{i-1}.$$ The coefficients of $f'$ must vanish, namely, $ia_i =  0$, but then either $a_i=0$ or $i$ is a multiple of $p$. We conclude that $f(X)=g(X^p)$ for some $g(X)\in K[X]$. 
        
        Now assume $f(X)=g(X^{p^r})$ where $r$ is chosen to be maximal. Then $g'\not=0$, otherwise, we can repeat above reasoning for $g$. Since $f$ is irreducible, so is $g$. And therefore $\gcd(g,g')=0$ and then $g$ is separable. We can say $$g(X)=\prod_{i=1}^n (X-a_i)\in \overline{K}[X],$$ where we can assume $f$ and hence $g$ are monic and $\overline{K}$ is a algebraic closure of $K$.  Let $b_i$ be the $p^r$-th root of $a_i$, then we have $$f(X)=\prod_{i=1}^n (X^{p^r}-b_i^{p^r})=\prod_{i=1}^n (X-b_i)^{p^r}\in \overline{K}[X].$$
        It is clear to see that the zeros of $f$ are all of multiplicity $p^r$.
    \end{enumerate} 
\end{proof}

\begin{defn}
    A field $F$ is said to be \emph{perfect} if every irreducible polynomial in $F[x]$ is separable.
\end{defn}

Fortunately, almost all the fields we have good feelings at are perfect, for example, Theorem \ref{ff1} says $\FF_p$ is perfect for any prime $p$.

\begin{thm}
    A field $F$ is perfect if and only if either $F$ has characteristic $0$, or $F$ has characteristic $p$ and the Frobenius map $\sigma: F\to F, x\mapsto x^p$ is an isomorphism.
\end{thm}

\begin{proof}
    Suppose that $F$ has characteristic 0. Let $f$ be an irreducible polynomial. Then $\deg(f')=\deg(f)-1\not=0$ and it follows from the irreducibility of $f$ that $\gcd(f,f')=1$. Therefore, $f$ is separable by Theorem \ref{sep}.  

    Now consider the case when the characteristic of $F$ is a prime $p$. We already see $\sigma$ is a field homomorphism last week. Since field homomorphisms are injective, we only need to consider the surjectivity  of $\sigma$.

    Suppose that $\sigma$ is not surjective and $a\in F$ is not in the image. Then the polynomial $f(x)=x^p-a$ has no roots in $F$. 
    
    Claim: $f(x)$ is irreducible.

    Proof of claim: By Theorem \ref{stem}, let $E/F$ be a finite extension containing a root $\bb$ of $f$ and so that $$f(x)=x^p-a=x^p-\bb^p=(x-\bb)^p \in E[x].$$
    Thus if $f$ factors non-trivially in $F[x]$, then a factor of $f$ looks like $(x-\bb)^j \in F[x]$ for some $1\leq j<p$. The coefficient of $x^{j-1}$ in $(x-\bb)^j$ is $-j\bb$. Since $j\not=0$ in $F$, we conclude $\bb$ lies in $F$ and reach a contradiction. 

    Notice that $f'(x)=px^{p-1} = 0$ in $F[x]$ and then $\gcd(f,f')=f\not=1$ and $f$ is inseparable. We have shown $f$ is irreducible and inseparable and then $F$ is not perfect.
    
    For another direction, suppose that $\sigma$ is surjective and that $f\in F[x]$ is irreducible and inseparable. Similarly to the argument in Theorem \ref{sep}, we get $f$ divides $f'$. If $f'$ was not the zero polynomial, then $\deg(f')< \deg(f)$, which is impossible given $f\mid f'$. Let $f(x)=\sum_{i=0}^d a_ix^i$ then we get $$0=f'(x)=\sum_{i=1}^dia_ix^{i-1}\in F[x].$$  Therefore, $ia_i=0$ for each $i$, which says $a_i=0$ or $i=0$ in $F$. In other words, $a_i=0$ unless $p|i$ and then we can write $$f(x)= \sum_{i=0}^m a_{ip}x^{ip}.$$ But $\sigma$ is surjective, then $a_{ip} =  (\aa^i)^p$ for some $\aa_i\in F$ for each $i$ and $$f(x) =  \sum_{i=0}^m (\aa_i)^p x^{ip} = (\sum_{i=1}^m \aa_ix^i)^p.$$ This polynomial is definitely reducible and we reach a contradiction.
\end{proof}

This theorem says that fields of characteristic 0 and finite fields are perfect. One has to work fairly hard to come up with an example of an inseparable extension. But these do come up naturally in algebraic geometry over fields of positive characteristic.

Now the discussion on finite fields in Week 1 makes more sense now.

\begin{thm}[Galois]
    If $p$ is prime and $n$ is a positive integer, then there exists a field having exactly $p^n$ elements.
\end{thm}

\begin{proof}
    Look at a splitting field of $g(x)=x^{p^n}-x\in \FF_p[x]$. The roots of $g(x)$ are all distinct and forms a subfield of the splitting field by checking addition and multiplication.  
\end{proof}

\begin{thm}[Moore]
    Any two finite fields having exactly $p^n$ elements are isomorphic.
\end{thm}

\begin{proof}
    They are all the splitting fields of $g(x)=x^{p^n}-x\in \FF_p[x]$, and hence isomorphic.
\end{proof}

\begin{remark}
    It follows immediately that $\FF^{p^n}/\FF_p$ is normal.
\end{remark}

A lot of results about field theory that are valid in characteristic 0 carry over to perfect fields in characteristic $p$ (but not everything), and the reader should be attentive to this point when reading texts which try to make life easy by always assuming fields have characteristic 0. You should always check if the theorems (and even proofs) go through to general perfect fields. 

\begin{defn}
    Let $E/F$ be an algebraic extension of fields, and let $\aa \in E$ be algebraic over $F$. We say that $\aa$ is \emph{separable} over $F$ if the minimal polynomial of $\aa$ over $F$ is separable. We say that $E/F$ is \emph{separable} if every element of $E$ is separable over $F$.
\end{defn}


\begin{thm}
    A field $K$ is perfect if and only if every finite extension of $K$ is a separable extension.
\end{thm}

\begin{proof}
    Suppose $K$ is perfect: every irreducible in $K[X]$ is separable. If $L/K$ is a finite extension then the minimal polynomial in $K[X]$ of every element (which is algebraic) of $L$ is irreducible and therefore separable, so $L/K$ is a separable extension.

    Now suppose every finite extension of $K$ is a separable extension. To show $K$ is perfect, let $f(X) \in K[X]$ be irreducible. Consider the stem field $L = K(\aa)$ from Theorem \ref{stem}, where $f(\aa) = 0$. This field is a finite extension of $K$, so a separable extension by hypothesis, so $\aa$ is separable over $K$. Since $f(X)$ is the minimal polynomial of $\aa$ in $K[X]$, it is a separable polynomial.
\end{proof}

It is clear that extensions of fields of characteristic 0 have characteristic 0, and that finite extensions of finite fields are finite, so this theorem is only non-trivial if $F$ is infinite, but has a positive characteristic.

\section{The primitive element theorem}

In this section we prove a result which is not entirely necessary for the discourse, but simplifies the proofs of several other theorems.

\begin{defn}
    A field extension $E/F$ is \emph{primitive} if there is an element $\aa\in E$ such that $E=F(\aa)$.
\end{defn}

\begin{example}
    We already show $\QQ(\sqrt{2},\sqrt{3})=\QQ(\sqrt{2}+\sqrt{3})$, so $\QQ(\sqrt{2},\sqrt{3})/\QQ$ is primitive.
\end{example}

\begin{thm}[Primitive Element Theorem]\label{primitive}
    Let $E/F$ be a finite, separable extension of fields. Then $E/F$ is a primitive extension,
\end{thm}

\begin{proof}
    We will first note that the theorem is trivial if $F$ is a finite field, since in this case $E$ is also finite, and $E^\times$ is a cyclic group. Any generator $\aa$ will clearly satisfy $E = F(\aa)$.

    We also note that it suffices to prove the theorem in the case $E = F(\aa,\bb)$, since the general case follows by induction ($E$ will always have some finite basis over $F$). So we'll suppose that $F$ is infinite and that $E=F(\aa,\bb)$ for some $\aa,\bb\in E$.
    Let $\aa=\aa_1,\aa_2,\cdots,\aa_n$ be the roots of the minimal polynomial $f(x)$ of $\aa$ over $F$ (in $\overline{F}$), and $\bb=\bb_1,\bb_2,\cdots,\bb_m$ the roots of the minimal polynomial $h(x)$ of $\bb$ over $F$ (in $\overline{F}$). All the roots are distinct since $E$ is separable. Since $F$ is infinite, we may choose some $a \in F$ such that $$a\not= \frac{\aa_i-\aa}{\bb-\bb_j}$$ for any $i$ and any $j \not= 1$. Let $\gamma = \aa+a\bb$

    Claim: $F(\aa,\bb)=F(\gamma)$.

    The theorem follows from the claim (and inductions) immediately.

    Proof of Claim: Note that since $\gamma = \aa+a\bb$, we have $F(\aa,\bb)=F(\gamma,\beta)$. It is enough to show $\beta$ is in $F(\gamma)$. Let $g(x) = f(\gamma - ax) \in F(\gamma)[x]$. Note that 
    $$g(\beta) =f(\gamma - a \beta) = f((\aa+a\bb)-a\bb) = f(\aa) = 0.$$
    On the other hand, we cannot have $g(\beta_j)=0$ for $j \not=1$, as $f(\gamma - a\beta_j)=0$ would imply $\gamma-a\beta_j=\aa_i$ for some $i$. Substituting $\gamma= \aa+a\bb$ inside, we reach at
     $$a(\bb-\bb_j)=\aa_i-\aa.$$
    This contradicts to our initial assumption about $a$. Thus $\gcd(g(x), h(x))\in F(\gamma)[x]$ has exactly one root, namely $x =\beta$. And since $h(x)$ has no repeated root, neither does $\gcd(g(x), h(x))$. It follows that $\gcd(g(x), h(x)) \in F(\gamma)[x]$ is a linear polynomial vanishing at $\beta$, and in fact $\beta \in F(\gamma)$. (A hidden fact here is that if $E/F$ is a field extension and $g(x),f(x)$ are in $F[x]$, then $\gcd_{F[x]}(g,f) = \gcd_{E[x]}(g,f)$. A simple explanation is that $\gcd(g,f)$ can be computed with the Euclidean algorithm, which operates on the coefficients of $g$ and $g$ and so never leaves $F$.)
\end{proof}

The next theorem gives a complete description of field extensions which admit a primitive
element.

\begin{thm}[Steinitz]\label{ste}
    Let $E/F$ be a finite extension of fields. Then $E = F(\aa)$ for some $\aa \in E$ if and only if there exist only finitely many distinct intermediate fields $F \subset K \subset E$.
\end{thm}

\begin{proof}
    We have seen that the primitive element property trivially holds in all finite fields, and the property of there only being finitely many intermediate fields does as well. We will suppose, then, that $F$ is infinite.
    Suppose that there are only finitely many intermediate fields. We will show that $E = F(\aa)$ for some $\aa$, and just as in the proof of Theorem \ref{primitive} (which is very similar) we are free to suppose that $E = F(\bb,\gamma$). Consider the fields $F(\bb + a\gamma)$, for $a \in F$. Since all of these fields lie between $F$ and $E$, there must be distinct $a_1 \not= a_2 \in F$ with $F(\bb + a_1 \gamma) = F(\bb + a_2\gamma)$. Now, since $\bb + a_1 \gamma$ and $\bb + a_2 \gamma$ are both in this field, so is $(a_2 - a_1)\gamma$, and hence $\gamma$ (since $a_2 - a_1 \not= 0$). It follows $\beta = (\beta+a_1\gamma) -a_1\gamma$ is in $F(\beta +a_1)\gamma$ as well. Thus $E = F(\bb,\gamma) = F(\bb + a_1\gamma)$, and we are done.

    For another direction, suppose that $E = F(\aa)$. For each intermediate field $F \subset K \subset E$, we take $f_K(x) \in K[x] \subset E[x]$ to be the minimal polynomial of $\aa$ over $K$. By unique factorization in $E[x]$, each $f_K(x)$ must be a (monic) divisor of $f_F(x)$ (where $f_F(x)$ is considered as a polynomial in $E[x]$), and there are only finitely many of these. It suffices to show, then, that this function $K \to f_K(x)$ is one-to-one. Let $F(f_K)$ be the field generated over $F$ by the coefficients of $f_K(x)$. Then certainly $F(f_K) \subset K$. On the other hand, $f_K(x)$ is an irreducible monic polynomial with coefficients in $F(f_K)$, which vanishes at $\aa$, and so $\aa$ has degree $\deg(f_K)$ over $F(f_K)$. It follows from the fact that
   $$[F(\aa):F(f_K)] = [F(\aa):K][K:F(f_K)]$$ that $[K : F(f_K)] = 1$. In other words, $K = F(f_K)$, showing that the map is injective. It follows at once that there are only finitely many subfields $F \subset K \subset E$.    
\end{proof}

\begin{remark}
    Patrick Ingram refers Theorem \ref{ste} as the primitive element theorem. But Rotman labels it after Steinitz. And many textbooks (Rotman included) refers Theorem \ref{primitive} as the primitive element theorem instead. We follow Rotman's conventions.
\end{remark}

\begin{thm}\label{acc}
    Let $F$ be a field, $\overline{F}$ its algebraic closure, and let $F \subset E \subset \overline{F}$ be a finite separable extension of $F$. Then there are precisely $[E:F]$ distinct embeddings $E\to \overline{F}$ which fix $F$.
\end{thm}

\begin{proof}
    By the primitive element theorem, we may suppose that $E = F(\aa)$, and let $f(x) \in F[x]$ be the minimal polynomial of $\aa$. For each root $\aa' \in \overline{F}$, we have an isomorphism $$E = F(\aa) \cong F[x]/(f(x)) \cong F(\aa') \subset \overline{F}.$$
    This gives us $\deg(f) = [E:F]$ embeddings of $E$ into $\overline{F}$, which must be distinct, since they send $\aa$ to the $\deg(f)$ distinct roots of $f$ (as $E/F$ is separable). 
    On the other hand, any embedding $F(\aa) \to \overline{F}$ which fixes $F$ is determined entirely by the value at $\aa$ (which must be a root of the same minimal polynomials in $F[x]$ as $\aa$), and so these are the only possible embeddings of $E \to \overline{F}$.
\end{proof}

\begin{cor}\label{cor1}
    Let $E/F$ be a finite separable extension. Then
     $$|\Aut(E/F)| \leq [E:F].$$
\end{cor}

\begin{proof}
    Each distinct automorphism $\sigma\in\Aut(E/F)$ gives a distinct embedding $\sigma : E \to E \subset \overline{F},$ which fixes $F$.
\end{proof}

\section{Separable degree}

We will now proceed with a slightly more detailed examination of extensions which are not separable.

\begin{defn}
    Let $E/F$ be a finite extension of fields, and let $\overline{F}$ be an algebraic closure of $F$ (and fix an embedding $\iota: F\hookrightarrow\overline{F}$). We define the separable degree of $E$ over $F$, denoted $[E : F]_s$, to be the number of distinct embeddings $$\sigma: E\to \overline{F},$$ which fix $F$ pointwise (indeed, $\sigma|_F=\iota$).
\end{defn}

\begin{lemma}\label{seplem1}
    Let $K \subset L = K(\aa)$ be a primitive algebraic field extension with minimal polynomial $f \in K[X]$ of $\aa$ over $K$.
    \begin{enumerate}
        \item The separable degree $[L : K]_s$ equals the number of different zeros of $f$ in an algebraic closure of $K$.
        \item The element $\aa$ is separable over $K$ if and only if $[L : K] = [L : K]_s$.
        \item Assume $\char(K) = p >$ 0 and let $p^r$ be the multiplicity of the zero $\aa$ of $f$. Then $[L:K] =  p^r [L:K]_s$.
    \end{enumerate}
\end{lemma}

\begin{proof}
    Parts of the proof have appeared discretely many times.
    \begin{enumerate}
        \item This is just a reformulation of Lemma \ref{splitlemma1}.
        \item $f$ have at most $\deg(f)=[L:K]$ roots and each distinct root gives a distinct embedding $L\to \overline{K}$. There are $[L : K]_s=[L : K]$ such embeddings hence all the roots are distinct.
        \item If a root of $f$ has multiplicity $p^r$, then $f(X)=g(X^{p^r})$ for some $g(X)\in K[X]$. Hence there are $\deg(f)/p^r$ distinct roots and so is the number of separation degree. In other words, $[L:K] =  p^r [L:K]_s$.
    \end{enumerate}
\end{proof}

\begin{thm}\label{sepdeg}
    Let $K \subset L \subset M$ be algebraic field extensions. Then
    $$[M : K]_s = [M : L]_s [L : K]_s.$$
\end{thm}

\begin{proof}
    Fix an algebraic closure $\overline{K}$ of $K$. Then $K \subset L \subset M \subset \overline{K}$, and we may view $K$ also as an algebraic closure of $M$ and of $L$. Furthermore, let
    $$\Hom_K(L,\overline{K}) = \{\sigma_i: i \in I\},\quad \Hom_L(M,\overline{K}) = \{\tau_j: j \in J\},$$ where in each case, the $\sigma_i$ as well as the $\tau_j$ are distinct. Now extend the $K$-homomorphisms $\sigma_i : L \to \overline{K}$ via Lemma \ref{extensionlemma} to $K$-automorphisms $\overline{\sigma_i}: \overline{K} \to \overline{K}$. 
    The desired multiplicative formula will then be a consequence of the following two assertions:
    \begin{enumerate}
        \item The maps $\overline{\sigma_i} \circ \tau_j : M \to \overline{K}, i \in I, j \in J$, are distinct.
        \item $\Hom_K(M,\overline{K}) = \{\overline{\sigma_i} \circ \tau_j ; i \in I, j \in J\}.$
    \end{enumerate}
    
    To verify assertion (1), consider an equation of type $\overline{\sigma_i} \circ \tau_j = \overline{\sigma_{i'}} \circ \tau_{j'}$. Since $\tau_j$ and $\tau_{j'}$ restrict to the identity on $L$, we can conclude that $\overline{\sigma_i} = \overline{\sigma_{i'}}$ and hence $i = i'$. The latter implies $\tau_j = \tau_{j'}$, and thus $j = j'$. It follows that the maps specified in (1) are distinct.
    
    Since they are  $K$-homomorphisms, it remains to show for (2) that every $K$-homomorphism $\tau : M \to \overline{K}$ is as specified in (1). For $\tau \in \Hom_K(M,\overline{K})$ we have $\tau|_L \in \Hom_L(M,\overline{K})$. Hence, there exists an index $i \in I$ such that $\tau|_L = \sigma_i$. Then we obtain $\overline{\sigma_i}^{-1}\circ\tau \in \Hom_L(M,\overline{K})$. Therefore, $\tau=\overline{\sigma_i}\circ \tau_j$ and (2) is clear.
\end{proof}

Put all the knowledge together, we get the following statement:

\begin{cor}
    Let $K \subset L$ be a finite field extension.
    \begin{enumerate}
        \item If $\char(K) = 0$, then $[L : K] = [L : K]_s$.
        \item If $\char(K) = p > 0$, then $[L : K] = p^r[L : K]_s$ for some integer $r$. In particular, if $\char(K)\nmid [L:K]$, then $[L : K]_s = [L : K]$.
    \end{enumerate}
\end{cor}

\begin{thm}
    For a finite field extension $K \subset L$ the following conditions are equivalent: \begin{enumerate}
        \item $L/K$ is separable.
        \item There exist elements $a_1, \cdots, a_n \in L$ that are separable over $K$ and satisfy $$L = K(a_1, \cdots, a_n).$$
        \item $[L : K]_s = [L : K]$.
    \end{enumerate}
\end{thm}

\begin{proof}
    The implication from (1) to (2) is trivial. If $a \in L$ is separable over $K$, then the same is true over every intermediate field of $L/K$. Therefore, using the multiplicative formulas, the implication from (2) to (3) can be reduced to the case of a simple field extension. However, that case was already dealt with in Lemma \ref{seplem1}.

    It remains to show that (3) implies (1). Consider an element $a \in L$ with its minimal polynomial $f \in K[X]$ over $K$. To show that $a$ is separable over $K$, which amounts to showing that f admits only simple zeros, we are reduced to the case $\char(K) = p > 0$. Then, by above  corollary, there is a integer $r$ such that every root of $f$ has multiplicity $r$ and  
    $$[K(a):K] = p^r[K(a):K]_s.$$
    Using the multiplicative formulas in conjunction with the estimate between the degree and the separable degree, we obtain
    $$[L:K]=[L:K(a)][K(a),K] \geq [L:K(a)]_s p^r [K(a):K]_s= p^r[L:K]_s.$$
    Now, if $[L : K]_s = [L : K]$, we must have $r = 0$. Then all zeros of $f$ are simple and $a$ is separable over $K$, which shows that (3) implies condition (1).
\end{proof}

\begin{cor}
    Let $K \subset L \subset M$ be algebraic field extensions. Then $M/K$ is separable if and only if $M/L$ and $L/K$ are separable. 
\end{cor}

\begin{cor}
    Let $K/F$ be a finite algebraic extension. If $K=F(\aa_1,\cdots,\aa_n)$ and each $\aa_i$ is separable, then $K/F$ is separable. 
\end{cor}

The usefulness of this corollary is that it gives a practical way to check a finite extension $L/K$ is separable: rather than show every element of $L$ is separable over $K$ it suffices to show there is a set of field generators for $L/K$ that are each separable over $K$.

\section{Inseparable extensions}

A field of characteristic 0 is perfect. We assume in this section that the characteristic of a field is $p$.

\begin{defn}
    A non-constant polynomial $f\in F[x]$ is \emph{purely inseparable} if it admits precisely one root.
\end{defn}

If $f$ is irreducible and purely inseparable, then $f(x)=(x-\aa)^{p^n} = x^{p^n}-c$. 

\begin{defn}
    Let $L/K$ be an algebraic field extension. An element $\aa\in L$ is \emph{purely inseparable} over $K$ if $\aa$ is a zero of a purely inseparable polynomial, namely, the minim polynomial $\irr(\aa,K)$ is of the form $x^{p^n}-c$ from some integer $n$ and $c\in K$. $L$ is \emph{purely inseparable} over $K$ if every element in $L\setminus K$ is purely inseparable.
\end{defn}

It follows immediately that purely inseparable field extensions are normal.

\begin{thm}
    For a finite algebraic extension $L/K$, the following conditions are equivalent: \begin{enumerate}
        \item $L$ is purely inseparable over $K$;
        \item There are elements $a_1,\cdots,a_n\in L$ such that $L=K(a_1,\cdots,a_n)$ and each $a_i$ is purely inseparable over $K$;
        \item $[L:K]_s=1$;
        \item For every element $a\in L\setminus K$, there is an integer $n$ such that $a^{p^n}\in K$.
    \end{enumerate}
\end{thm}

\begin{proof} 
    We will show $(1)\Longrightarrow (2) \Longrightarrow (3) \Longrightarrow (4) \Longrightarrow (1)$.


    (1) $\Longrightarrow$ (2) is clear.


    (2) $\Longrightarrow$ (3):  It is enough to show $[K(a_i):K]_s=1$ for all $i$, because any $K$-map $L\to \overline{K}$ is uniquely determined by the image of the elements $\aa_i$. On the other hand, the minimal polynomial of $a_i$ admits only a single root in $\overline{K}$ and only one $K$-map $L\to \overline{K}$. Hence $[K(a_i):K]_s=1$ and so $[L:K]_s=1$.
    
    
    (3) $\Longrightarrow$ (4): For any $a\in L\setminus K$, we have $$[L:K(a)]_s [K(a):K]_s = [L:K]_s.$$ So $[K(a):K]_s=1$. So the minimal polynomial of $a$ admits only one (repeated) root which is of the form $x^{p^n}-c$. Hence $a^{p^n}\in K$.


    (4) $\Longrightarrow$ (1):  For any $a\in L\setminus K$, we have $a^{p^n}=c\in K$. Hence $\irr(a,K)$ divides $x^{p^n}-c$ which has only one (repeated) root. So is $\irr(a,K)$ and then $a$ is purely inseparable over $K$.
\end{proof}

\begin{example}
    Consider the field extension $\FF_p(t)/\FF_p(t^n)$. $x^p-t^p\in \FF_p(t^n)[x]$ is irreducible (by Eisenstein's Criterion) and purely inseparable over $\FF_p(t)$. It is indeed a purely inseparable extension by above theorem. 
\end{example}

We finish this section by showing that every algebraic field extension can be decomposed into a separable extension, followed by a purely inseparable extension. 

\begin{thm}
    Let $L/K$ be an algebraic field extension. Then there exists a unique intermediate field $$K_s=\{a\in L:a \mbox{ separable over} K\}$$ of $L/K$ such that $L/K_s$ is purely inseparable and $K_s/K$ is separable. The field $K_s$ is called the separable closure of $K$ in $L$. We also have  $[L : K]_s = [K_s : K]$. If $L/K$ is normal, the extension $K_s/K$ is normal, too.
\end{thm}

\begin{proof}
    We first note that $K_s$ is a field. Indeed, for $a, b\in K_s$, that $K(a, b)$ is a separable extension of $K$, so that $K(a, b) \subset K_s$. Therefore, $K_s$ is the biggest separable extension of $K$ that is contained in $L$.
    
    Now consider an element $a\in L$ and let $f\in K_s[X]$ be the minimal polynomial of $a$ over $K_s$. Then, there exists a separable polynomial $g\in K_s[X]$ such that $f(x) = g(x^{p^r})$ for some integer $r$. Moreover, $g$ is irreducible, since $f$ is irreducible. It follows that $g$ is the minimal polynomial of $c = a^{p^r}$ over $K_s$ and that $c$ is separable over $K_s$ hence separable over $K$. However, then we must have $c\in K_s$ and therefore $g(x) = x - c$, as well as $f(x) = x^{p^r} - c$. Thus, $a$ is purely inseparable over $K_s$, and the same is true for $L$ over $K_s$.


    Since $L/K_s$ is purely inseparable and $K_s/K$ is separable, we get the stated relation on degrees $$[L:K]_s=[L:K_s]_s[K_s:K]_s=[K_s:K].$$
    To justify the uniqueness of $K_s$, consider an intermediate field $K'$ of $L/K$ such that $L/K'$ is purely inseparable and $K'/K$ is separable. Then we have $K' \subset K_s$ by the definition of $K_s$, and the extension $K_s/K'$ is separable. On the other hand, the latter extension is purely inseparable, since $L/K'$ is purely inseparable.This shows that $K_s/K'$ is trivial and hence that $K_s$ is unique, as claimed.


    It remains to show that $K_s/K$ is normal if the same is true for $L/K$. To do this, consider a $K$-homomorphism $\sigma : K_s\to \overline{L}$ into an algebraic closure of $L$. Since we can view $\overline{L}$ as an algebraic closure of $K$ as well, we can extend $\sigma$ due to Lemma \ref{extensionlemma} to a $K$-homomorphism $\sigma' : L\to \overline{L}$. Now, assuming $L/K$ to be normal, $\sigma'$ restricts to a $K$-automorphism of $L$. Furthermore, the uniqueness of $K_s$ implies that $\sigma$ restricts to a $K$-automorphism of $K_s$, and it follows that $K_s/K$ is normal.
\end{proof}

\begin{example}
    Consider the function field $F=\FF_x(x)$ and the field extension $E=F(x^{1/6})$. Then it is easily seen that $E=F(x^{1/2},x^{1/3})$. $x^{1/2}$ is purely inseparable and $x^{1/3}$ is separable. Hence $F(x^{1/3})$ is the separable closure and $[E:F]_s=3$.
\end{example}

\chapter*{Week 4}
\setcounter{chapter}{4}

\section{Galois extensions}

There is an alternate characterization of normal extensions in the separable case:

\begin{thm}\label{num1}
    Let $E/F$ be a finite separable extension. The $E/F$ is normal if and only if $$|\Aut(E/F)|=[E:F].$$
\end{thm}

\begin{proof}
    Suppose that $E/F$ is normal. By Theorem \ref{acc} there are $[E:F]$ embeddings $E\to \overline{F}$ fixing $F$ since $E/F$ is finite and separable. And Theorem \ref{normal} says each of the embedding has image $E$ as well since $E/F$ is normal.
    
    On the other hand, suppose $|\Aut(E/F)| = [E:F]$. By Theorem \ref{acc} there are precisely $[E:F]$ distinct embeddings $E\to \overline{F}$ fixing $F$ since $E/F$ is finite and separable. But $|\Aut(E/F)| = [E:F]$, so each embedding must give rise to an automorphism $E\to E$, since $|\Aut(E/F)|$ is just the number of these embeddings under which the image of $E$ is $E$. So $|\Aut(E/F)| = [E:F]$ implies $E/F$ is a normal extension.
\end{proof}

Indeed, the equality $|\Aut(E/F)|=[E:F]$ is all we want.

\begin{defn}
    Let $E/F$ be a finite field extension. $E/F$ is a \emph{Galois extension} if $E/F$ is separable and normal. The \emph{Galois group} of the extension is simply $\Gal(E/F)=\Aut(E/F)$.
\end{defn}

We will simply say $E$ is Galois if $F$ is clear in the context. The change of notation of the group $\Aut(E/F)$ to $\Gal(E/F)$ is simply to remind us this is a Galois extension.

In other words, a finite extension $E/F$ is Galois if it is separable, and contains either no roots or all of the roots of a given irreducible polynomial $f(x)\in F[x]$. Since every finite separable extension is primitive, $E = F(\aa)$, we see that $E$ is Galois if and only if $E$ contains all of the roots of the minimal polynomial of $\aa$ (in some algebraic closure of $F$).

\begin{example}
    We have seen that $\QQ(\sqrt{2})/\QQ$ is Galois while $\QQ(\sqrt[3]{2})/\QQ$ is not.
\end{example}

\begin{prop}
    Let $L/K$ be a finite extension. Then
    \begin{enumerate}
        \item if $L/K$ is separable, then the normal closure $L'/K$ is Galois;
        \item if $L/K$ is normal, the the separable closure $K_s/K$ is Galois.
    \end{enumerate}
\end{prop}

\begin{example}[Quadratic extensions]
    Let $F$ be a field of any characteristic other than 2 and let $E/F$ be an extension with $[E:F]=2$.

    Claim: $E/F$ is Galois.

    Note that for any $\aa\in E\setminus F$, we have $[F(\aa):F] > 1$. In the mean time, $$2 = [E:F]=[E:F(\aa)][F(\aa):F] > [E:F(\aa)],$$ so we must have $[E:F(\aa)]=1$, namely, $E=F(\aa)$. If the minimal polynomial of $\aa$ is $$f(x)=x^2+bx+c,$$ we can replace $\aa$ with $\aa+b/2$ ($\char(F)\not=2$!) and assume with out of generality that $b=0$ by noting that $F(\aa)=F(\aa+b/2)$. And then the minimal polynomial turns out to be $f(x)=x^2-D,$ for some non-zero $D\in F$. $f$ is separable since $f'(x)=2x\not=0$ ($\char(F)\not=2$!) and then $\gcd(f,f')=1$.

    There are two embeddings $E\to \overline{F}$, namely that defined by $\aa\mapsto\aa$ and that defined by $\aa\mapsto -\aa$. But $-\aa\in F(\aa)$, so they are just automorphisms of $E$. Hence $$\Gal(E/F)\cong \ZZ/2\ZZ.$$
\end{example}

\begin{example}
    Let $E$ be the field $\QQ(\sqrt[4]{2})$, where $\sqrt[4]{2}$ is the real positive fourth root of $2$. Clearly, the minimal polynomial of $\sqrt[4]{2}$ is $f(x)=x^4-2$ in $\QQ[x]$ and so $[E:\QQ]=4$. But $|\Aut(E/\QQ)|=2$ since $\sigma(\sqrt[4]{2})$ is also a fourth root of 2 and $E\subset \RR$ contains two fourth root of 2 only. The non-trivial element $\sigma\in\Aut(E/\QQ)$ is then defined by $$\sigma(a_1+a_2\sqrt[4]{2} +a_3\sqrt{2} +a_4 (\sqrt[4]{2})^3) =a_1-a_2\sqrt[4]{2} +a_3\sqrt{2} -a_4 (\sqrt[4]{2})^3.$$


    The splitting field of $f(x)$ is the larger field $K=\QQ(\sqrt[4]{2},i)=E(i)$, indeed, the normal closure of $E$. It is clear that this is the same field generated by the roots of $f(x)$, which are $\sqrt[4]{2},i\sqrt[4]{2},-\sqrt[4]{2},-i\sqrt[4]{2}$. Note that $[K:E]=2$ and so $[K:\QQ]=[K:E][E:\QQ]=8$. By Theorem \ref{num1}, $|\Gal(K/\QQ)|=8$.

    The computations of Galois group will be discuss in details later on. Nevertheless, we can still investigate $\Gal(K/\QQ)$ in this case. Since $K=\QQ(\sqrt[4]{2},i)$, an element $\sigma\in \Gal(K/\QQ)$ is entirely determined by values $\sigma(\sqrt[4]{2})$ and $\sigma(i)$, where $\sigma(\sqrt[4]{2})$ is a root of $x^4-1$ and $\sigma(i)$ is a root of $x^2+1$. So there are 8 possible choices given by the combinations of $$\sigma(\sqrt[4]{2})\in\{\sqrt[4]{2},i\sqrt[4]{2},-\sqrt[4]{2},-i\sqrt[4]{2}\}\mbox{ and }\sigma(i)\in\{i,-i\}.$$
    There are 8 options only and indeed $|\Gal(K/\QQ)|=8$, so each combination must defines an element of $\Gal(K/\QQ)$. 
    
    To simplify notations, let $\sigma_{a,b}\in \Gal(K/\QQ)$ be the automorphism defined by $\sqrt[4]{2}\mapsto i^a \sqrt[4]{2}$ and $i\to (-1)^bi$. Note that $i^4=1$ and $(-1)^2=1$, so we can require $a\in\ZZ/4\ZZ$ and $b\in \ZZ/2\ZZ$. $\sigma_{0,1}$ and $\sigma_{2,0}$ are two elements of order 2, hence $\Gal(K/\QQ)$ is not isomorphic to $\ZZ/8\ZZ$ nor the quaternion group $Q_8$. $\sigma_{1,0}$ is an element of order 4. So we suspect that $\Gal(K/\QQ)$ is isomorphic to the dihedral group $D_4$ since it is not abelian. Indeed, we have $$\sigma_{c,d}\circ\sigma_{a,b}(i) = \sigma_{c,d}((-1)^b i) = (-1)^{b+d} i$$
    and \begin{align*}
        \sigma_{c,d}\circ\sigma_{a,b}(\sqrt[4]{2}) &= \sigma_{c,d}(i^a \sqrt[4]{2}) =  \sigma_{c,d}(i^a)\sigma_{c,d}(\sqrt[4]{2}) \\
        &= ((-1)^d i)^a i^c\sqrt[4]{2} = i^{a+c+2ad} \sqrt[4]{2}.
    \end{align*}
    That is $$\sigma_{c,d}\circ\sigma_{a,b} = \sigma_{c+(1+2d)a,b+d}.$$ Therefore, $\Gal(K/\QQ) \cong \ZZ/4\ZZ \rtimes \ZZ/2\ZZ.$
\end{example}

The following theorem of Artin gives a simple way of constructing examples
of Galois extensions.

\begin{thm}[Artin]\label{artin}
    Let $E$ be a field, and let $G \subset \Aut(E)$ be a subgroup with $n$ elements. If $$E^G = \{ \aa\in E : \sigma(\aa) = \aa \mbox{ for all } \sigma \in G\},$$ then $E^G$ is a field, $E/E^G$ is a Galois extension satisfying $G = \Gal(E/E^G)$ and $[E : E^G] = n$.
\end{thm}

We have to be somewhat careful with the theorem above. In particular, given an extension of fields $E/F$ it is not always true (or even "usually true") that $F = E^{\Aut(E/F)}$. The best one can say for sure is that $F \subset E^{\Aut(E/F)}$. A good example of this is $E = \RR$ and $F = \QQ$, where $\Aut(\RR/\QQ)$ is trivial, and so $$\RR^{\Aut(\RR/\QQ)} = \RR \not= \QQ.$$
In this case, Artin's Theorem just says that $\RR/\RR$ is a Galois extension with trivial Galois group. This should actually be somewhat reassuring, since $[\RR : \QQ]$ is not finite! 

Before proving Artin's Theorem, we will state a lemma which will be
needed.

\begin{lemma}
    Let $E/F$ be a separable, algebraic extension. Suppose that for some natural number $n$, every $\aa \in E$ satisfies $[F(\aa) : F] \leq n$. Then $E/F$ is a finite extension, and $[E : F] \leq n$.
\end{lemma}

\begin{proof}
    Let $n$ in the statement be as small as possible, and choose $\aa \in E$ such that $[F(\aa) : F] = n$. 
    
    We claim that, in fact, $E = F(\aa)$.

    If not, then there is some $\bb \in E\setminus F(\aa)$, and so $F(\aa,\bb)$ is a proper extension of $F(\aa)$. Now, $F(\aa,\bb)/F$ is a finite, separable extension of fields, and so by Theorem \ref{primitive}, there is a $\gamma \in F(\aa,\bb)$ with $F(\aa,\bb) = F(\gamma)$. But since $\bb \notin F(\aa)$, we have $\gamma\in E$ and
    $$[F(\gamma) : F] = [F(\aa,\bb) : F(\aa)][F(\aa) : F] > [F(\aa) : F] = n.$$
    This is impossible by the hypotheses of the theorem, and so we must have had $E = F(\aa)$.
\end{proof}

\begin{proof}[Proof of Theorem \ref{artin}]
    It is easy to check that $E^G$ is a subfield of $E$. We need to show that $E/E^G$ is a finite, separable, and normal extension. For each element $\aa \in E$, choose a subset $S \subset G$ such that $\{\sigma(\aa): \sigma\in S\}$ is as large as possible, but such that $\sigma_1(\aa) \not= \sigma_2(\aa)$ for $\sigma_1\not=\sigma_2$.
    Now define $$f_\aa(x) =\prod_{\sigma\in S}(x-\sigma(\aa)) \in E[x].$$
    
    Our first claim is that $f_\aa(x) \in E^G[x]$.
    To see this, first note that if $\tau \in G$, then
    $$\{\tau\circ\sigma(\aa):\sigma\in S\}\subset \{\sigma(\aa): \sigma\in S\}.$$
    If not, then there is some $\sigma \in S$ such that $\tau \circ \sigma(\aa)$ is distinct from all $\sigma'(\aa)$ for $\sigma' \in S$, and so $S\sup \{ \tau\circ \sigma \}$ contradicts the maximality of $S$. But $\tau$ is an injection, so $$|\{\tau\circ\sigma(\aa):\sigma\in S\}|=|\{\sigma(\aa): \sigma\in S\}|$$ and hence the inclusion above shows that $$\{\tau\circ\sigma(\aa):\sigma\in S\} = \{\sigma(\aa): \sigma\in S\}.$$
    So applying $\tau \in S$ to the elements of $\{\sigma(\aa): \sigma\in S\}$ simply permutes them. But the coefficients of $f_\aa(x)$ are symmetric in the $\sigma(\aa)$, and so they are fixed by $\tau$. Since $\tau \in G$ was arbitrary, the coefficients of $f_\aa(x)$ are in $E^G$.

    Now, we also note that $f_\aa(\aa) = 0$. If not, adding the identity map to $S$ would again contradict the maximality of $S$. Thus $\aa$ is the root of the polynomial $f_\aa(x) \in E^G[x]$ which, by construction, has no repeated roots, and has degree at most $|G|$. As $\aa \in E$ was arbitrary, we see that the extension $E/E^G$ is algebraic and separable, and that for all $\aa \in E, [E^G(\aa) : E^G] \leq |G|$. By the preceding lemma, we may conclude that $E/E^G$ is a finite extension of degree at most $|G|$. Now, note that any element of $G$ fixes $E^G$ (by definition), and so $G \subset \Aut(E/E^G)$. But combining the fact $E/E^G$ is a finite, separable extension this gives $$|G| \leq |\Aut(E/E^G)| \leq [E:E^G] \leq |G|.$$ 
    And so in fact we have the equalities above and that the equality $\Aut(E/E^G) = [E : EG]$ implies $E/E^G$ is normal by Theorem \ref{num1}. Therefore, $E/E^G$ is Galois and $\Gal(E/E^G) = G$.
\end{proof}

Artin's Theorem has an immediate, and very useful, corollary:


\begin{cor}\label{artincor}
    If $E/F$ is a finite Galois extension, then $E^{\Gal(E/F)} = F$.
\end{cor}

\begin{proof}
    It's clear that $F \subset E^{\Gal(E/F)}\subset E$, simply because every element of $F$ is fixed by every element of $\Gal(E/F)$. So we have $$|\Gal(E/F)|=[E : F] = [E : E^{\Gal(E/F)}][E^{\Gal(E/F)} : F] =|\Gal(E/F)|[E^{\Gal(E/F)} : F].$$
    It must be the case that $[E^{\Gal(E/F)} : F] = 1$, and so the fields are the same.
\end{proof}

\begin{example}
    We know that $\CC/\RR$ is a Galois extension, with Galois group generated by complex conjugation. The corollary above shows that $\RR$ is exactly the set of complex numbers fixed by complex conjugation (which, of course, we already knew).

    By contrast, if $E =\QQ(\sqrt[3]{2})$, we know that $E^{\Aut(E/\QQ)} = E \not= \QQ$.
\end{example}


\begin{cor}
    For every finite group $G$, there is a Galois extension $E/F$ such that $$\Gal(E/F) \cong G.$$
\end{cor}

\begin{proof}
    Write $G=\{g_1,\cdots,g_n\}$. Take $E=\QQ(x_{g_1},\cdots,x_{g_n})$ the fraction field of $n$ indeterminates which are labeled by elements of $G$. Then we can extend the canonical the action of $G$ on $\{x_{g_1},\cdots,x_{g_n}\}$ to $E$ via $$\sigma_g: E\to E,\quad  \frac{f(x_{g_1},\cdots,x_{g_n})}{g(x_{g_1},\cdots,x_{g_n})}\mapsto  \frac{f(x_{gg_1},\cdots,x_{gg_n})}{g(x_{gg_1},\cdots,x_{gg_n})},$$ for each element $g\in G$. It is straightforward to check $\sigma_g$ is an automorphism fixing $\QQ$. By Artin's Theorem, $E/E^{S_n}$ is Galois. By the fundamental theorem, $E/E^G$ is also a Galois extension with Galois group $G$.
\end{proof}


\section{The fundamental theorem}


The original aim was to understand finite field extensions $E/F$ using the group $\Aut(E/F)$. In general, it's possible that $E/F$ is non-trivial, but $\Aut(E/F)$ is trivial, which means in general it's unlikely that $\Aut(E/F)$ will shed a lot of light on the structure of $E/F$. In the case where $E/F$ is Galois, though, we know that $\Aut(E/F)$ is as large as it possibly could be, and hence gives as much information as possible. One might expect, than, that Galois extensions are the ones for which $\Aut(E/F)$ gives as much information as possible. The purpose of this section is to show that $\Gal(E/F)$ gives a great deal of information about the intermediate fields. Artin's Theorem raises a natural question: Given a finite Galois extension $E/F$, for any subgroup $H \subset  \Gal(E/F)$, the field $E^H$ is a subfield of $E$, and $E/E^H$ is a Galois extension; what is the exact relation between $H$ and $E^H$?


For one thing, if $H \subset  \Gal(E/F)$, then $F \subset  E^H \subset  E$, and $E/E^H$ is a Galois extension of degree $|H|$. All of this follows from Artin's Theorem. Perhaps surprisingly, it turns out that this function is both injective and surjective (onto the collection of intermediate fields between $E$ and $F$). This is, essentially, the fundamental theorem of Galois Theory. We will prove this theorem as a sequence of lemmas, and then state the complete result at the end of the section.


\begin{lemma}
    Let $E/F$ be a finite Galois extension. Then the function $H \mapsto E^H$, defined on subgroups of $\Gal(E/F)$, is inclusion-reversing.
\end{lemma}

\begin{proof}
    We want to show that if $H_1 \subset H_2$ are two subgroups of $E$, then $E^{H_1}\supseteq E^{H_2}$. Suppose that $\aa\in E^{H_2}$. Then for every $\sigma\in H_2$, we have $\sigma(\aa) = \aa$. But $H_1 \subset  H_2$, so $\sigma(\aa) = \aa$ for every $\sigma\in H_1$. This shows that $\aa\in E^{H_1}$. Since $\aa\in E^{H_2}$ was arbitrary, we have $E^{H_1}\supseteq E^{H_2}$.
\end{proof}

\begin{lemma}
    Let $E/F$ be a finite Galois extension. Then the function $H \mapsto E^H$ is a bijection between subgroups of $\Gal(E/F)$ and intermediate fields $F \subset  K \subset  E$.
\end{lemma}

\begin{proof}
    We will prove this by exhibiting an inverse for the function $H \mapsto E^H$. We claim that the inverse function is $K \mapsto \Gal(E/K)$. One side of this is Artin's Theorem, which shows that $E/E^H$ is a Galois extension, and $\Gal(E/E^H) = H$. On the other hand, if $F \subset  K \subset  E$, then $E/K$ is a finite, separable extension (separability follows from the fact that the minimal polynomial of any element of $E$ over $K$ divides the minimal polynomial over $F$, which will have no repeated roots). The extension $E/K$ also has to be normal, because if $E$ is the splitting field of $f(x)\in F[x]$ over $F$, then $E$ is also the splitting field of $f(x)$ over $K$. So $E/K$ is a Galois extension whenever $F \subset  K \subset  E$. It follows from Corollary \ref{artincor} that $E^{\Gal(E/K)} = K$. Thus we've shown that the maps $H \mapsto E^H$ and $K \mapsto \Gal(E/K)$ are inverses. They are, then, both bijections.
\end{proof}


It is always true, when $E/F$ is a finite Galois extension, and $F \subset  K \subset  E$, that $E/K$ is a Galois extension, as noted in the proof above. It is \emph{not} typically true that $K/F$ is also a Galois extension. A lemma below tells us precisely when this happens, but first we'll see a simple example of the bijection above.

\begin{example}
    Let $E = \QQ(\sqrt{2},\sqrt{3})$. Then $E$ is the splitting field over $\QQ$ of the polynomial $(x^2- 2)(x^2 - 3)$, and hence is a Galois extension. As we've seen before, $[\QQ(\sqrt{2})] = 2$. This confirms the plausible statement that $[E : \QQ] = 4$. So there are four automorphisms of $E$ fixing $\QQ$, and they are defined by what they do to $\sqrt{2}$ and $\sqrt{3}$. If $\sigma\in \Gal(E/\QQ)$, then $\sigma(\sqrt{2})=\pm \sqrt{2}$ and $\sigma(\sqrt{3})=\pm \sqrt{3}$. So we see precisely what the four automorphisms are. Indeed, $\Gal(E/\QQ) \cong \ZZ/2\ZZ \times \ZZ/2\ZZ$, by the isomorphism $(a, b) \mapsto \sigma_{a,b}$ defined by $$\sigma_{a,b}(\sqrt{2})=(-1)^a \sqrt{2}\mbox{ and }\sigma_{a,b}(\sqrt{3})=(-1)^b \sqrt{3}.$$

    The group $\Gal(E/\QQ)$ has three distinct subgroups:
    $$H_1=\{id, \sigma_{1,0}\}, H_2=\{id,\sigma_{0,1}\}, H_3=\{id, \sigma_{1,1}\}.$$ 
    We see that each of the fields $E^{H_1}, E^{H_2}, E^{H_3}$ are degree 2 extensions of $\QQ$. Since $\sigma_{1,0}(\sqrt{3})=\sqrt{3}, \sigma_{0,1}(\sqrt{2})=\sqrt{2}$ and $\sigma_{1,1}(\sqrt{6})=\sqrt{6}$, we see that $$E^{H_1} = \QQ(\sqrt{3}), E^{H_2}=\QQ(\sqrt{2}), E^{H_3}=\QQ(\sqrt{6}).$$
    By the lemmas above, these three fields (along with $F$ and $E$) are the only intermediate fields between $F$ and $E$.
\end{example}

We now return to the question of when an intermediate field is a Galois
extension of $F$.

\begin{lemma}
    Let $E/F$ be a finite Galois extension, and let $F \subset  K \subset  E$. Then $K/F$ is a Galois extension if and only if $\Gal(E/K) \subset  \Gal(E/F)$ is a normal subgroup. Furthermore, in the case that $K/F$ is a Galois extension, we have $$\Gal(K/F)\cong \Gal(E/F)/\Gal(E/K).$$
\end{lemma}

Before beginning the proof, we note that since $E/F$ is separable, it is automatic that $K/F$ is separable. Thus the lemma says that $K/F$ is normal if and only if $\Gal(E/K) \subset  \Gal(E/F)$ is normal.

\begin{proof}
    First we suppose that $K/F$ is a Galois (indeed, normal) extension of $E$, and define a homomorphism $\Gal(E/F) \to  \Gal(K/F)$. Let $\sigma\in \Gal(E/F)$, and let $\aa\in K$ have minimal polynomial $f(x)\in F[x]$ over $F$.

    Since $f(\sigma(\bb)) = \sigma(f(\bb))$ for all $\bb\in E$, we see that $\sigma(\aa)$ must be a root of $f(x)$. But $f(x)$ is an irreducible polynomial over $F$ with a root in $K$ and so, by the definition of a normal extension, all of the roots of $f(x)$ are in $K$. Thus $\sigma(\aa)\in K$. So $\sigma|_K$, the restriction of $\sigma$ to $K$, is an injective homomorphism whose image is contained in $K$. But it must also be a surjective map, since $\sigma^{-1}|_K$ has the same property. Thus $\sigma|_K$ is an automorphism of $K$ for any $\sigma\in \Gal(E/F)$, and clearly this automorphism of $K$ fixes $F$. This shows that $\sigma \mapsto \sigma|_K$ is a well-defined map $\Gal(E/F) \to  \Gal(K/F)$. It is easy to check that $(\sigma \circ \tau )|_K = \sigma|_K \circ \tau|_K$, so this map is a homomorphism. It also must be surjective, since any automorphism $\tau\in \Gal(K/F)$ is an embedding $K \to  F$ fixing $F$, which extends to an embedding $E \to  F$ fixing $F$ by Lemma \ref{extensionlemma}, and this extension will be an element of $\Gal(E/F)$ by Theorem \ref{normal}. Thus we have a surjective homomorphism $\phi: \Gal(E/F) \to  \Gal(K/F)$, and by the first isomorphism theorem,
    $$ \Gal(E/F)/\ker(\phi)\cong \Gal(K/F).$$

    It remains to show that $\ker(\phi) = \Gal(E/K)$, which will also show that the latter subgroup is normal. But $\sigma\in \ker(\phi)$ if and only if $\sigma|_K$ is trivial, which happens if and only if $\sigma(x) = x$ for all $x\in K$. Thus $\sigma\in \ker(\phi)$ if and only if $\sigma\in \Gal(E/K)$. This completes one direction of the proof, and shows that in this case $\Gal(K/F)$ is the expected quotient.

    The other direction, that $\Gal(E/K)$ being a normal subgroup of $\Gal(E/F)$ is sufficient for $K/F$ to be a normal extension, is left as an exercise.
\end{proof}

So, in total, we've proven the following theorem.

\begin{thm}[The fundamental theorem of Galois Theory]\label{galois}
    Let $E/F$ be a Galois extension. Then the function $H \mapsto E^H$ from subgroups of $\Gal(E/F)$ to intermediate fields of $E/F$ is an inclusion-reversing bijection, and the inverse is given by $K \mapsto \Gal(E/K)$. Moreover, $H \trianglelefteq \Gal(E/F)$ if and only if $K = E^H$ is a Galois extension of $F$, in which case $$\Gal(E/F)/\Gal(E/K) \cong \Gal(K/F).$$
\end{thm}

\begin{cor}
    Every finite separable field extension $L/K$ admits only finitely many intermediate fields.
\end{cor}

\begin{proof}
    Pass to the normal closure $L'/K$. Then it is Galois and hence admits only finitely many intermediate fields. 
\end{proof}

\begin{example}
    In the example $E=\QQ(\sqrt{2},\sqrt{3})$, as an extension of $\QQ$, the Galois group is $C_2\times C_2$. Three non-trivial proper subgroups are all normal. The corresponding intermediate fields are all Galois extensions of $\QQ$.
\end{example}

\begin{example}
    Let $E/\QQ$ be the splitting field of $x^3 - 2$, so that $E = \QQ(\aa; \omega)$, where $\aa^3 = 2$ (it does not matter, but we can specify $\aa\in \RR$), and $\omega$ is a primitive cube root of unity. As a splitting field over a perfect field, this is automatically a Galois extension, and since $[E :\QQ] = 6$, we see that the Galois automorphisms are exactly the functions $\sigma_{a,b}$ defined by $$\sigma_{a,b}(\aa)=\omega^a \aa, \sigma_{a,b}(\omega)=\omega^{2^b}.$$
    Note that $\omega^3=1$ and $\omega^{2^2}=\omega$, we can take $a\in\ZZ/3\ZZ$ and $b\in\ZZ/2\ZZ$. And then we can compute the group operation explicitly: $$\sigma_{c,d}\circ\sigma_{a,b}(\omega)=\sigma_{c,d}(\omega^{2^b}) = (\sigma^{2^d})^{2^b} = \sigma^{2^{b+d}}$$
    and $$\sigma_{c,d}\circ\sigma_{a,b}(\aa)= \sigma_{c,d}(\omega^a \aa) = (\omega^{2^d})^a \omega^c\aa.$$
    In other words, $$\sigma_{c,d}\circ\sigma_{a,b}=\sigma_{c+2^da,b+d}.$$
    This gives an explicit isomorphism between $\Gal(E/\QQ)$ and a certain semidirect product of $\ZZ/3\ZZ$ with $\ZZ/2\ZZ$. Indeed, since the group is not abelian, we have $\Gal(E/\QQ)\cong S_3$.
    
    Now the subgroups of $\Gal(E/F)$ (other than the group itself, and the trivial subgroup) have order either 2 or 3, and so must be cyclic. We simply need to identify the elements of $\Gal(E/F)$ of order 2 or 3. One checks that $\sigma_{a,0}\circ\sigma{b,0} = \sigma_{a+b,0}$, and so the elements $\sigma_{1,0}$ and $\sigma_{2,0}$ generate the same subgroup of order 3. Also note that $\sigma_{0,1}^2 = \sigma_{1,1}^2 = \sigma_{2,1}^2 = id$, so the remaining three elements of this group of order 6 each have order 2. Now, $E^{\langle \sigma_{1,0}\rangle} \subset E$ should satisfy $[E : E^{\langle \sigma_{1,0}\rangle}] = 3$. Note that $\sigma_{1,0}(\omega) = \omega$, and $[E : \QQ(\omega)] = 3$, so we must have $E^{\langle \sigma_{1,0}\rangle} = \QQ(\omega)$. Similarly, $[E : E^{\langle \sigma_{0,1}\rangle}] = 2$, and $\sigma_{0,1}(\aa) = \aa$, so we must have $E^{\langle \sigma_{0,1}\rangle} = \QQ(\aa)$. Finally, a bit of trial-and-error shows that $\sigma_{1,1}(\omega^2\aa) = \omega^2\aa$, and $\sigma_{2,1}(\omega\aa) = \omega\aa$, and so comparing degrees gives $E^{\langle \sigma_{1,1}\rangle} = \QQ(\omega^2\aa)$ and $E^{\langle \sigma_{2,1}\rangle} = \QQ(\omega\aa)$.

    The one thing that remains is to decide which of these is a Galois extension of $\QQ$. Let's cheat here: the only non-trivial proper normal subgroup of $S_3$ is $A_3$, so as we already see that $\QQ(\omega)$ (being quadratic) is, while the other three are not.
\end{example}

\chapter*{Week 5}
\setcounter{chapter}{5}

We have already seen one broad class of Galois extensions, namely the quadratic extensions (of a field of characteristic other than 2). In this part of the notes, we'll see some more families of examples.

\section{Cyclotomic extensions}
Cyclotomic extensions are extensions generated by roots of unity. Recall that we have Equation \ref{cycoeq1}.

\begin{defn}
    If $d$ is a positive integer, then the $d$-th cyclotomic polynomial is defined by $$\Phi_d(x) =\prod (x - \zeta),$$ where $\zeta$ ranges over all the primitive $d$-th roots of unity.
\end{defn}

\begin{thm}
    Let $n$ be a positive integer and regard $x^n - 1 \in  \ZZ[x]$. Then
    \begin{enumerate}
        \item $$x^n-1 = \prod_{d|n} \Phi_d(x),$$ where $d$ ranges over all the positive divisors $d$ of $n$ (in particular, $\Phi_1(x) = x - 1$ and $\Phi_n(x)$ occur).
        \item $\Phi_n(x)$ is a monic polynomial in $\ZZ[x]$ and $\deg(\Phi_n) = \phi(n)$, the Euler $\phi$-function.
        \item For every integer $b\geq 1$, we have $n=\sum_{d|n} \phi(d)$.
        \item Each $\Phi_n(x)$ is irreducible in $\QQ[x]$.
    \end{enumerate}
\end{thm}

\begin{proof}
    Actually, part (3) is a classical result in number theory.
    \begin{enumerate}
        \item For each divisor $d$ of $n$, collect all terms in the equation $x^n- 1 = \prod(x- \zeta)$ with $\zeta$ a primitive $d$-th root of unity, since every $n$-th root of unity is a primitive $d$-th root of unity for some divisor $d$ of $n$.
        \item We prove that $\Phi_n(x) \in  \ZZ[x]$ by induction on $n \geq  1$. The base step is true, for $\Phi_1(x) = x- 1 \in  \ZZ[x]$. For the inductive step, let $f(x) =\prod_{d|n,d<n} \Phi_d(x)$, so that $$x^n - 1 = f(x)\Phi_n(x).$$
        By induction, each $\Phi_d(x)$ is a monic polynomial in $\ZZ[x]$, and so $f$ is a monic polynomial in $\ZZ[x]$. Now $f$ divides $x^n-1$ in $\QQ(\zeta_n)[x]$, where $\zeta_n$ is a primitive $n$-th root, hence also divides it in $\QQ[x]$ by the division algorithm. Gauss' Lemma that the quotient $\Phi_n(x)=(x^n- 1)/f(x)$ is a monic polynomial in $\ZZ[x]$ indeed. 
        \item Immediately from (1) and (2):
        $$n=\deg(x^n-1)=\deg(\prod_{d|n}\Phi_n(x))=\sum_{d|n}\deg(\Phi_n(x))=\sum_{d|n} \phi(d).$$
        \item The case when $n=p$ is a prime is easy to prove by Eisenstein's Criterion. The general case is highly non-trivial.
        Suppose that $\Phi_n(x) = f(x)g(x)$ in $\QQ[x]$ with $f$ of positive degree. With Gauss' lemma, we can suppose that both $f$ and $g$ are monic and are in $\ZZ[x]$. Let $x - \zeta$ be a linear factor of $f(x)$ in $k[x]$ for an extension field $k$ of $\QQ$. We wish to show that $x - \zeta^a$ is also a linear factor of $f$ for every $a \in (\ZZ/n\ZZ)^*$, and thus that $$ \deg f = \phi(n) = \deg (\Phi_n)$$ concluding that $f = \Phi_n$.

        Since each $a \in (\ZZ/n\ZZ)^*$ is a product of primes $p$ not dividing $n$, it suffices to show that $x - \zeta^p$ is a linear factor of $f(x)$ for all primes $p$ not dividing $n$, and the iterating the results $x-\zeta^a$ is also a root. If not, then $x - \zeta^p$ is necessarily a linear factor of $g(x)$, by unique factorization in $k[x]$. That is, $\zeta$ is a root of $g(x^p) = 0$ in $k[x]$, so $x - \zeta$ divides $g(x^p)$ in $k[x]$.
        Thus, in $\QQ[x]$ the $\gcd$ of $f(x)$ and $g'(x)=g(x^p)$ is not 1 --- they are both divisible by the minimal polynomial of $\zeta$ in $\QQ[x]$. Let's say 
        $$f(x)=a(x)d(x) \mbox{ and } g(x^p)=b(x)d(x),$$ where $a(x),b(x),d(x)$ are all monic in $\ZZ[x]$. 

        Mapping everything to $\FF_p[x]$, where $g(x^p) = g(x)^p$, we have $g(x)^p= b(x)d(x)$ in $\FF_p[x]$. Let $\delta(x) \in \FF_p[x]$ be an irreducible dividing $d(x)$ in $\FF_p[x]$. Then since $\delta(x)$ divides $g(x)^p$ in $\FF_p[x]$ it divides $g(x)$. Also $\delta(x)$ divides $f(x)$ in $\FF_p[x]$, so $\delta(x)^2$ apparently divides $\Phi_n(x) = f(x)g(x)$ in $\FF_p[x]$. But $p$ does not divide $n$, $\gcd(x^n-1,nx^{-1}-1) = 1$ and so $x^n-1$ and thus $\Phi_n(x)$ has no repeated root in $\FF_p[x]$, contradiction. Thus, it could not have been that $\Phi_n(x)$ factored properly in $\QQ[x]$.
    \end{enumerate}
\end{proof}

We can calculate the cyclotomic polynomials based on this theorem without going into complex analysis. Here is a list of first few ones:
\begin{align*}
    \Phi_1(x) &= x-1 \\
    \Phi_2(x) &= x+1 \\
    \Phi_3(x) &= x^2+x+1 \\
    \Phi_4(x) &= x^2+1 \\
    \Phi_5(x) &= x^4+x^3+x^2+x+1 \\
    \Phi_6(x) &= x^2-x+1 \\
    \Phi_7(x) &= x^6+x^5+x^4+x^3+x^2x+1 \\
    \Phi_8(x) &= x^4+1 \\
    \cdots
\end{align*}

\begin{thm}
    Let $N \geq 1$, and let $\zeta$ be a root of $\Phi_N$. Then $\QQ(\zeta)/\QQ$ is a Galois extension of degree $\phi(N)$, with
    $$\Gal(\QQ(\zeta)/\QQ) \cong (\ZZ/N\ZZ)^\times.$$
\end{thm}

An extension of the above sort is called a \emph{cyclotomic extension}.

\begin{proof}
    Since $\Phi_N(x)$ is irreducible over $\QQ$, we have $$[E : Q] = \deg(\Phi_N) = \phi(N).$$
    Also, note that if $\Phi_N(\zeta) = 0$, then the roots of $\Phi_N(x)$ in $\CC$ are precisely the numbers of the form $\zeta^a$ for $\gcd(a,N) = 1$, all of which are in $E = \QQ(\zeta)$. So $E/\QQ$ is a normal extension, and hence Galois. Now, any $\sigma \in \Gal(E/\QQ)$ is entirely determined by what it does to $\zeta$, and $\sigma(\zeta)$ must be $\zeta^a$ for some $\gcd(a,N) = 1$. Since there are precisely $\phi(N)$ such values of $a (\mod N)$, and since $|\Gal(E/\QQ) = \phi(N)$, we must have an automorphism $\sigma_a \in \Gal(E/\QQ)$ defined by $\sigma_a(\zeta) = \zeta^a$ for each $a \in (\ZZ/N\ZZ)^\times$. But the group operation is $$\sigma_a \circ \sigma_b = \sigma_{ab},$$
    and so the bijection $a \mapsto \sigma_a$ is actually an isomorphism $(\ZZ/N\ZZ)^\times \to \Gal(E/\QQ)$.
\end{proof}


Note that $\phi(N)$ is even for $N \geq 3$, which you can see either by developing the formula for $\phi(N)$ in terms of the prime factorization of $N$:
$$\phi(p_1^{e_1}\cdots p_n^{e_n}) = (p_1-1)p_1^{e_1-1}\cdots (p_n-1) p_n^{e_n-1},$$
or by noting the non-triviality of the automorphism $x \mapsto -x$ of the unit group in $\ZZ/N\ZZ$. In particular, there is an element of order 2 in $\Gal(E/\QQ)$, and hence a field of which $E$ is a quadratic extension. It turns out to be relatively easy to describe this field. The automorphism $x \mapsto -x$ of $(\ZZ/N\ZZ)^\times$ corresponds to the element $\sigma_{-1}\in \Gal(E/\QQ)$. This element is exactly the complex conjugation by observing $$e^{-2k\pi i/N} =\overline{e^{2k\pi i /N}}.$$
In other words, the fixed field of $\sigma_{-1}$ is exactly $K = E\cap \RR$, and $E$ is a quadratic extension of this. We can actually describe this field quite nicely. Clearly $\zeta + \zeta^{-1} \in K$, since it is fixed by $\zeta \mapsto \zeta^{-1}$, and so $\QQ(\zeta + \zeta^{-1}) \subset K$. We could use an argument about symmetric polynomials to conclude that everything fixed by $\sigma_{-1}$ is actually a polynomial in $\zeta + \zeta^{-1}$, but there is a more direct proof that in fact $K = \QQ(\zeta + \zeta^{-1})$. To see this, note that
$$\zeta^2 - (\zeta + \zeta^{-1})\zeta + 1 = 0,$$
and so $\zeta$ satisfies a quadratic polynomial over $\QQ(\zeta + \zeta^{-1})$. Thus, $$\QQ(\zeta + \zeta^{-1}) \subset K \subset E,$$
with $[E : K] = 2$, and $[E : \QQ(\zeta + \zeta^{-1})] \leq 2$. It follows that $K = \QQ(\zeta + \zeta^{-1})$.

Perhaps the most amazing fact about cyclotomic extensions is that, at least over $\QQ$, they completely explain abelian extensions.

\begin{thm}[Kronecker-Weber]
    Let $E/\QQ$ be a finite Galois extension such that $\Gal(E/\QQ)$ is abelian. Then there is a root of unity $\zeta$ such that $E \subset \QQ(\zeta)$.
\end{thm}

This theorem is one of the earliest known results in class field theory. We shall prove it. But the proof requires sophisticated tools like ramification theory and so let's push on at this stage.

\section{Cubic extensions}

We've seen that, at least if $\char(F) \not= 2$, every extension of degree 2 is Galois. We know that this is not true for extensions of degree 3, since $\QQ(\sqrt[3]{2})/\QQ$ is not a Galois extension. Rather than address the problem of which degree-three extensions of a field are Galois, we'll instead ask what the splitting field of a cubic polynomial looks like (which ultimately answers the other question).


Let $F$ be a field with $\char(F) \not= 2$ or 3, let $f(x) = x^3 + ax + b$ be an irreducible cubic polynomial. (We may assume $F$ to be perfect. But we will see it is really not necessary. We will see more about separable extensions later.) Note we can always complete the cube to write $f(x)$ in this way by replacing $x$ with $x-p/3$ and we do need $\char(F)\not=3$. Let $f(\aa) = 0$. If $E/F$ is the splitting field of $f(x)$ over $F$, then since $f(x)$ has a linear factor over $F(\aa)$, we have $[E : F(\aa)] \leq 2$, and so $[E : F] = 3$ or 6. $f(x)$ factors into three linear terms or a product of a linear term and a quadratic in $F(\aa)[x]$. Since $\char(F) \not= 2$, the quadratic is separable and so is $f(x)$. Note that $\Gal(E/F)$ is a subgroup of $S_3$, the symmetric group on the three roots of $f(x)$, and so the question of whether or not $E = F(\aa)$ comes down to whether or not $\Gal(E/F)$ contains an element of order 2; if it does, it is the full group $S_3$, and if it doesn't it's the alternating group $A_3$.


Let $f(x) = (x - \aa_1)(x - \aa_2)(x - \aa_3)$ in $E[x]$, and let
$$\delta = (\aa_1 - \aa_2)(\aa_2 - \aa_3)(\aa_1 - \aa_3).$$

Then clearly $\sigma(\delta) = \pm\delta$ for each $\sigma \in \Gal(E/F)$. It is easy to check that an arbitrary transposition of two elements in $\{\aa_1, \aa_2, \aa_3\}$ changes the sign of $\delta$, and so for $\sigma \in \Gal(E/F) \subset S_3$ a permutation of the $\aa_i$, we have
$$\sigma(\delta) = (-1)^{\sgn(\sigma)} \delta.$$

In other words, if $\Gal(E/F) =A_3$, then $\sigma(\delta) = \delta$ for all $\sigma\in \Gal(E/F)$, and so $\delta\in F$. If $\Gal(E/F) = S_3$, then $\Gal(E/F)$ contains an odd element which doesn't fix $\delta$, and hence $\delta \notin F$. But $\delta^2\in F$, since $\sigma(\delta^2) = (\pm\delta)^2 =\delta^2$ for all $\sigma\in \Gal(E/F)$. Note that, in the case that $\Gal(E/F) = S_3$, so that $\delta \notin F$, there must be a quadratic extension of $F$ corresponding to the subgroup $A_3 \subset \Gal(E/F)$. But clearly $\delta\in E$, and $\delta^2\in F$, so $F(\delta)$ is precisely this quadratic extension. A simple calculation shows that $$\Delta =\delta^2 =  -(4a^2+27b^2).$$

\begin{thm}
    Let $F$ be a field with $\char(F) \not= 2$ or 3, let $f(x) = x^3+ax+b\in F[x]$  be irreducible, let $E$ be the splitting field of $f(x)$ over $F$. Then if $\Delta = -(4a^2+27b^2)$ is the square of an element in $F$, then $\Gal(E/F) \cong A_3$. Otherwise, if $\Delta$ is not a square, then $\Gal(E/F) \cong S_3$.
\end{thm}

\begin{example}
    We can check that $x^3 - x - 1$ has no roots in $\QQ$, and hence (being cubic) is irreducible. We can also calculate the discriminant:
    $$\Delta(x^3 - x - 1) = -(4(-1)3 + 27(-1)2) = -23.$$
    This is not a square in $\QQ$, and so if $E$ is the splitting field of $x^3 - x - 1$, we have $\Gal(E/\QQ) \cong S_3$. The unique quadratic extension corresponding to $A_3 \subset S_3$ is $\QQ(\sqrt{-23})$.
\end{example}

\begin{example}
    We can check that $x^3 - 3x + 1$ is irreducible over Q, and
    $$\Delta(x^3 - 3x + 1) = 9^2.$$
    Thus if $\aa_3 - 2\aa + 1 = 0$, the extension $\QQ(\aa)=\QQ$ is Galois.
\end{example}

\begin{example}
    The polynomial $x^3 -x-1$ has no roots in $\FF_{73}$, and so (being cubic) is irreducible. We know that $\Delta(x^3 - x - 1) = -23$, but
    $$14^2 + 23 = 3 \cdot 73,$$ and so $-23$ is a perfect square in $\FF_{73}$. In other words, if $E/\FF_{73}$ is the splitting field of $x^3 - x - 1$, then $\Gal(E/F) = A_3$. Note that, since $[E : \FF_{73}] = 3$, we must have $E = \FF_{73^3}$.
\end{example}


\chapter*{Week 6}
\setcounter{chapter}{6}

% \section{Kummer extensions}
% Another relatively common type of field extension is an extension by an $n$-th root. It is much easier to study these fields if we start with the assumption that the ground field contains the $n$-th roots of unity. Of course we must be careful, since a field obtained by adjoining the sixth root of a perfect square is really just a cubic extension. Recall Theorem \ref{kummer1}, let's study its Galois group. 

% \begin{thm}
%     Let $F$ be a field of characteristic not dividing $n$, containing a primitive $n$-th root of unity, and let $E$ be the splitting field of $x^n-c\in F[x]$. In other words $E =F(\aa)$, where $\aa^n=c \in F$ and $\aa\in E$. Then $E/F$ is Galois and and $\Gal(E/F) \cong \ZZ/r\ZZ$.
% \end{thm}

% \begin{proof}
%     We already showed that $E/F$ is normal in Theorem \ref{kummer1}. Now, since $\char(F) \nmid n$, we also have $\char(F) \nmid r$, and so $E/F$ is separable hence Galois.

%     Note any $\sigma\in \Gal(E/F)$ is entirely determined by what it does to $\aa^{1/n}$ and $\sigma(\aa)$ must be $\zeta^a\aa$. We get an bijection, indeed, an isomorphism between $\Gal(E/F)$ and $\ZZ/r\ZZ$
% \end{proof}

\section{Characters}

In this section, we will discuss some methods from linear algebra that are of special interest for applications in Galois theory, in particular for the  study of cyclic extensions in next week.

\begin{defn}
    Let $G$ be a group and $K$ a field. A $K$-valued character of $G$ is a group homomorphism $\chi: G \to K^*$.
\end{defn}

For a group $G$ and a field $K$, there exists always the trivial character $G \to K^*$, mapping every element $g \in G$ to the unit element $1 \in K^*$. Furthermore, the $K$-valued characters of $G$ form a group, whose law of composition is induced from the multiplication on $K^*$. Indeed, the product of two characters $\chi_1, \chi_2 : G \to K^*$ is given by $$\chi_1 \cdot \chi_2: G\to K^*, g\mapsto \chi_1(g)\chi_2(g).$$

\begin{thm}[Dedekind]
    Distinct characters $\chi_1, \cdots, \chi_m$ on a group $G$ with values in a field $K$ are $K$-linearly independent.
\end{thm}

\begin{proof}
    We proceed indirectly and assume that the assertion of the proposition is false. Then there is a minimal number $n \in N$ such that there exists a linearly dependent system of $K$-valued characters $\chi_1, \cdots , \chi_n$ on $G$. Of course, we must have $n \geq 2$, since every character assumes values in $K^*$ and therefore cannot coincide with the zero map. Now we can get a linear equation
    \begin{equation}\label{chareq1}
        a_1\chi_1+\cdots + a_n\chi_n = 0,
    \end{equation}
    where all $a_i$ from $K$ are non-zero due to the minimality of $n$. Let $g,h\in G$ and  we evaluate Equation \ref{chareq1} at $gh$ an get 
    \begin{equation*}
        a_1\chi_1(gh)+\cdots + a_n\chi_n(gh) = 0.
    \end{equation*}
    We choose $g$ in the way such that $\chi_1(g)\not=\chi_2(g)$. By run $h$ through all elements in $G$, we get a new relation 
    \begin{equation}\label{chareq2}
        a_1\chi_1(g)\chi_1+\cdots + a_n\chi_n(g)\chi_n = 0.
    \end{equation}
    Now multiplying Equation \ref{chareq1} by $\chi_1(g)$ and subtracting Equation \ref{chareq2}, we get a third relation   
    \begin{equation*}
        a_2(\chi_1(g)-\chi_2(g))\chi_2 +\cdots + a_n(\chi_1(g)-\chi_n(g))\chi_n = 0.
    \end{equation*}
    This is a nontrivial relation of length $n-1$, since $a_2(\chi_1(g)-\chi2(g)) \not= 0$. However, this contradicts the minimality of $n$, and it follows that the assertion of the proposition is true.
\end{proof}


The preceding proposition can be applied in various settings. For example, if $L/K$ is an algebraic field extension, we see that the system $\Aut(L/K)$ of all $K$-automorphisms of $L$ is linearly independent in the $L$-vector space of all maps $L \to L$. Indeed, we have:

\begin{cor}\label{charcor}
    Let $L/K$ be a finite separable field extension and $x_1, \cdots , x_n$ a basis of $L$ as a $K$-vector space. Furthermore, let $\sigma_1, \cdots , \sigma_n$ denote the $K$-homomorphisms of $L$ to an algebraic closure $\overline{K}$ of $K$. Then the vectors
    \begin{align*}
        \xi_1 &= (\sigma_1(x_1),\cdots,\sigma_1(x_n)),\\
        &\cdots,\\
        \xi_n &= (\sigma_n(x_1),\cdots,\sigma_n(x_n)),
    \end{align*}
    give rise to a system that is linearly independent over $\overline{K}$. Therefore, they form an $L$-basis of $L^n$ if $L/K$ is Galois.
\end{cor}

\begin{proof}
    The linear dependence of the $\xi_i$ would imply the linear dependence of the $\sigma_i$. However, by preceding theorem, the $\sigma_i$ form a linearly independent system.
\end{proof}

\begin{example}
    An $\RR$-basis of $\CC$ is $\{1,i\}$. The corollary says $\{1+i, 1-i\}$ is a $\CC$-basis of $\CC^2$.
\end{example}


\begin{defn}
    Let $L/K$ be a finite Galois extension of degree $n$. A $K$-\emph{conjugate} of an element $\aa\in L$ is $\sigma(\aa)$, where $\sigma$ is an element in $\Gal(L/K)$. A \emph{normal basis} of $L/K$ is a basis of $L$ consisting of $K$-conjugates only, namely,  $\{\sigma(\aa):\sigma\in \Gal(L/K)\}$ for some $\aa\in L$.
\end{defn}

\begin{example}
    The usual basis $\{1,i\}$ is not a normal basis since the terms are not related by an element in Galois group. $\{i,-i\}$ is a set of $\RR$-conjugate, but it is linearly dependent. The set $\{1+i,1-i\}$ is a normal basis.
\end{example}

\begin{example}
    Consider the field extension $\QQ(\sqrt{2}+\sqrt{3})/\QQ$. The set of four conjugates of $\sqrt{2}+\sqrt{3}$ is not a normal basis, since it is not linearly independent. If $\aa=1+\sqrt{2}+\sqrt{3}+\sqrt{6}$, then the four conjugates of $\aa$ are
    $$1+\sqrt{2}+\sqrt{3}+\sqrt{6},1-\sqrt{2}+\sqrt{3}-\sqrt{6},1+\sqrt{2}-\sqrt{3}-\sqrt{6},1-\sqrt{2}-\sqrt{3}+\sqrt{6}.$$ They are linearly independent by checking the determinate and spans $\QQ(\sqrt{2}+\sqrt{3})$. Therefore, they form a normal basis. 
\end{example}

The normal basis theorem says that every finite Galois extension admits a normal basis. However, we only give a proof of this theorem when $K$ is infinite.

\begin{thm}[Normal Basis Theorem]
    Every finite Galois extension of an infinite field admits a normal basis.
\end{thm}

\begin{proof}
    Let $L/K$ be a Galois extension and $[L:K]=n$ and $\Gal(L/K)=\{\sigma_1,\cdots,\sigma_n\}$. We aim to find an element $\aa\in L$ such that $\sigma_1(\aa),\cdots,\sigma_n(\aa)$ are linearly independent.
    If we have a linear dependence relation for some $\aa\in L$: $$\sum_{j=1}^n a_n\sigma_j(\aa) = 0,$$ for some $a_1,\cdots,a_j\in K$, then we can get a new one by applying $\sigma_i^{-1}$: $$\sum_{j=1}^n a_n(\sigma_i^{-1}\sigma_j)(\aa) = 0.$$
    Collecting these together for $i=1,\cdots,n$, we get $$\begin{pmatrix}
        \sigma_1^{-1}\sigma_1(\aa) &\cdots& \sigma_1^{-1}\sigma_n(\aa)\\
        \vdots &\ddots&\vdots\\
        \sigma_n^{-1}\sigma_1(\aa) &\cdots & \sigma_n^{-1}\sigma_n(\aa)
    \end{pmatrix} \begin{pmatrix}
        a_1 \\
        \vdots\\
        a_n
    \end{pmatrix} = \begin{pmatrix}
        0\\
        \vdots\\
        0
    \end{pmatrix}.$$
    For force a trivial solution for $a_1,\cdot,a_n$, we must have $\det((\sigma_i^{-1}\sigma_j)(\aa)) \not= 0.$
    Let $\{e_1,\cdots,e_n\}$ be a $K$-basis of $L$. Then an arbitrary element $\aa\in L$ is of the form $\sum_{k=1}^n b_k e_k$ with $b_k\in K$. And then we have $$(\sigma_i^{-1}\sigma_j)(\aa)=\sum_{k=1}^n b_k((\sigma_i^{-1}\sigma_j)(e_k)).$$
    Consider the polynomial $$\Delta(X_1,\cdots, X_n) = \det(\sum_{k=1}^n X_k((\sigma_i^{-1}\sigma_j)(e_k)))\in L[X_1,\cdots,X_n].$$
    Let $\sigma_1$ be the identity automorphism. Corollary \ref{charcor} says there are $c_1,\cdots, c_n\in L$ such that    $$\sum_{k=1}^nc_k(\sigma_1(e_k),\cdots,\sigma_n(e_k))=(1,0,\cdots,0).$$ Reading this off component-wise, $$\sum_{k=1}^n c_ke_k=1, \quad \sum_{k=1}^n c_k \sigma(e_k)=0 \mbox{ for }\sigma\not= id.$$
    Thus $$ \sum_{k=1}^n c_k (\sigma_i^{-1} \sigma_j)(e_k)=1$$ if $i=j$ and $=0$ otherwise. Therefore the matrix $(\sum_{k=1}^n c_k (\sigma_i^{-1} \sigma_j)(e_k))$ is the identity matrix and so $\Delta(c_1,\cdots,c_n)  =1 $, which means $\Delta(X_1,\cdots,X_n)$ is not the zero polynomial.

    Since $K$ is infinite, there must be $b_1,\cdots, b_n\in K$ such that $\Delta(b_1,\cdots,b_n)\not=0$. Then $\aa=\sum_{k=1}^n b_k e_k$ is the desired element.
\end{proof}

\section{Norm and trace}

\begin{defn}
    Let $L/K$ be a finite field extension. For an element $a \in L$, consider the multiplication map $\phi_a : L \to L, x \mapsto ax$, as a linear transformation of $L$ as a $K$-vector space. Then $$\Tr_{L/K}(a)= \Tr(\phi_a),\quad \nm_{L/K}(a) = \det(\phi_a)$$ are called the trace and the norm of $a$ with respect to the extension $L/K$.
\end{defn}

In particular, $\Tr_{L/K} : L \to K$ is a homomorphism of $K$-vector spaces, or in more precise terms, a linear functional on $L$ viewed as a $K$-vector space. Likewise, $\nm_{L/K} : L^* \to K^*$ is a group homomorphism.

\begin{example}
    If $z = x+yi$ is the decomposition of a complex number $z$ into its real and imaginary parts, then the multiplication by $z$ on $\CC$ is described relative to the $\RR$-basis $1, i$ by the matrix $$    \begin{pmatrix}
      x & -y \\
      y &  x
    \end{pmatrix}.$$
    And then $\Tr_{\CC/\RR}(z)= 2\Re(z)$ and $\nm_{\CC/\RR} = z\overline{z}$.
\end{example}

\begin{example}
    Consider the field extension $\QQ(\sqrt{2})/\QQ$. For $\aa=a+b\sqrt{2}$, the action $\phi_\aa$ on a $\QQ$-basis $\{1,\sqrt{2}\}$ of $\QQ(\sqrt{2})$ is given by $$\phi_\aa(1)=a+b\sqrt{2},\quad \phi_\aa(\sqrt{2})=2b+a\sqrt{2}.$$ The matrix represetation of $\phi_\aa$ is then $$\begin{pmatrix}
        a & 2b\\
        b & a
    \end{pmatrix},$$=and then $\Tr(\aa)=2a, \nm(\aa)=a^2-2b^2.$
\end{example}

\begin{example}
    Here we use trace and norm to show $1+\sqrt[3]{2}$ is not a perfect square in $F=\QQ(\sqrt[3]{2})$. Let $\{1,\sqrt[3]{2},\sqrt[3]{4}\}$ be the $\QQ$-basis of $F$. Then the action $\phi_{1+\sqrt[3]{2}}$ has the matrix representation $$\begin{pmatrix}
        1 & 0 & 2\\
        1 & 1 & 0\\
        0 & 1& 1
    \end{pmatrix}.$$
    Hence $\nm_{F/\QQ}(1+\sqrt[3]{2}) =  3$. If $1+\sqrt[3]{2}$ is a perfect square, then so is its norm. But 3 is not a square of any rational number.  
\end{example}

\begin{lemma}
    Let $L/K$ be a finite field extension of degree $n = [L : K]$, and consider an element $a \in L$.
\begin{enumerate}
    \item If $a \in K$, then
    $$\Tr_{L/K}(a) = na,\quad \nm_{L/K}(a) = a^n.$$
    \item If $L = K(a)$ and $X^n + c_{n-1}X^{n-1} +\cdots + c_0$ is the minimal polynomial of $a$ over $K$, then
    $$\Tr_{L/K}=-c_{n-1},\quad \nm_{L/K}(a)=(-1)^nc_0.$$
\end{enumerate}
\end{lemma}

\begin{proof}
    For $a \in K$= the linear map $\phi_a : L \to L$ is described by $a$ times the unit matrix of $K_{n\times n}$. This justifies the formulas in (i). 
    
    
    Furthermore, if $L = K(a)$, the minimal polynomial of a coincides with the minimal polynomial of the endomorphism $\phi_a$, and hence by reasons of degree, must coincide with the characteristic polynomial of $\phi_a$. Therefore, the formulas in (ii) follow from the description of $\Tr(\phi_a)$ and $\det(\phi_a)$ in terms of the coefficients of the characteristic polynomial of $\phi_a$.
\end{proof}

% \begin{lemma}
%     Let $E/F$ be a finite extension, and let $f$ be the minimal polynomial of $\aa\in E$. Let $\chi(x)$ be the characteristic polynomial of $\phi_a: E\to E, x\mapsto ax$. Then $$\chi(x)=f(x)^{[E:F(\aa)]}.$$
% \end{lemma}

Now we can extend the lemma above to compute the norm and the trace of elements when one is dealing with arbitrary field extensions.

\begin{lemma}
    Consider an element $a \in L$ of a finite field extension $L/K$, and let $s = [L : K(a)]$. Then $$\Tr_{L/K}(a)=s\Tr_{K(a)/K}(a),\quad \nm_{L/K}(a)=(\nm_{K(a)/K}(a))^s$$
\end{lemma}

\begin{proof}
    Choose a $K$-basis $x_1, \cdots , x_r$ of $K(a)$, as well as a $K(a)$-basis $y_1, \cdots , y_s$ of $L$. Then the products $x_iy_j$ form a $K$-basis of $L$. Let $A \in K_{r\times r}$ be the matrix describing the multiplication by $a$ on $K(a)$ relative to the basis $x_1, \cdots , x_r$. It follows that, relative to the basis consisting of the $x_iy_j$, the multiplication by $a$ on $L$ is given by the matrix
    $$\begin{pmatrix}
       A &0 &\cdots & 0\\
       0 & A &\cdots & 0\\
      \vdots & \ddots & \cdots  & \vdots\\
       0 &0 &\cdots & A
    \end{pmatrix}.$$
\end{proof}

\begin{thm}
    Let $L/K$ be a finite field extension of degree $[L : K] =n= qr$, where $r = [L : K]_s$ is the separable degree of $L/K$. If $\sigma_1, \cdots , \sigma_r$ are the $K$-homomorphisms of $L$ into an algebraic closure $\overline{K}$ of $K$, the following formulas hold for elements $a \in L$: $$\Tr_{L/K}(a)=q\sum_{i=1}^r \sigma_i(a),\quad \nm_{L/K}(a) = (\prod_{i=1}^r \sigma_i(a))^q. $$ In particular, if $L/K$ is not separable, then $\Tr_{L/K}(a)=0$ for all $a\in L$.
\end{thm}

\begin{proof}
    Let $$g(x)=(\prod_{j=1}^r(x-\sigma_j(a)))^q \in \ok[x]$$. The degree of $g$ is exactly $qr=[L:K]$.

    Claim: $g(x)$ is in $K[x]$ and has precisely the same root as the minimal polynomial $p(x)=\irr(a,K)$ of $a$ over K.
    
    Let's postpone the proof of claim. It follows that $p$ divides $g$ immediately since all roots of $g$ are roots of $p$.  And then the only irreducible factor of $g$ is $p$ and so $g(x)=p(x)^{n/m}$, where $m$ is the degree of $p$. The characteristic polynomial $\chi(x)$ of $\phi_a$ is also divisible by the minimal polynomial of $\phi_a$, which is $p(x)$. By arguing over the degrees of polynomials, we have $\chi(x)=g(x)$. Multiplying $g(x)$ out and looking at the constant term and the coefficient of $x^{n-1}$, we have $$\Tr_{L/K}(a)=q\sum_{i=1}^r \sigma_i(a),\quad \nm_{L/K}(a) = (\prod_{i=1}^r \sigma_i(a))^q. $$

    Now it remains to prove the claim.
    \begin{proof}[Proof of Claim]
        To see that $g,p$ have the same roots, we note that each $\sigma_j(a)$ is a root of $p$ since $\sigma_j$ fixed $K$. If $a'\in \ok$ is a root of $p(x)$, then by the extension theorem, we have a $K$-map $\tau:\ok\to \ok$ with $\tau(a)=a'$. Since $\tau|_K$ is one of $\sigma_j$, say $\tau|_K=\sigma_i$, then $a'=\tau(a)=\sigma_i(a)$. So $a'$ is also a root of $g$. This proves $g,p$ have the same roots.
        
        To see that $g(x)$ is in $K[x]$, we aim to show that $g=p^{n/m}$. We have $[L:K(a)]_s [K(a):K]_s=[L:K]_s$ by Theorem \ref{sepdeg}. In lieu of the proof of Theorem \ref{sepdeg}, we have $$\Hom_K(L,\ok)\cong \Hom_{K(a)}(L,\ok)\times\Hom_K(K(a),\ok).$$ In other words, an element $\sigma$ in $\Hom_K(L,\ok)$ can be unqiuely written as $\sigma = \overline{\rho}\circ\tau,$ where $\tau\in\Hom_K(K(a),\ok)$ and $\rho\in  \Hom_{K(a)}(L,\ok)$ and $\sigma|_{K(a)}=\tau$. Write $\Hom_K(K(a),\ok)=\{\tau_1,\cdots,\tau_s\}$ and then we have  

        \begin{align*}
            p(x) &= (\prod_{i=1}^s(x-\tau_s(a)))^{m/s},\\
            g(x) &= (\prod_{j=1}^r(x-\sigma_j(a)))^q =((\prod_{i=1}^s(x-\tau_i(a)))^{r/s})^q = p(x)^{n/m},
        \end{align*} by counting the multiplicity. Then $g$ is in $K[x]$ as well.
    \end{proof}

    For the "in particular" part, if $a$ is inseparable, then the characteristic of $K$ must be a prime $l$. And then the trace of $a$ is the coefficient of $x^{n-1}$ term in $g(x)=p(x)^{n/m}$, which is a multiply of $l$ and hence is 0. In particular, $\Tr_{L/K}(a)=[L:K] a = 0$ for $a\in K$ since $[L:k]=l^uv$ is zero in a characteristic $l$ field.
\end{proof}

\begin{thm}
    Let $M/L$ and $L/K$ be finite field extensions. Then $$\Tr_{M/K} = \Tr_{L/K}\circ\Tr_{M/L},\quad \nm_{M/K}=\nm_{L/K}\circ\nm_{M/L}.$$
\end{thm}

\begin{proof}
    Note that $$[L:K]=q_1[L:K]_s,\quad [M:L]=q_2[M:L]_s,\quad [M:K]=q_1q_2[M:K]_s.$$ Write $$\Hom_K(L,\overline{K}) = \{\sigma_i: i \in I\},\quad \Hom_L(M,\overline{K}) = \{\tau_j: j \in J\},$$ then $$\Hom_K(M,\overline{K}) = \{\overline{\sigma_i} \circ \tau_j ; i \in I, j \in J\}.$$ And then a straightforward calculation by the formula above shows \begin{align*}
        \Tr_{M/K}(a)&=q_1q_2\sum_{i,j} \overline{\sigma_i}\circ\tau_j(a) \\
        &= q_1 \sum_i\overline{\sigma_i}(q_2\sum_j \tau_j(a)) \\
        &=\Tr_{L/K}(\Tr_{M/L}(a)),
    \end{align*} 
    as well as a similar chain equalities of norm.
\end{proof}

\begin{cor}
    If $K/F$ is Galois with Galois group $G$, then for all $a\in K$ we have $$\Tr_{K/F}(a)=\sum_{\sigma\in G}\sigma(a),\quad \nm_{K/F}=\prod_{\sigma\in G}\sigma(a).$$
\end{cor}

\begin{example}
    Let $F$ be a field with $\char(F)\not=2,3$ and let $d\in F\setminus F^2$. We already see that $F(\sqrt{d})/F$ is Galois and $\Gal(F(\sqrt{d})/F)=\{id,\sigma\}$, where $\sigma(\sqrt{d})=-\sqrt{d}$. Therefore, 
    \begin{align*}
        \Tr_{F(\sqrt{d})/F}(a+b\sqrt{d})&=(a+b\sqrt{d})+(a-b\sqrt{d})=2a,\\
        \nm_{F(\sqrt{d})/F}(a+b\sqrt{d}) &=(a+b\sqrt{d})(a-b\sqrt{d})=a^2-b^2d.
    \end{align*} 
\end{example}

\begin{thm}
    A finite extension $L/K$ is separable if and only if the $K$-linear map $\Tr_{L/K}: L\to K$ is nontrivial and hence surjective. If $L/K$ is separable, the symmetric bilinear map $$\Tr:L\times L\to K,\quad (x,y)\mapsto \Tr_{L/K}(xy)$$ is nondegenerate. In other words, $\Tr$ induces an isomorphism $$L\to \hat{L}, \quad x\mapsto \Tr(x,\cdot)$$ of $L$ into its dual $\hat{L}$.
\end{thm}

\begin{proof}
    We already see that if $L/K$ is separable, then $\Tr_{L/K}$ is the zero map.
    Now assume $L/K$ is separable. If $\sigma_1,\cdots,\sigma_r$ are all the $K$-map $L\to\overline{K}$, where $\ok$ is an algebraic closure of $K$, then we have $$\Tr_{L/K}=\sigma_1+\cdots+\sigma_r.$$ Dedekind's Theorem says characters are linearly independent hence the sum is not the zero map.


    Now consider an element $x$ of the kernel of $L\to \hat{L}$. Then we get $\Tr_{L/K}(xL)=0$ which says $x=0$. Otherwise, $xL=x$ and then $\Tr_{L/K}$ vanished on $L$. Contradiction. And an injection homomorphism between finite dimensional vector spaces is necessarily an isomorphism. 
\end{proof}

\section{Cyclic extensions}

\section{Hilbert Theorem 90}


\chapter*{Week 7}
\setcounter{chapter}{7}

\section{Kummer Theory}

\section{Solvable groups}

\chapter*{Week 8}
\setcounter{chapter}{8}

\section{Solvability by radicals}

\section{The transcendence of $\pi$ and $e$}

\chapter*{Week 9}
\setcounter{chapter}{9}

\section{Profinite groups}

\section{Infinite Galois extensions}


\chapter*{Week 10}
\setcounter{chapter}{10}

\section{Transcendental extensions}


% \newpage
% \begin{thebibliography}{99} % don't worry about the 99

% % \bibitem{ExampleBook}
% % H.~L.~Montgomery and R.~C.~vaughan, {\em Multiplicative Number Theory I: Classical Theory}, Cambridge University Press (2007).

% \end{thebibliography}

\end{document}
