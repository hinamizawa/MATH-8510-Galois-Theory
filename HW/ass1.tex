\documentclass[12pt,a4paper]{article}
\usepackage[utf8]{inputenc}
\usepackage{fullpage}
\usepackage{amsfonts}
\usepackage{amsmath}
\usepackage{amssymb}
\usepackage{times}
\usepackage{graphicx}
\usepackage[margin=1in,footskip=0.25in]{geometry}

\def\AA{\mathfrak{A}}
\def\BB{\mathcal{B}}
\def\OO{\mathcal{O}}
\def\ZZ{\mathbb{Z}}
\def\NN{\mathbb{N}}
\def\CC{\mathbb{C}}
\def\QQ{\mathbb{Q}}
\def\RR{\mathbb{R}}
\def\FF{\mathbb{F}}
\def\im{\text{im}}
\def\Tr{\text{Tr}}
\def\sgn{\text{sgn}}
\def\nm{\text{N}}
\def\tk{\theta_K}
\def\Hom{\text{Hom}}
\def\Aut{\text{Aut}}
\def\Inn{\text{Inn}}
\def\aa{\alpha}
\def\bb{\beta}
\def\aaa{\mathfrak{a}}
\def\bbb{\mathfrak{b}}
\def\mm{\mathfrak{m}}
\def\Gal{\text{Gal}}
\def\char{\text{char}}
\def\irr{\text{irr}}

\title{Assignment 1}
\author{}
\date{}
\begin{document}

\maketitle

\noindent\textbf{Q1:}
Consider the field extension $F=\QQ(\sqrt{2},\sqrt[3]{2},\sqrt[4]{2},\cdots)$. Clearly, $F/\QQ$ is an algebraic extension and $F\subsetneq \overline{\QQ}$ since $i$ is not in $F$. Given any positive integer $n$, we see that $\QQ(\sqrt[n+1]{2})\subset F$. The minimal polynomial of $\sqrt[n+1]{2}$ is $x^{n+1}-2\in \QQ[x]$ --- the irreducibility is given by Eisenstein's Criterion taking prime $p=2$. Hence $$[F:\QQ] \geq [\QQ(\sqrt[n+1]{2}): \QQ] = n+1 >n.$$
Therefore, $F$ is infinite dimensional over $\QQ$.

\bigskip

\noindent\textbf{Q2:}
Suppose that $F(\aa)\not= F(\aa^3)$. Clearly, $F(\aa)/F(\aa^3)$ is a finite extension. We see that  $\aa$ is a root of the polynomial $x^3-\aa^3\in F(\aa^3)[x]$, then the minimal polynomial of $\aa$ over $F(\aa^3)$ divides $x^3-\aa^3$. But $\aa$ is not in $F(\aa^3)$, so the minimal polynomial of $\aa$ have degree 2 or 3 and hence $[F(\aa):F(\aa^3)]=2$ or 3. But then, 
\begin{align*}
    [K:F]& =[K:F(\aa)][F(\aa):F(\aa^3)][F(\aa^3):F] \\
    & =2[K:F(\aa)][F(\aa^3):F]  \mbox{ or } 3[K:F(\aa)][F(\aa^3):F].
\end{align*}
But this contradicts to the assumption $[K:F]$ is relatively prime to 6. Hence we must have $F(\aa)=F(\aa^3)$.

\bigskip

\noindent\textbf{Q3:} 
Consider $F=\QQ, K=\QQ(\sqrt{2}), L=\QQ(\sqrt[4]{2})$. We insist real roots so they are subfields of $\RR$. We claim that $L/K$ and $K/F$ are normal but $L/F$ is not. 

The minimal polynomial of $\sqrt{2}$ over $\QQ$ is clearly $x^2-2\in \QQ[x]$, whose roots are $\pm \sqrt{2}$. But $K$ contains both $\pm\sqrt{2}$ and so is the splitting field of $x^2-2$. Therefore, $K/F$ is normal. Similarly, the minimal polynomial of $\sqrt[4]{2}$ over $\QQ(\sqrt{2})$ is $x^2-\sqrt{2}$, whose roots are $\pm\sqrt[4]{2}$ both lying in $\QQ(\sqrt[4]{2})$. Hence $L$ is the splitting field $x^2-\sqrt{2}$ and so $L/K$ is normal.

On the other hand, the minimal polynomial of $\sqrt[4]{2}$ over $\QQ$ is $x^4-2$ --- the irreducible is checked by Eisenstein's Criterion with prime $p=2$. But the roots are then $$\sqrt[4]{2},-\sqrt[4]{2},i\sqrt[4]{2},-i\sqrt[4]{2}.$$ But $i\sqrt[4]{2}$ is not in $E$, then $x^4-2$ cannot fact completely in $\QQ(\sqrt[4]{2})[x]$. So $L/F$ is not normal.

\bigskip

\noindent\textbf{Q4:}
Let $F$ be perfect and $E/F$ an algebraic extension. Let $f(x) \in E[x]$ be an irreducible polynomial. Assume $\aa$ is a root of $f$ in an algebraic closure $\overline{F}$ ---  note that $\overline{F}$ is also an algebraic closure of $E$ hence we definitely can find such a root in $\overline{F}$. Let $f'(x)\in F[x]$ be the minimal polynomial of $\aa$ over $F$. Since $F$ is perfect, $f'$ has no repeated root. Regarding $f'$ as a polynomial in $E[x]$, we see $f$ divides $f'$ and hence has no repeat root as well. Therefore, $f$ is separable and $E$ is perfect.


As an counter example, the function field $\FF_2(t)$ is not perfect --- $\FF_2(t^{1/2})/\FF_2(t)$ is not a separable extension. The main difference is that we cannot find a root of a polynomial, say $X^2-t \in \FF_2(t)[X]$, in $\overline{\FF_2}$.

\bigskip

\noindent\textbf{Q5:} Assume $[\overline{F}:F]$ is finite. Then $\overline{F}$ is also finite and has $|F|^{[\overline{F}:F]}$ elements. Consider the polynomial $$f(x)=1+\prod_{a\in\overline{F}} (x-a) \in \overline{F}[x].$$ This is a well define polynomial since it is a product of finitely many terms. But $f(\aa)=1$ for every element $\aa\in \overline{F}$, in other words, $f$ has no root over $\overline{F}$. This contradicts to the definition of $\overline{F}$.


\bigskip

\noindent\textbf{Q6:}
Take any element $\aa\in E$. The minimal polynomial $f_K$ of $\aa$ over $K$ is purely inseparable, namely, has one root only. But the minimal polynomial $f_F$ of $\aa$ over $F$ divides $f_K$ (by treating $f_K$ as a polynomial in $F[x]$) and hence has one root only as well. In other words, $f_F$ is purely inseparable and so $E/F$ is purely inseparable.

\bigskip

\noindent\textbf{Q7:} I happen to have a copy of Hungerford's algebra in my hand. The problem is indeed on page 256. I will follow the hint.

We want to show $K^{\Aut(K(x)/K)}=K$, where $x$ is an indeterminate and $K$ is an infinite field, by breaking it into several small claims.

Let $t\in K(x)$ be in the lowest form $\frac{p(x)}{q(x)}$ with $q(x)\not=0$ and $p(x),q(x)\in K[x]$.

Claim 1: $p(X)-tq(X)\in K(t)[X]$ ($X$ is another indeterminate not $x$) is irreducible and has $x$ as a root.


Proof of Claim 1: Since $K[t]$ is a PID and has $K(t)$ as its fraction field, Gauss's Lemma says $p(X)-tq(X)$ is irreducible in $K(t)[X]$ if and only if it is irreducible in $K[t][X]$. But $(K[t])[X]=(K[X])[t]$ and $p(X)-tq(X)$ is linear in $(K[X])[t]$ and thus irreducible. Therefore,  $p(X)-tq(X)$ is irreducible over $K(t)$. Moreover, $p(x)-tq(x) =  p(x)-\frac{p(x)}{q(x)}q(x)=0$, so $x$ is a root.



\bigskip
Claim 2:  The degree of $p(X)-tq(X)\in K(t)[X]$ as a polynomial in $X$ with coefficients in $K(t)$ is the maximum of the degrees of $p(x)$ and $q(x)$. And so $[K(x):K(t)]=\max\{\deg(p),\deg(q)\}$.


Proof of Claim 2: Let $n=\max\{\deg(p),\deg(q)\}$. Then $p(x)=a_nx^n+\mbox{ (lower degree terms)}$ and $q(x)=b_nx^n+\mbox{ (lower degree terms)}$ and at least one of $a_n,b_n$ is nonzero. Clearly, $\deg(p(X)-tq(X))\leq n$ and the coefficient of $X^n$ is then $a_n-tb_n$. Since $t\in K(x)$ but $t\notin K$, it follows that $a_n-tb_n\not=0$ and so $\deg(p(X)-tq(X))=n$. Note $p(X)-tq(X)$ is the minimal polynomial of $x$ over $K(t)$ and hence $[K(x):K(t)]=\deg(p(X)-tq(X))=\max\{\deg(p),\deg(q)\}$.


\bigskip
Claim 3: If $E\not=K$ is an indeterminate field, then $[K(x):E]$ is finite.


Proof of Claim 3: Since $E\not=K$, we can find a rational function $t=\frac{p}{q}\in (K(x)\cap E) \setminus K$. But then $K(t)\subset E$ and $[K(x):K(t)]= \max\{\deg(p),\deg(q)\}$. Thus $[K(t):E]$ is a finite extension.



\bigskip
Now we define a map $\phi: K(x)\to K(x), f(x)\mapsto f(\frac{ax+b}{cx+d})$, where $a,b,c,d\in K$. If $ad-bc=0$, then $\frac{ax+b}{cx+d}=\aa\in K$ and $\phi$ is an evaluation map at $\aa$ not well defined on $K(x)$ since $\aa$ can be a singular point of $f(x)$. We want to exclude this case.

Claim 4: $\phi$ is a $K$-map when $ab-bc\not=0$. Moreover, $K(x)=K(\frac{ax+b}{cx+d})$ and $\phi$ is indeed an automorphism.


Proof if Claim 4: It is straightforward to check the $\phi$ is a $K$-map. Let $f,g\in K(x)$, then \begin{align*}
    \phi((f+g)(x))&=(f+g)(\frac{ax+b}{cx+d})=f(\frac{ax+b}{cx+d})+g(\frac{ax+b}{cx+d})=\phi(f(x))+\phi(g(x)),\\
    \phi((fg)(x))&=(fg)(\frac{ax+b}{cx+d})=f(\frac{ax+b}{cx+d})g(\frac{ax+b}{cx+d})=\phi(f(x))\phi(g(x)).
\end{align*}
And it is trivial to see that $\phi$ fixes $K$, which are constant functions in $K(x)$. From Claim 2, we see that $[K(x):K(\frac{ax+b}{cx+d})]=\max\{\deg(ax-b),\deg(cx-d)\} = 1$ since $ad-bc\not=0$. Hence $K(x)=K(\frac{ax+b}{cx+d})$ and the surjectivity from the equality $\im(\phi)=K(\frac{ax+b}{cx+d})$ implies $\phi$ is an automorphism indeed.



\bigskip
Claim 5: If $\phi\in \Aut(K(x)/K)$, then $\phi$ has the from as in Claim 4. 


Proof of Claim 5: Let $f(x)\in K(x)$ be in the lowest form $f(x)=\frac{\sum_{i=0}^n a_ix^i}{\sum_{j=0}^m b_j x^j}$. Note that 
$$\phi(f(x))==\frac{\phi(\sum_{i=0}^n a_ix^i)}{\phi(\sum_{j=0}^m b_j x^j)}=\frac{\sum_{i=0}^n a_i\phi(x^i)}{\sum_{j=0}^m b_j \phi(x^j)}=f(h(x)),$$ where $h(x)=\phi(x)=\frac{p(x)}{q(x)}\in K(x)$ is the lowest form where $p(x),q(x)\in K[x]$. But then $$1=[K(x):K(h(x))]= \max\{\deg(p),\deg(q)\}.$$ Hence $p,q$ have degree less or equal to 1 and so $\phi$ has the form in Claim 4. But if $ad-bc=0$, then $h(x)=\frac{ax+b}{cx+d}$ is a constant and hence $\phi$ is an evaluation map but not an automorphism. So we have $ad-bc\not=0$.


\bigskip
CLaim 6: $\Aut(K(x)/K)\cong \mbox{PGL}_2(K)$.


Proof of Claim 6: We already have a map $\mbox{GL}_2(K)\to \Aut(K(x)/K)$ base on Claim  4 \& 5. What left is to determine the kernel. If $\phi$ is the identity map, then $\frac{ax+b}{cx+d}= x$ or $ax+b=cx^2+dx$ and the only possibility is $b=d=0, a=c\not=0$ by arguing about the degree. So the kernel is of the form $\begin{pmatrix}
    a & 0\\
    0 & a
\end{pmatrix}=aI$ where $a$ can be any nonzero element in $K$ hence $\Aut(K(x)/K)\cong \mbox{PGL}_2(K)$. 

\bigskip

Claim 7: $K^{\Aut(K(x)/K)}=K$.

Proof of Claim 7: Clearly, $K^{\Aut(K(x)/K)}$ is a field. Suppose $K^{\Aut(K(x)/K)}=E\not= K$. By Claim 3, $K^{\Aut(K(x)/K)}/E$ is a finite extension. And then  $\Aut(K(x)/K))=\Aut(K(x)/E)$. It follows that  $$|\Aut(K(x)/K)|=|\Aut(K(x)/E|\leq [K(x):E].$$
However, $K$ is an infinite field and so $\mbox{PGL}_2(K)\cong \Aut(K(x)/K)$ is also infinite. We reach a contradiction.



\end{document}