\documentclass[12pt,reqno]{article}
\usepackage{fullpage}
\usepackage{amsfonts}
\usepackage{amsthm}
\usepackage{amssymb}
\usepackage{times}
\usepackage{graphicx}
\usepackage{mathtools}
\usepackage{pgfplots}
\vfuzz=2pt

\pgfplotsset{compat=1.17}


\DeclarePairedDelimiter\ceil{\lceil}{\rceil}
\DeclarePairedDelimiter\floor{\lfloor}{\rfloor}

% some "funny lines" referred to later:
\newtheorem{theorem}{Theorem}[section]
\newtheorem{corollary}[theorem]{Corollary}
\newtheorem{lemma}[theorem]{Lemma}
\newtheorem{proposition}[theorem]{Proposition}
{\theoremstyle{remark}\newtheorem*{remark}{Remark}}
\theoremstyle{definition}
\newtheorem{definition}[theorem]{Definition}
\newtheorem{example}[theorem]{Example}
\newtheorem{innercustomgeneric}{\customgenericname}
\providecommand{\customgenericname}{}
\newcommand{\newcustomtheorem}[2]{%
  \newenvironment{#1}[1]
  {%
   \renewcommand\customgenericname{#2}%
   \renewcommand\theinnercustomgeneric{##1}%
   \innercustomgeneric
  }
  {\endinnercustomgeneric}
}
\newcustomtheorem{solution}{Solution}

\newcommand{\ccc}{\mathcal{C}}
\newcommand{\gal}{\mbox{Gal}}
\newcommand{\ta}[1]{\langle #1 \rangle}
\newcommand{\cc}{\mathbb{C}}
\newcommand{\ff}{\mathbb{F}}
\newcommand{\qq}{\mathbb{Q}}
\newcommand{\zz}{\mathbb{Z}}
\newcommand{\im}{\mbox{im}}



\begin{document}

\title{Homework 2}
\author{LU Junyu}

\begin{solution}{1}
	We already know $\qq(\sqrt{2},\sqrt{3})/\qq$ is Galois with its Galois group isomorphic to $\zz/2\zz\times \zz/2\zz$ generated by $\phi$ and $\psi$, where $\phi(\sqrt{2})=-\sqrt{2},\phi(\sqrt{3})=\sqrt{3}$ and $\psi(\sqrt{2})=\sqrt{2}, \psi(\sqrt{3})=-\sqrt{3}$.

	\begin{enumerate}
		\item[(a).] Suppose $a=c^2$ for some $c\in F=\qq(\sqrt{2},\sqrt{3})$. Then $\psi(a)=\psi((2+\sqrt{2})(3+\sqrt{3}))=(2+\sqrt{2})(3-\sqrt{3})$. Consider the Galois extension $F/\qq(\sqrt{2})$, whose Galois group is $\{1,\psi\}$. Then
		      \begin{align*}
			      N_{F/\qq(\sqrt{2})}(a) & =a\phi(a)=(2+\sqrt{2})(2+\sqrt{3})(2+\sqrt{2})(3-\sqrt{3})=6(2+\sqrt{2})^2 \\
			                             & =N_{F/\qq(\sqrt{2})}(c^2)=(N_{F/\sqrt{2}}(c))^2.
		      \end{align*}
		      This implies $6(2+\sqrt{2})^2$, and hence 6, is a square in $\qq(\sqrt{2})$, namely,
		      $$6=(a+b\sqrt{2})^2=a^2+2b^2+2ab\sqrt{2}$$ for some $a,b\in \qq$. But this cannot be true: if $a=0$, then $6=2b^2$ or $b^2=3$, which is not possible for any $b\in \qq$; if $b=0$, then $6=a^2$, which is not possible for any $a\in \qq$ either; if $a\not=0\not=b$, then it implies $\sqrt{2}$ is rational, which is also not possible. Hence, $a$ cannot be a square in $F$.
		\item[(b).] Since $a$ is not a square in $F$ but $a\in F$, it is easily seen that $[F(\sqrt{a}=\alpha):F]=2$ since the minimal polynomial of $\sqrt{a}=\alpha$ over $F$ is $x^2-a\in F[x]$. Hence $$[F(\alpha):\qq]=[F(\alpha):F][F:\qq]=2\cdot 4=8.$$
		      Apparently, $E=\qq(\alpha)\subset F(\alpha)$. For the reverse inclusion, note that $$a=\alpha^2=(2+\sqrt{2})(3+\sqrt{3})=6+3\sqrt{2}+2\sqrt{3}+\sqrt{6}\in E$$
		      and so $b=3\sqrt{2}+2\sqrt{3}+\sqrt{6} = a-6\in E$ and then
		      $c=(b^2-36)/12=\sqrt{2}+\sqrt{3}+\sqrt{6}\in E$
		      and then $d=b-2c=\sqrt{2}-\sqrt{6}\in E$ and then $e = (8-d^2)/4=\sqrt{3} \in E $ and then $\sqrt{2}=(c+d-2)\in E$. And so $F=\qq(\sqrt{2},\sqrt{3})\subset E$ and $F(\alpha)\subset E$. Therefore, $E=F(\alpha)$ and $[E:\qq]=8$. Keep squaring $\alpha$ and removing rationals, we get a polynomial rational $\alpha^8-24\alpha^6+133\alpha^4-288\alpha^2+144=0$. Since $[E:\qq]=\deg$ of the minimal polynomial of $\alpha$ over $\qq$, the minimal polynomial must be $f(x)=x^8-24x^6+144-288x^2+144\in \qq[x]$. $f(x)$ is irreducible and has 8 roots which can be checked directly that they are $\pm\sqrt{(2\pm\sqrt{2})(3\pm\sqrt{3})}$.
		\item[(c).] Since $\qq$ is perfect, all finite extensions are separable and so we need to check $E/\qq$ is normal, which can be done via checking that all the roots of $f(x)$ lie in $E$ and $E$ is a splitting field of $f(x)$ and so is normal. This is straightforward.
		      \[\alpha\sqrt{(2-\sqrt{2})(3+\sqrt{3})} =\sqrt{(2+\sqrt{2})(3+\sqrt{3})(2-\sqrt{2})(3+\sqrt{3})}=(3+\sqrt{3})\sqrt{2} \in E\]
		      and therefore, $\sqrt{(2-\sqrt{2})(3+\sqrt{3})} =  (3+\sqrt{3})\sqrt{2}/ \alpha\in E$.
		      Similarly, \[\alpha\sqrt{(2+\sqrt{2})(3-\sqrt{3})} =\sqrt{(2+\sqrt{2})(3+\sqrt{3})(2+\sqrt{2})(3-\sqrt{3})}=(2+\sqrt{2})\sqrt{6} \in E\]
		      and therefore, $\sqrt{(2+\sqrt{2})(3-\sqrt{3})} = (2+\sqrt{2})\sqrt{6}/ \alpha\in E$.
		      Finally,  \[\alpha\sqrt{(2-\sqrt{2})(3-\sqrt{3})} =\sqrt{(2+\sqrt{2})(3+\sqrt{3})(2-\sqrt{2})(3-\sqrt{3})}=2\sqrt{3} \in E\]
		      and therefore, $\sqrt{(2-\sqrt{2})(3-\sqrt{3})} = 2\sqrt{3}/ \alpha\in E$. The rest of them are the negatives of these four and hence also are in $E$. So indeed, $E$ is a splitting field of $f(x)$ over $\qq$ and normal.
		\item[(d).] Since $E=\qq(\alpha)$, an element in the Galois group is entirely determined by its action on $\alpha$. For convenience, we write $\beta = \sqrt{(2-\sqrt{2})(3+\sqrt{3})}$ and $\gamma=\sqrt{(2+\sqrt{2})(3-\sqrt{3})}$. Then from part (c), wee that $\alpha\beta=\sqrt{2}(3+\sqrt{3})$ and $\alpha\gamma=\sqrt{6}(2+\sqrt{2})$. Now $\sigma(\alpha^2)=\beta^2$, namely, $$\sigma(\alpha^2)=\sigma((2+\sqrt{2})(3+\sqrt{3}))=(2-\sqrt{2})(3+\sqrt{3}).$$ Note that $\alpha^2=a\in F$ and $\sigma|_F\in \gal(F/\qq)=\ta{\phi,\psi}$. So we must have $\sigma|_F=\phi$. More precisely, consider the canonical surjection from the fundamental theorem on Galois theory
		      $$\pi:\gal(E/\qq)\to \gal(F/\qq), \tau\mapsto\tau|_L.$$
		      Then $\ker(\pi)=\gal(E/F)$ and $\pi(\sigma)=\phi$. And so $$\sigma(\alpha\beta)=\sigma(\sqrt{2}(3+\sqrt{3}))=-\sqrt{2}(3+\sqrt{3})=-\alpha\beta.$$ It follows immediately that $\sigma(\beta)=-\alpha$ and so $\sigma$ is of order 4.
		\item[(e).] Similar arguments apply on $\tau$. We see that $\tau(\alpha^2)=\gamma^2$, namely, \[\tau(\alpha^2)=\tau((2+\sqrt{2})(3+\sqrt{3}))=\gamma^2 =(2+\sqrt{2})(3-\sqrt{3}).\] Hence $\pi(\tau)=\psi$. And then $$\tau(\alpha\gamma)=\tau(\sqrt{6}(2+\sqrt{2}))=-\sqrt{6}(2+\sqrt{2})=-\alpha\gamma.$$ Therefore, $\tau(\gamma)= -\alpha$ and $\tau$ is of order 4. If $\gal(E/\qq)$ is cyclic, it cannot have two elements of order $4$. Thus $\gal(E/\qq)$ is not cyclic. Now consider $\ta{\sigma}$, which is a cyclic group of order 4. We note that $\sigma^2(\alpha)=-a$ and $\sigma^3(\alpha)=-\beta$. Since $\ta{\sigma}$ is of index 2 and $\tau\notin \ta{\sigma}$, $\gal(E/\qq)$ is partitioned by $\ta{\sigma}$ and $\tau\ta{\sigma}$. Hence, $\gal(E/\qq)=\ta{\sigma,\tau}$. Note that $\sigma^2(\alpha)=-\alpha=\tau^2(\alpha)$, so $\sigma^2=\tau^2$. Note that $$\sigma\tau(\gamma)=\sigma(-\alpha)=-\sigma(\alpha)=-\beta$$ and $$\tau\sigma^3(\beta)=\tau\sigma^2(-\alpha)=\tau(\alpha)=\gamma.$$ By a similar argument as part (d), we then see that $\tau\sigma^3(\gamma)=-\beta=\sigma\tau(\gamma)$. And hence $\sigma\tau=\tau\sigma^3$, or equivalently, $\tau\sigma = \sigma^{-1}\tau$ by the conjugation of $\sigma^{-1}$.

		      One of the standard presentation (as per Wikipedia) of $Q_8$ is $$\ta{a,b|a^4=e,a^2=b^2,ba=a^{-1}b}.$$ The isomorphism between $\gal(E/\qq)$ and $Q_8$ is then trivially by $\sigma\mapsto a$ and $\tau\mapsto b$.
	\end{enumerate}
\end{solution}

\begin{solution}{2}
	By Eisenstein's Criterion, $f(x)=x^4+px+p\in \qq[x]$ is irreducible for any prime $p$. So we can safely apply the classification about the Galois groups of quartics. The resolvent of $f(x)$ is then
	\[r(x)=x^3-4px-p^2\in \qq[x].\]
	And the discriminant is
	\[\Delta=-4(-4p)^3-27(-p^2)^2=p^3(256-27p).\]

	Note that if $p$ is odd, then $p^3\mid \Delta$ but $p^4\nmid \Delta$ since $p\nmid (256-27p)$. So $\Delta$ is never a square in $\qq$ when $p$ is an odd prime. If $p=2$, then $\Delta= 2^3(256-27\cdot 2)=1616$ is not a square in $\qq$ either. Therefore, $\Delta$ is never a square in $\qq$ whatever the prime $p$ is.

	The thing left is to determine the irreducibility of $r(x)$. If $r(x)$ is reducible, then it has a rational root as being a cubic. By the rational root test, if the rational root is $a/b$ in the lowest form, then $b\mid 1$ and $a\mid -p^2$. And so the only possible roots are $\pm1,\pm p,\pm p^2$. But $r(1)=1-4p-p^2 < 0$, so $1$ is never a root. And $r(-1)=1+4p-p^2=5-(p-2)^2$ can never be 0 since $5$ is not a square. Now $$r(p^2)=p^6-4p^3-p^2 > p^2(p^4-4p-1)> p^2(8p-4p-1)>3p^3>0.$$
	And $$r(-p^2)=-p^6+4px^3-p^2=-p^2(p^4-4p+1) < -p^2(p^4-p^2+(p-2)^2) <-p^2<0.$$ And so $\pm p^2$ can never be roots of $r(x)$. Thus we only to check whether $\pm p$ are roots of $r(x)$.
	Note that $$r(p)=p^3-4p^2-p^2=p^2(p-5)$$ has roots $p=0,5$ and $$r(-p)=-p^3+4p^2-p^2=(3-p)p^2$$ has roots $p=0,3$.

	Therefore, when $p\not= 3,5$, $r(x)$ is irreducible. Combining the fact that $\Delta$ is never a square, we see that the Galois group $G_f\cong S_5$.

	If $p=3$, then $r(x)=x^3-12x-9=(x+3)(x^2-3x-3)$. The roots of $r(x)$ are $-3,(3\pm \sqrt{21})/2$. So the splitting field of $r(x)$ is $L=\qq(\sqrt{21})$. The polynomial to be tested to determine the Galois group is $h(x)=(x^2+3x+3)(x^2+3)$. We see that $\sqrt{i\sqrt{3}}$ does not lie in $L$ and so $h(x)$ does not split over $L$ and so $G_f\cong D_4$.

	If $p=5$, then $r(x)=x^3-20x-25=(x-5)(x^2+5x+5)$ with roots $5,(-5+\sqrt{5})/2$. SO the splitting field of $r(x)$ is $\qq(\sqrt{5})$. The polynomial to be tested to determine the Galois group is $h(x)=(x^2+5x+5)(x^2-5)$, which has roots $\pm\sqrt{5}, (-5\pm\sqrt{5})/2$ all in $L$. So $h(x)$ splits over $L$. Therefore, the Galois group $G_f\cong \zz/4\zz$.
\end{solution}

\begin{solution}{3}
	Let $a$ be a real root of $f$ and $b\not=0$ a complex root. Since complex roots appear in pair by conjugation, we see $\overline{b}$ is also a root. Let $L$ be the splitting field of $f$ over $\qq$ and we further require $\qq\subset\overline{\qq}\subset \cc$. Then $L\subset \cc$. Denote the complex conjugation on $\cc$ by $\phi: \cc\to \cc, x\mapsto \overline{x}$. Then $\phi|_L$ is a homomorphism from $L$ to $\cc$ fixing $\qq$. But $L$ as the splitting field is normal and so indeed we have $\phi|_L\in \gal(L/\qq)=G_f$. Since $f$ is irreducible, the Galois group $G_f$ is transitive and so we can find an element $\tau\in G_f$ such that $\tau(a)=b$. But then $\phi|_L$ and $\tau$ do not commute in the sense that $$\tau\phi|_L(a)=\tau(a)=b\not= \phi|_L\tau(a)=\phi|_L(b)=\overline{b}.$$ Therefore, $G_f$ is not an abelian group.

	We cannot drop the assumption that $f$ is irreducible since we need the transitivity of $G_f$. As a counter example, $f(x)=(x-1)(x^2+1)$ and $f(x)$ has the obvious Galois group $G_f\cong \zz/2\zz$ consisting of the identity map and the complex conjugation map.
\end{solution}

\begin{solution}{4}
	I think this is a problem on the lower bound of the Euler function $\varphi(n)$.

	Claim: $\varphi(n)\geq \sqrt{n/2}$.


	With this claim, suppose $E/\qq$ is a finite extension. Then we can find a positive integer $n$ such that $[E:\qq] <\sqrt{n/2}$. Then $E$ cannot contain any primitive $m$-th root for any $m>n$. Otherwise, $$[E:\qq] \geq [\qq(\zeta_m):\qq]=\varphi(m) \geq \sqrt{m/2} >\sqrt{n/2} > [E:\qq],$$ which is a contradiction. But there are only finitely many $q$-th primitive root of unity for $q\leq n$.


	Proof of Claim: Let $n=p_1^{e_1}\dots p_k^{e_k},$ where $p_i$ are distinct primes and $e_i\geq 1$. Then we have
	\begin{align*}
		\varphi(n)^2 & =  (n \prod_{i=1}^k(1-\frac{1}{p_i}))(\prod_{i=1}^k p_i^{e_i-1}(p_i-1)) \\
		             & = n \prod_{i=1}^{k} (1-\frac{1}{p_i})p_i^{e_i-1}(p_i-1)                 \\
		             & \geq n \prod_{i=1}^k \frac{(p_i-1)^2}{p_i}                              \\
		             & \geq \frac{n}{2},
	\end{align*}
	because if $p_i=2$, then ${(p_i-1)^2}/{p_i}= 1/2$ and if $p_i\geq 3$, then ${(p_i-1)^2}/{p_i}\geq 1$.
	Therefore, we have the desired inequality.
\end{solution}

\begin{solution}{5}
	One direction shall be easy, suppose $n$ is not square free, say $n=p^2q$ for some prime $p$. Let $\zeta_n$ be a primitive $n$-th root. Then $\zeta_n^{pq}$ is a primitive $p$-th root of unity. Hence $$\Phi_p(\zeta_n^{pq}) = 1+ \zeta_n^{pq} +(\zeta_n^{pq})^2+\dots+(\zeta_n^{pq})^{p-1} =0.$$ Multiplying  both sides by $\zeta_n$, we get \[  \zeta_n + \zeta_n^{pq+1} +\zeta_n^{2pq+1}+\dots+\zeta_n^{(p-1)pq+1}=0.\] Note that $\gcd(p^2q, ipq+1)=1$, so each element $\zeta_n^{pq+1}$ is also a primitive $n$-th root. Hence we get a nontrivial linear dependence relation between the $n$-th primitive roots. So they cannot form a basis.

	For the other direction, I did not really find a way to apply the normal basis theorem. Maybe you can explain to me after the presentation.
\end{solution}

\begin{solution}{6}
	One direction is clear. Let $\{\sigma(a):\sigma\in G\}$ be a normal basis for some $a\in L$. Then we claim that $\{a\}$ is a basis of $L$ as a $KG$-module and so $L$ is a cyclic hence free $KG$-module. We only need to prove the claim. Since $\{\sigma(a):\sigma\in G\}$ is a $K$-basis for $L$, any element $x\in L$ can be expressed as $x=\sum_{\sigma\in G} k_{\sigma} \sigma(a)$, where $k_\sigma\in K$. But then take the element $s_x=\sum_{\sigma\in G} k_\sigma \sigma\in KG$. We see that $s_xa=(\sum_{\sigma\in G} k_\sigma)a = \sum_{\sigma\in G} k_{\sigma} \sigma(a) = x$. So $\{a\}$ is a spanning set. To see $\{a\}$ is linearly independent, we assume there is a linear dependence relation $$0=(\sum_{\sigma\in G} k_\sigma \sigma)a=\sum_{\sigma\in G} k_{\sigma} \sigma(a).$$ But $\{\sigma(a):\sigma\in G\}$ is a $K$-basis, then we must have $k_\sigma=0$ for all $\sigma\in G$ and so  $(\sum_{\sigma\in G} k_\sigma \sigma)=0$.


	For the other direction, suppose $L$ is a free $KG$-module, namely, $L=\ta{a_1}\oplus \dots\oplus \ta{a_k}$, where $\ta{a_i}\cong KG$ for each direct summand. Note that $KG$, hence $\ta{a_i}$, has a $K$-vector space structure with $\dim_K(KG)=|G|=[L:K]$. So by counting the dimension, $L$ is necessarily a cyclic $KG$-module, say, $L=\ta{a}$. Then $\{\sigma(a): \sigma\in G\}$ is a normal basis. Again by counting dimension, we only need to show it is a spanning set. Now since $L=\ta{a}$, for any $x\in L$ we can find $s_x=\sum_{\sigma\in G} k_\sigma\in KG$, where $k_\sigma\in K$, such that $x=(\sum_{\sigma\in G} k_\sigma)a =\sum_{\sigma\in G} k_{\sigma} \sigma(a)$. In other words, $x$ is a $K$-linear combination of  $\{\sigma(a): \sigma\in G\}$.

\end{solution}

\begin{solution}{7}
	Let $[L:K]=m$. Then $L=\ff_{q^m}$ (up to isomorphism), where $q=p^n$, and the Galois group is cyclic and generated by the Frobenius map $\phi: L\to L, x\mapsto x^q$.
	% And hence for $a\in L$, the formula for norm is then \[N_{L/K} = \prod_{i=0}^{m-1} \phi^{i}(a) = \prod_{i=0}^{m-1} a^{q^{i}} = a^{_{i=0}^{m-1} q^{i}} = a^{\frac{q^m-1}{q-1}}.\]
	% If $N_{L/K}(a)=1$, then $a$ is a root of the polynomial $x^{q^m-1}-1\in $
	By Hilbert 90, $N_{L/K}(\alpha)=1$ if and only if we can find $a\in L$ such that $\alpha=\phi(a)/a$. Hence $$\ker(N_{L/K})=\{\phi(a)/a: a\in L\}.$$ To find the cardinality of $\{\phi(a)/a: a\in L\}$, we need to eliminate the duplicity. Now consider the map $$f: L^*\to L^*, x\mapsto \phi(x)/x=x^p/x.$$ Since both $phi$ and inverse map are group homomorphisms, so is $f$. Hence by group isomorphism theorem, we have $$L^*/\ker(f) \cong \im(f) = \{\phi(a)/a: a\in L\}.$$ Now $x\in \ker(f)$ if and only if $\phi(x)=x$, namely, $x$ is fixed by $\phi$ and then by the Galois group, so if and only if $x\in K^*$. And hence $$|\ker(N_{L/K})|=  |L^*/\ker(f)| = |L^*|/|K^*| = (q^m-1)/(q-1),$$ where $m=[L:K]$ and $q=p^n$.
\end{solution}

\end{document}
