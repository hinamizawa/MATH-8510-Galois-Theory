\documentclass[12pt,reqno]{amsart}
\usepackage{fullpage}
\usepackage{amsfonts}
\usepackage{amssymb}
\usepackage{times}
\usepackage{graphicx}
\vfuzz=2pt

% some "funny lines" referred to later:
\newtheorem{thm}{Theorem}[section]
\newtheorem{cor}[thm]{Corollary}
\newtheorem{lem}[thm]{Lemma}
\newtheorem{prop}[thm]{Proposition}
{ \theoremstyle{remark}\newtheorem*{remark}{Remark} }


\def\BB{\mathcal{B}}
\def\OO{\mathcal{O}}
\def\ZZ{\mathbb{Z}}
\def\NN{\mathbb{N}}
\def\CC{\mathbb{C}}
\def\QQ{\mathbb{Q}}
\def\RR{\mathbb{R}}
\def\FF{\mathbb{F}}
\def\Tr{\mbox{Tr}}
\def\sgn{\mbox{sgn}}
\def\nm{\mbox{N}}
\def\tk{\theta_K}
\def\Hom{\mbox{Hom}}
\def\Aut{\mbox{Aut}}
\def\Inn{\mbox{Inn}}



\begin{document}

\title{MATH 8510 Galois Theory}
% or if you want, simply \title{Title of the article}
\author{LU Junyu}
%\email{luj9@myumanitoba.ca}

% \begin{abstract}
% Here's a \LaTeX\ template.
% \end{abstract}

\maketitle

\section{Week 1}
\label{Week 1}

\noindent Let's agree on some facts and conventions from elementary abstract algebra, in particular those with polynomial rings before we dig into Galois theory.

A ring is always commutative with multiplicative identity 1 unless otherwise stated. $R^*$ is the multiplicative group of units in $R$ and $R^\times = R\setminus \{0\}$. We can use these two notations interchangeably when $R$ is a field.

\smallskip

Let $F$ be a field. A polynomial ring $F[X]$ with an indeterminate $X$ is an $F$-vector space with basis $1,X,X^2,...,X^n,...$ with the multiplication $$(\sum_i a_iX^i)(\sum_j b_j X^j) =  \sum_k (\sum_{i+j=k}a_ib_j)X_k,$$ where $X^0$ is defined to be 1. Alternatively, we can identify $R[X]$ with $$R^{(\NN)}=\{(a_i): a_i\in R, a_i=0 \mbox{ for all but finitely many }i\in\NN\}$$ in an obvious way. The degree function has the following properties:
\begin{enumerate}
    \item $\deg(f+g) \leq \max(\deg f,\deg g),$
    \item $\deg(fg)=\deg f+\deg g.$
\end{enumerate}

\begin{thm}
Let $F$ be a commutative ring. Then $F[X]$ is a PID if and only if $F$ is a field.
\end{thm}

Hence or otherwise $\ZZ[X]$ is not a PID. Indeed, $\langle 2, X\rangle$ is an example of an ideal that cannot be generated by a single polynomial. $K[X,Y]$ is not a PID as $\langle X, Y \rangle$ is not principal.

\begin{thm}
    An ideal in a PID is prime if and only if it is maximal.
\end{thm}

\begin{thm}[Gauss's Lemma]
    A polynomial $f(X)\in \ZZ[X]$ is irreducible if and only if it is irreducible over $\QQ[X]$.
\end{thm}

\begin{thm}[Eisenstein Criterion]
    Let $f(X) = a_0+a_1X+...+a_nX^n\in \ZZ[X]$ be a polynomial over integers with $a_n\not= 0$. Suppose that there exists a prime $p$ such that \begin{enumerate}
        \item  $p\nmid a_n$, 
        \item  $p\mid a_i$  for  $i=0,1,...,n-1$,
        \item  $p^2\nmid a_0$.
    \end{enumerate}
    Then $f(X)$ is irreducible over $\ZZ[X]$.
\end{thm}



% \newpage
% \begin{thebibliography}{99} % don't worry about the 99

% % \bibitem{ExamplePaper}
% % E.~Edelman, ``The probability that a random real Gaussian matrix has $k$ real eigenvalues, related distributions, and the circular law'', {\em J.~Multivariate~Anal.} {\bf 60} (1997), no.~2, 202--232.

% % \bibitem{ExampleBook}
% % H.~L.~Montgomery and R.~C.~Vaughan, {\em Multiplicative Number Theory I: Classical Theory}, Cambridge University Press (2007).

% \end{thebibliography}

\end{document}
